\documentclass[12pt]{article}

\usepackage{amsmath}

\usepackage{microtype}

\usepackage{graphicx}

\usepackage{hyperref}

\usepackage[utf8]{inputenc}

\usepackage{listings}

\usepackage{matlab-prettifier}
% % voor code syntax highlighting
% \usepackage{minted}

% beter font
\usepackage[T1]{fontenc}
\usepackage{helvet}
\setcounter{secnumdepth}{5}
\setcounter{tocdepth}{5}
\renewcommand{\familydefault}{\sfdefault}
\graphicspath{{imagesanalyse/}}
\begin{document}
\begin{titlepage}
    \author{Xander Vandooren}
    \title{Energie-analyse en
    forecasting}
\end{titlepage}
\pagenumbering{gobble}
\maketitle
\newpage
\tableofcontents
\newpage
\pagenumbering{arabic}
\section{Week 1:}
\subsection{Inleiding:}
\subsubsection{Internet of Energy:}
\paragraph{Data science (analyse) in Energie.}
\begin{itemize}
    \item Optimalisatie van energie productie met HEB
    \item Nationaal energiemarkt-werking
    \item Voorspelling van toekomstige E-vraag om productie en distributie aan te passen. 
    \item Herkenning van energieverbruik patronen om aanbods op maat te maken en fouten te detecteren.
    \item Analyse van werking/status/fouten bij aparaten
    \item Data gedreven DSM
\end{itemize}
\paragraph{Weer voorspelling:}
Waarom:\begin{itemize}
    \item Optimalisatie van energie-stromen 
    \item Verzeker de decarbonisatie van de energiesector 
    \item Kosten te verminderen 
    \item Business decissions based on forecasting 
    \item Beperken downtime en verbeteren van veiligheid en productiviteit 
    \item Voorspel veranderingen in de vraag demand 
    \item Trade-off: middelen - betrouwbaarheid - kosten 
    \item Internet of Energy 
\end{itemize}
\textbf{Use-case:}\begin{itemize}
    \item Voorspelling van energieverbruik 
    \item operatie van een gebouw 
    \item fout detectie 
\end{itemize}
\subparagraph{Voorspelling van energieverbruik:}
Voorbeeld:\begin{itemize}
    \item Input:\begin{itemize}
        \item W-PV 
        \item Weer informatie 
        \item seizoen 
        \item Uur van de dag 
        \item WKK-status 
        \item Time of day (day-ahead)
    \end{itemize}
    \item Model (machine learning, deep learning,...)
    \item output:\begin{itemize}
        \item Afname elektriciteit
    \end{itemize}
\end{itemize}
\subsubsection{Tools:}
Type problemen:\begin{itemize}
    \item Description:\begin{itemize}
        \item Patronen of trends identificatie 
        \item Wat beinvloedt energieverbruik 
        \item Wat beinvloed energieopwekking
    \end{itemize}
    \item Voorspelling:\begin{itemize}
        \item Opwekking forecast 
        \item Optimale scheduling
    \end{itemize} 
    \item Detectie:\begin{itemize}
        \item Zijn er fouten in het systeem
    \end{itemize}
    \item Dynamische evaluatie:\begin{itemize}
        \item Welke subsidies of tarieven stimuleren de efficientie en betrouwbaarheid
    \end{itemize}
\end{itemize}
Bedrijven in energy analytics:\begin{itemize}
    \item 3E 
    \item N-Side 
    \item Energis 
    \item Oktow 
    \item Actemium
    \item Capgemini 
    \item Teradata
    \item \dots 
\end{itemize}
\subsubsection{Data verkenning en data wrangling:}
\section{Week 2:}
\subsection{Statistiek:}
\begin{itemize}
    \item Beschrijvende statistiek: (wij werken in de les met dit soort statistiek)\begin{itemize}
        \item  Methoden om samen te vatten en te beschrijven de belangrijkste kenmerken van een dataset zoals centrale tendens, variabiliteit en distributie.
    \end{itemize}
    \item Inferentiele statistiek:\begin{itemize}
        \item omvat van trekken van conclusies van de populatie.
    \end{itemize}
\end{itemize}
\subsubsection{verschillende maten voor meten/berekenen}
\paragraph{Centrummaten}
Is een maat dat probeer een hele dataset te beschrijven. De waarde
vertegenwoordigt het midden of de center van de dataset distributie.
\subparagraph{Gemiddelde}
Het is de opsomming van twee of meerdere waraden gedeeld op de aantal
waarden die opgesomd worden.\begin{itemize}
    \item Meest gebruikte maat van centrale tendens 
    \item Sterk beinvloedt door uitschieters 
    \item Wordt gezien zoals een balanspunt
\end{itemize}
Gemiddele $\rightarrow$ balanspunt (kwadratische fout)
\subparagraph{Mode (Nodus)}
De meest voorkomende waarde in een dataset.\begin{itemize}
    \item Maat van centrale tendens 
    \item Niet beinvloedt door uitschieters 
    \item In een dataset kunnen er geen of meerdere mode waarden aanwezig zijn 
    \item Het kan gebruit worden zowel voor kwalitatieve als kwantitatieve data 
\end{itemize}
\begin{itemize}
    \item Indien elke waarde maar 1 keer voorkomt in een dataset is er geen modus 
    \item Indien er waardes zijn die meerdere keren voorkomen is de meest voorkomende de dataset (er kunnen meerdere modus aanwezig zijn)
\end{itemize}
\subparagraph{Mediaan}
Het is de middelste waarde van een dataset. De mediaan komt overeen
met de 50e percentile van een dataset, m.a.w. is het midden van een
dataset met een half van de warden kleiner en de andere half groter dan
de mediaan\begin{itemize}
    \item Als n oneven is, is de mediaan de middelste waarde van de gesorteerde rij 
    \item Als n even is, is de mediaan  het gemiddelde van de twee middelste waarden van de gesorteerde rij 
    \item positie van de median is $\frac{n+1}{2}$
    \item Wordt niet beinvloedt door uitschieters 
\end{itemize}
\subparagraph{Skewness:}
\begin{itemize}
    \item negatively skewed (gemiddelde ligt meer naar links)
    \item Normal (no skew) (alle 3 de waarden liggen op elkaar)
    \item Positively skewed (gemiddelde ligt meer naar rechts)
\end{itemize}
Centrummaten alleen zijn niet genoeg data te beschrijven.
\paragraph{Spreidingsmaten}
\subparagraph{Spreidingsbreedte}
Het is het verschil tussen de grootste en de kleinste warden in de
dataset\begin{itemize}
    \item $x_max$-$x_min$
    \item Sterk gevoelig aan extreme waarden/uitschieters
    \item Geeft geen informatie over de distributie van de data
\end{itemize}
\begin{itemize}
    \item Makkelijk te bepalen 
    \item Neemt maar 2 waarden (min,max)
    \item Sterk gevoelig aan uitschieters
\end{itemize}
\subparagraph{Kwartielafstand (IQR)}
Het wordt ook benoemd zoals middenspreiding, midden 50\%, 4de
spreading of H-spreading. Het wordt gedefinieerd zoals het verschil
tussen de 75e en 25e percentielen van de data.
\begin{itemize}
    \item splitsts een gesorteerde dataset in 4 gelijke delen
    \item Eerste kwartiel is de 25e percentiel
    \item Tweede kwartiel is de 50e percentiel=mediaan
    \item Derde kwartiel is de 75e percentiel
\end{itemize}
\begin{itemize}
    \item Eenvoudig te bepalen 
    \item Vertegenwoordigt de spreiding van de middenste 50\% 
    \item niet gevoelig aan uitschieters
\end{itemize}
\subparagraph{Boxplot}
Boxplot= box en whisker plot
Geeft een 5 getallen samenvatting van een dataset weer.\begin{itemize}
    \item Minimum 
    \item 1ste kwartiel 
    \item mediaan 
    \item 3de kwartiel 
    \item maximum
\end{itemize}
Skewness kan je ook zien op een boxplot.
\subparagraph{Variantie}
Telt de spreiding van de data t.o.v het gemiddelde. Het kan bepaald worden voor een sample ($s^2$) of een populatie ($\sigma^2$)
\subparagraph{Standaardafwijking}
Net zoals de variantie kan bepaald worden voor zowel een sample
(s) of een populatie ($\sigma$). In beide gevallen is de vierkantswortel van
de variantie
\begin{itemize}
    \item moeilijker te bepalen 
    \item Gebasseerd op alle waarden 
    \item Gevoelig aan uitschieters voor kleine datasets
\end{itemize}
\subparagraph{Variatiecoefficient:}
Het wordt gedefineerd zoals de verhouding van de Standaardafwijking tot het gemiddelde. V wordt uigedrukt in percentages.\begin{itemize}
    \item V<5\% $\rightarrow$ heel kleine relatieve variatie 
    \item 5\% < V <10\% $\rightarrow$ kleine relatieve variatie 
    \item V>50\% $\rightarrow$ heel grote variatie
\end{itemize}
\subparagraph{Relatieve spreidingsmaten}
Ze worden gebruikt om de vergelijken te doen van de spreiding van twee of meer
datasets. Relatieve betekent dat zonder grootheden is \begin{itemize}
    \item Beschrijft de positie van een datapunt t.o.v. de rest van de data 
    \item Twee veel voorkomende maten zijn:\begin{itemize}
        \item Percentiel 
        \item Z-score 
    \end{itemize}
    \item Een optie om uitschieters te detecteren
\end{itemize}
\begin{itemize}
    \item Percentiel:\begin{itemize}
        \item Waarde van 0 tot 100 dat de percentage geeft dat gelijk of onder is 
        \item \begin{itemize}
            \item Mediaan is 50e percentiel 
            \item K1 is de 25e percentiel 
            \item K3 is de 75e percentiel
        \end{itemize}
    \end{itemize}
    \item Z-score:\begin{itemize}
        \item Het beschrijft het verband van een datapunt met het gemiddelde
        van de groep. Het wordt gemeten in termen van de
        standaardafwijkingen van het gemiddelde.
        \item Dataset met een absolute waarde van z-score groter dan 3
        Voor heel skewed data distributies, een absolute waarde van z-score
        groter dan 2
    \end{itemize}
\end{itemize}
\subsection{Data Visualisatie}
\begin{figure}[t]
    \includegraphics[width=\textwidth]{Charts.PNG}
    \caption{Vershillende charts voor specifieke situaties}
    \centering
\end{figure}
\begin{itemize}
    \item Data wrangling 
    \item Missing data and outliers 
    \item Features and KPI
\end{itemize}
\begin{itemize}
    \item Histogrammen:\begin{itemize}
        \item 1 variabele
    \end{itemize}
    \item Box plot:\begin{itemize}
        \item 1 variabele
    \end{itemize}
    \item Scatter plots\begin{itemize}
        \item 2 variabelen 
    \end{itemize}
    \item Lijn plots:\begin{itemize}
        \item 2 variabelen
    \end{itemize}
    \item Scatter plot matrix:\begin{itemize}
        \item Indien meer dan 2 variabelen
    \end{itemize}
    \item Heatmaps:\begin{itemize}
        \item Indien meer dan 2 variabelen
    \end{itemize}
\end{itemize}
\subsubsection{Sankey Diagrams}
$\rightarrow$ energie flow
\subsection{Tools voor data Visualisatie}
\begin{itemize}
    \item Commerciele tools:\begin{itemize}
        \item Power-BI, Looker, Tableau, Qlik sense, Google data studio
    \end{itemize}
    \item Eigen tools:\begin{itemize}
        \item Jupyter-voila, Streamlit, Plotly Dash, Flask, Django
    \end{itemize}
\end{itemize}
\subsection{Verkennende data analyse}
\begin{itemize}
    \item Profielaggregatie (jaarlijks, maandelijks, per week, per dag) 
    \item Energieverbruik distributie - duration curve 
    \item Correlatie tussen variabelen
    \item Heatmaps
\end{itemize}
\section{Week 3:}
Hoe meer 
\end{document}