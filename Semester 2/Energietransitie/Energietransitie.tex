\documentclass[12pt]{article}

\usepackage{amsmath}

\usepackage{microtype}

\usepackage{graphicx}

\usepackage{hyperref}

\usepackage[utf8]{inputenc}

\usepackage{listings}

\usepackage{matlab-prettifier}
% % voor code syntax highlighting
% \usepackage{minted}

% beter font
\usepackage[T1]{fontenc}
\usepackage{helvet}
\usepackage{adjustbox}

\setcounter{secnumdepth}{5}
\setcounter{tocdepth}{5}
\renewcommand{\familydefault}{\sfdefault}
\graphicspath{{imagessoft/}}
\begin{document}
\begin{titlepage}
    \author{Xander Vandooren}
    \title{Energietransitie}
\end{titlepage}
\pagenumbering{gobble}
\maketitle
\newpage
\tableofcontents
\newpage
\pagenumbering{arabic}
\section{Wat is de energietransitie?}
De energietransitie is de overgang van onze huidige manier van energie gebruiken naar een zuinigere, slimmere en groenere aanpak. We bouwen mee aan klimaatneutraal Europa tegen 2050: binnen minder dan 30 jaar leven en werken we zonder schadelijk effect op het klimaat. Daarvoor moeten we in onze gebouwen, bedrijven en op de baan afstappen van stookolie, aardgas en benzine of diesel. Zo beperken we de opwarming van de aarde en zetten we in op onze energieonafhankelijkheid en onze welvaart.
\subsection{energieverbruik:}
Energieverbruik blijft stijgen door onder andere elektrificatie. 28\% van de energievraag is afkomstig van airco's en data centers. Elektrische voertuigen, warmte pompen en electrolyse voor waterstof zijn ook grote factoren.
\subsection{Opportuniteit of probleem?}
meer productie dan verbruik. Dit komt meer en meer voor.
\subsubsection{Wat?}
\begin{itemize}
    \item Capactiteitsdiensten (dit is opzich verzadigd en wordt vooral gebruikt voor als kerncentrales ofzo uitvallen of als er onbalans is).
    \item Energiediensten
\end{itemize}
\subsection{Energieopslag als oplossing:}
\subsection{EMS als oplossing:}
Energiemanagementsystemen (EMS) zouden de eerste stap moeten zijn in optimalisatie, maar op vandaag nog onbekend... \bigbreak
Welke meetsystemen bestaan er allemaal:\begin{itemize}
    \item Stem je verwachtingen af met het te kiezen product, er is veel keuze opde markt maar....\begin{itemize}
        \item Maak een keuze tussen monitoring of sturing 
        \item Alarmering/rapportering 
        \item Sturing eigenverbruik of marktwerking 
        \item \dots
    \end{itemize}
\end{itemize}
\subsubsection{Energiemonitoring \& Visualisatie}
Opvolging/ alarmering (technisch personeel)\begin{itemize}
    \item Lekken 
    \item Stroomonderbreking 
    \item Toestellen in storing (vb. WKK, PV, warmtepomp)
    \item Afnemend rendement zon (vb. door vervuiling)
\end{itemize}
Analyse (energiemanager)\begin{itemize}
    \item Detecteren van onnodige sluimerverbruikers (vb. HVAC die blijft draaien)
    \item Verwerken energiefacturatie/ doorfacturatie 
    \item Advies door besparing/optimalisatie
\end{itemize}
\subsubsection{Energievisualisatie + monitoring:}
Sturing kan op maat a.d.h.v het volgende:\begin{itemize}
    \item Zelfconsumptie/zelfvoorziening 
    \item Dynamisch contract 
    \item Peakshaving 
    \item Aanstuurbare assets (zon, WKK, batterij, laadpaal, verwarming/koeling, HVAC)
\end{itemize}
invloed EMS:\begin{itemize}
    \item 'Domme laadpaal' $\rightarrow$ slim laden  
    \item Hoge kwartierkpiek (7,5kw) $\rightarrow$ lagere kwartierpiek (3.3-4.5 kW)
    \item Veel zonne injectie $\rightarrow$ meer eigenverbruik
    \item geen loadbalancing $\rightarrow$ loadbalancing
\end{itemize}
\subsection{Case studie:}
\subsubsection{Stap 1: Data verzamelen}
Data vereist voor dimensioneringsanalyse\begin{itemize}
    \item Mogelijkheid 1: Kwartierdata opvragen
\end{itemize}
totaalverbruik=productie-injectie+afname (per kwartier).
\paragraph{Stap 1: verbruiksanalyse}
We kunnen aan zijn verbruiksprofiel zien dat een batterij interessant kan zijn voor deze use-case veel zonne productie maar s'nachts hebben we verbruik maar geen zonneproductie.
\paragraph{Stap 2:Dimensionering hernieuwbare of decentrale energieproductie (bv. WKK)}
\begin{itemize}
    \item [] We zien hier dat het PV systeem geoverdimensioneerd is. zit in de rode zone 1.2 op de chart terwijl optimaal rond de 0.6 zou zitten. Ze hebben naar totaal verbruik jaarlijks gekeken maar verbruik is gelijkaardig over dag en nacht waardoor s'nachts er geen opbrengst zon is dus het systeem is dubbel overgedimensioneerd voor wat we eigelijk kunnen gebruiken. 
\end{itemize}
\paragraph{Stap 3: slim sturen}
\begin{itemize}
    \item Stem verbruik af op productie...
    \item Kan ik verbruik in de ochtend/avond shiften zonder comfortverliers?
    \item Wat is het sluimerverbruik?
\end{itemize}
Inzicht in verbruikers is kritiek voor optimalisatie, maar door inzicht kan men grote besparingen realiseren...
\paragraph{Stap 4: Batterijopslag (voor eigenverbruik)}
\begin{center}
    \begin{adjustbox}{width=\textwidth}
    \begin{tabular}{|c|c|c|c|}
    Batterij specificaties & 200 kWh & 500 kWh & 2000 kWh \\ 
    Investeringskost per kWh & 500 €/kWh & 450 €/kWh & 350 €/kWh \\  
    \# Cycli & Minstens 7000 cycli & Minstens 7000 cycli & Minstens 7000 cycli \\
    C-rate & 0,5C & 0,5C& 0,5C \\ 
    Type batterij & LFP & LFP & LFP\\
    Responstijd & ms & ms & ms 
    \end{tabular}
\end{adjustbox}
\end{center}
\begin{itemize}
    \item[] Het is een must dat batterijleveranciers samenwerken met energieleveranciers voor optimale sturing\begin{itemize}
        \item energieleveranciers stuurt 0/1 en de sturing kiest er om er al dan niet op in te gaan ifv controlestrategie.
    \end{itemize}
\end{itemize}
\begin{itemize}
    \item Einde optimale zone 1: 0,85 kWh/MWh 
    \item Einde zone 2: 1,5 kWh
\end{itemize}
Deze batterij houdt geen rekening met de energieprijs op de
markt! Enkel optimalisatie eigenverbruik – zonneproductie!!
\begin{center}
    \begin{tabular}{|c|c|}
        & 2023\\
    verbruik & 202,01 MWh\\
    Productie & 236,67 MWh\\
    Ratio O/V & 1,17\\
    Optimale batterij -zone 1 & 171,71 kWh\\
    Batterij - zone 2 & 303,02 kWh
    \end{tabular}
\end{center}
Opgelet, dit is een optimum voor dit specifiek verbruiks-
/opbrengstprofiel!!! Dit is niet voor iedereen hetzelfde!
In dit geval kijken we ook al of we niet in de gele zone de batterij zouden plaatsen. hangt beetje van case tot case af. je moet niet bang zijn van soms te kijken naar de gele zone.
\paragraph{Stap 4: Batterij – Verdienmodel verhogen}
klant is aangesloten op een day ahead contract
\begin{itemize}
    \item Strategie 1: Optimalisatie eigenverbruik 
    \item Strategie 2: Strategie 1 + Sturing day-ahead
    \item Strategie 3: Strategie 2 + Onbalansmarkt
\end{itemize}
\begin{center}
    \begin{tabular}{|c|c|c|}
    200 kWh batterij  & Totale elektriciteitsfactuur& Besparing / Winst\\
    initiele kost zonder batterij & 20.877 € & -\\
    Totale kost met strategie 1 & 10.869 € & 10.008 €\\
    Totale kost met strategie 2 & 7.530 € & 13.347 €\\
    Totale kost met strategie 3 & -1.248,8 € & 22.125 €
    \end{tabular}
\end{center}
\begin{itemize}
    \item ! Opgelet, nu gaan we uit van een perfecte voorspelling
    van de energieleverancier op de energiemarkt !
    \item ! Opgelet, geen marge voor energieleverancier
    ingerekend !
\end{itemize}
All-in kost voor batterijopslag:\begin{itemize}
    \item 500 Euro/kWh totaalprijs
\end{itemize}
\begin{center}
    \begin{tabular}{|c|c|c|c|}
    Andere resultaten  & Strategie 1 & Strategie 2 & Strategie 3\\
    Terugverdientijd & 10 jaar & 7,5 jaar & 4,5 jaar\\
    \# Cycli & 159 & 298 & 730\\
    Max. piek & 124 kW & 160 kW & 397 kW\\
    Zelfconsumptie & 59,38\% & 55,23\% & 53,33\%\\
    Zelfvoorziening & 69,45\% & 64,67\% & 61,79\%\\
    Levensduur bat. & 44 jaar & 23 jaar & 9,6 jaar
    \end{tabular}
\end{center}
! Eigenverbruik daalt, risico op investering stijgt...
\paragraph{Stap 4: Batterij – Verdienmodel verhogen}
Leverancier is niet in staat om een perfecte voorspelling te realiseren op de onbalansmarkt!\begin{itemize}
    \item Winstmarge moet met minstens 30 à \% gereduceerd worden:\begin{itemize}
        \item ifv hoe goed men de prijzen kan voorspellen (afwijking van minstens 20 - ...\%)
        \item Alsook een marge voor de energieleverancier (10 - ... \%)
    \end{itemize}
\end{itemize}
$\rightarrow$ Winst van 22.125€ daalt naar 19.052,7€ (35\% gap gerekend op onbalans
+ marge leverancier). Terugverdientijd stijgt van 4,5 naar 5.25 jaar. 500Kwh en 2 MWH batterij zijn te groot voor onze case en zijn niet interessant voor optimalisatie eigenverbruik. in deze case is een 200 kWh batterij optimaal.
\subsubsection{Voorbeeld steun}
Bekijk zeker de VLIF steun\begin{itemize}
    \item 40\% tot zelf 50\% voor jonge boeren
    \item Minimum capaciteit van 20 kWh 
    \item Wetgeving recent gewijzigd! Steun afgetopt tot 100 kWh (= 45.215 €) 
    \item Batterij moet optimalisatie aanbieden van eigenverbruik (overproductie zonnepanelen) 
    \item Subsidiabel zijn batterij en omvormers 
    \item 1 batterij kan subsidie krijgen in periode 2023 - 2027
    \item Vb. Max grootte batterij:\begin{itemize}
        \item Batterij van 2 MWh voldoet niet aan voorwaarde voor VLIF steun voor
        deze klant( Productie/Totaalverbruik) < 121 = max. 1,67 MWh
    \end{itemize}
\end{itemize}
Vroeger was het degressief zonder beperking... Dus grote wijziging!
\subsubsection{Conclusie:}
evaluatie batterijopslag:\begin{itemize}
    \item Eigenverbruik stijgt amper, risico wordt groter om verdienmodel te laten afhangen van onbalans
\end{itemize}
\paragraph{Batterij: verdienmodel verhogen:}
\begin{itemize}
    \item Meerwaarde van onbalansmarkt bij kleine
    batterij is laag
    \item Bij grote batterij is dit de enige parameter om
    batterij terugverdienbaar te houden, maar
    enorm risico...
    \item Marge voor day-ahead en eigenverbruik
    satureert bij grotere batterijen, dus geen
    noemenswaardige meerwaarde! 
    \item Capaciteitstarief speelt grote rol!
\end{itemize}
$\rightarrow$ Blijf realistisch, de markt is onvoorspelbaar...\bigbreak
Investeren of niet?\begin{itemize}
    \item Bepaal en spreid het risico (cfr. Belegging bij bank)\begin{itemize}
        \item Dynamisch versus defensief?
    \end{itemize}
\end{itemize}
Is een investering in energieopslag slecht?\begin{itemize}
    \item Zeker niet, maar denk aan risicospreiding door \# assets
    te implementeren\begin{itemize}
        \item E-boiler, elektrische mobiliteit, batterijopslag,... 
        \item Denk na over het slim sturen van eigen proces!
    \end{itemize}
\end{itemize}
Vergelijking met zonnepanelen.\begin{itemize}
    \item 2022, een financiële opsteker voor een investering in
    grote zonne-installaties!\begin{itemize}
        \item Hoe groter de installatie, hoe sneller
        terugverdiend door hoge injectieprijzen
    \end{itemize}
    \item 2024, de injectievergoeding is zeer laag geworden,
    steeds meer negatieve prijzen\begin{itemize}
        \item Grote zonne-installaties worden afgestraft!
    \end{itemize}
\end{itemize}
\section{Smart mobility:}

\end{document}
