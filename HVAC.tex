
\documentclass[12pt]{article}

\usepackage{amsmath}

\usepackage{microtype}

\usepackage{graphicx}

\usepackage{hyperref}

\usepackage[utf8]{inputenc}

\usepackage{listings}

\usepackage{matlab-prettifier}
% % voor code syntax highlighting
% \usepackage{minted}

% beter font
\usepackage[T1]{fontenc}
\usepackage{helvet}
\newcommand{\myparagraph}[1]{\paragraph{#1}\mbox{}\\}
\setcounter{secnumdepth}{5}
\setcounter{tocdepth}{5}
\renewcommand{\familydefault}{\sfdefault}
\graphicspath{{imagessoft/}}
\begin{document}
\begin{titlepage}
    \author{Xander Vandooren}
    \title{HVAC}
\end{titlepage}
\pagenumbering{gobble}
\maketitle
\newpage
\tableofcontents
\newpage
\pagenumbering{arabic}
Voor vragen mailen naar: pieter.salomez@jci.com
school email in cc.
\section{Week 1:}
Belangrijke slides: 6,7,8,9,10,11,12,13,14,15,17,18,19,20
\textbf{HVAC staat voor heating ventilation en airconditioning}
\subsection{Basisbegrippen Thermodynamica:}
\subsubsection{\textbf{Energiebronnen $\rightarrow$ energie omzetters}}
Belangrijkste bron van energie is de zon.
\begin{itemize}
    \item V\textbf{\underline{erdamping \& neerslag}}\begin{itemize}
        \item waterkracht $\rightarrow$ elektriciteit
    \end{itemize}
    \item \textbf{\underline{Biomassa}}\begin{itemize}
        \item Elektriciteit-centrale $\rightarrow$ elektriciteit
        \item Brandsotf conversie $\rightarrow$ warmte
    \end{itemize}
    \item \textbf{\underline{Fossiele brandstof}}\begin{itemize}
        \item Elektriciteit-centrale $\rightarrow$ elektriciteit
        \item Warmte-centrale $\rightarrow$ warmte
        \item verbrandings-motoren $\rightarrow$ arbeid
    \end{itemize}
    \item \textbf{\underline{Wind}}\begin{itemize}
        \item Windturbine $\rightarrow$ elektriciteit
    \end{itemize}
    \item \textbf{\underline{Zonlicht}}\begin{itemize}
        \item Zonnepanelen $\rightarrow$ elektriciteit
        \item zonnecollector $\rightarrow$ warmte 
    \end{itemize}
    \item \textbf{\underline{van oppervlak en atmosfeer}}\begin{itemize}
        \item warmtepomp $\rightarrow$ warmte
    \end{itemize}
\end{itemize}
\subsubsection{Kinetische energie:}
Is een vorm van energie, eigen aan een bewegend lichaam, vanwege de traagheid van massa. De kinetische energie van een bewegend lichaam is.
$E_{kin} = 1/2 * m * v^2$ hierbij is $m$ de massa en v de snelheid in $km/h$.\newline
De eenheid voor kinetische energie is  in Joule ($J$)
opdracht:
\subsubsection{Potentiele energie:}
In de natuurkunde is de potentiele energie de arbeid die een voorwerp in staat is (= de potentie heeft) te verrichten als gevolg van de toestand waarin het voorwerp zich bevindt. Deze toestand kan de situering in een krachtenveld zijn of de interne configuratie van het voorwerp. $E_{Pot}= m.g.h$  hierbij is $m$ de massa (in kg) $g$ de zwaartekracht ($9.81^{m/s^2}$) en $h$ de hoogte.
\subsubsection{Inwendige energie}
\subsubsection{Chemische energie}
$H_{ch}=m_{br}*h_{0}$
\subsubsection{Warmte-Nulde wet va de Thermodynamica}
De 'nulde wet' stelt voor dat als A dezelfde temperatuur heeft als B, en B heeft dezelfde temperatuur als C, A ook dezelfde termperatuur heeft als C.

Een aandere formulering is dat \textbf{warmte alleen van een hogere naar een lagere temperatuur stroomt.}

\myparagraph{Warmtestromen}
\begin{itemize}
    \item Convectie\begin{itemize}
        \item Warmte overdracht door verplaatsing van een warme vloeistof of een warm gas, of van een koude vloeistof of een koud gas. Wanneer warmte  door stroming wordt meegevoed,kan de mate daarvan worden uitgedrukt met de \textbf{warmteoverdrachtscoefficient h [W/$m^2$K]}
    \end{itemize}
    \item conductie\begin{itemize}
        \item Dit is de warmteoverdracht binnen de desbetreffende stof, waarbij warmte stroomt van deeltjes 
    \end{itemize}
    \item radiatie
\end{itemize}
\subparagraph{Warmtestroomdichtheid}
%formule toevoegen en wat meer uitleg.
\textbf{Warmtestroomdichtheid}
\subparagraph{conductie}
Dit is de warmteoverdrach binnen de desbetreffende stof , waarbij warmte stroomt van deeltjes met de hogere kinetische energie (temperatuur) naar minder energierijke (koude) deeltjes. De warmtestroom is afhankelijk van de het temperatuursverschil over de afstand (temperatuurgradient) en de interne weerstand tegen de warmtestroom van het betreffende materiaal, die de thermische geleidbaarheid of warmtegeleidingscoefficient l [W/mK]genoemd wordt.
%formule toevoegen
\subparagraph{convectie}
Warmtetransport door conectie is een proces waarbij warmte wordt overgedragen door de beweging van vloeistoffen of gassen. DOt gebeurdt wanneer een warmer deel van de vloeistof het gas zich verplaatst naar een kouder deel, waardoor warmte wordt getransporteerd. Convectie kan op twee manieren plaatsvinden.

\begin{itemize}
    \item \textbf{Natuurlijke convectie:}\begin{itemize}
        \item Dit gebeurtdt door temeperatuurverschillen die leiden tot dichtheisverschillen in de vloeistof of het gas. Bijvoorbeeld warme lucht stijgt op omdat het minder discht is dan de houde lucht.
    \end{itemize}
    \item \textbf{Geforceerde convectie:}\begin{itemize}
        \item Hierbij wordt de beweging van de vloiestof of het gas veroorzaakt door externe krachten, zoals een ventilator of pomp.
    \end{itemize}
%formule toevoegen
\myparagraph{Stofeigenschappen}
\subparagraph{Dichtheid}
Dichtheid, ook wel massadichtheid genoemd, is een maat voor de massa per volume-eenheid van een materiaal. Het wordt vaak aangeduid met de Griekse letter $\rho$ (rho).
$\rho=m/v$
\subparagraph{Specifieke warmte}
\myparagraph{Toestandsgrootheden}
\subparagraph{Entropie}
\end{itemize}
\section{week 2:}
belangrijke slides:5-21,23-36
\subsection{Ideale gaswet-oefening 2:}
In een zuurstoffles bevindt zich 50 $dm^3$ zuurstof van 17C bij een drik van 150 bar.
\subsection{Wet van dalton $\rightarrow$ gasmengsels:}
De totale druk van een gasmengsel is gelijk aan de som van de rukken welk elk gas zou uitoefenen \textbf{partieel-druk}
\myparagraph{voorbeeld:}
we hebben een waterboiler met een inhoud van 120L bij een temp van 60 graden celsius.
er wordt water afgetapt met een volumedebiet van 5L/min. er wordt water aangevuld met een temperatuur van 10 graden celsius. Het water is homogeen verdeelt.
Wat is de watertemperatuur na 10 minuten?
\subsection{Eerste wet van de thermodynamica:}
Veert men aan een lichaam (kan bvb een gasmassa zijn) een hoeveelheid warmte Q toe, dan zal de toestand waarin dit lichaam verkeert veranderen. Dit \textbf{verschil} is de energie die met name de snelheid van de moleculen m.a.w de \textbf{kinetische energie van de moleculen vergroot}. Deze \textbf{Kinetische energie} is nu juist ed \textbf{inwendige energie U}.
\paragraph{Eerste hoofdwet: "Wet van behoud van energie":}
Heeft een massa van 1kg een inwendige energie van u1 en voert men Q joule aan de warmte toe, dan wordt de inwendige energie u2 en wordt er arbeid naar buiten toe geleverd (de volumevergroting moet duwen tegen de omgeving).
$\Delta U=Q+W$
\begin{itemize}
    \item \textbf{Exotherm proces:} van binnen naar buiten $\Delta U < 0$
    \item \textbf{Endotherm proces:} van buiten naar binnen $\Delta U > 0$
\end{itemize}
\myparagraph{Behoudswet- Massabalans}
$(Toegevoerde-Afgevoerde)_{SG} + Productie_{SYST}=Teename_{SYST}$
\myparagraph{Efficiency van een systeem:}
\begin{itemize}
    \item Mechanische rendement: Wordt gebruikt bij processen waarbij mechanische energie van een energiedrager wordt overgedragen naar een andere energiedrager.
    \item Thermisch rendement: Bij omzetting van warmte in arbeid wordt gesproken over thermisch rendement.
    \item Coefficient of performance (COP) : is een verhouding die weergeeft in welke mate de primaire energie wordt omgezet in een daardwerkelijke eindresultaat. (hoe hoger uw COP hoe beter)
    \item Primaire energie ratio: Primaire energie ratio (PER) is een maatstaf voor de hoeveelheid primaire energie die nodig 
\end{itemize}
\myparagraph{Arbeid geleverd door een gas:}
Als een gas uitzet of krimpt, duwt het een of meer van zijn begrenzde wanden voor zich uit waardoor het volume toe- of afneemt. Het gas levert dan arbeid op zijn omgeving (+) of neemt arbeid op (-).\newline
Als de druk van het gas p is, is de grootte van de kracht, door het gas op de binnenkant van de zuiger uitgeoefend. $F =p.A$
wanneer de zuiger over een zeer kleie of elementaire afstand dx verplaatst dan is: $W=F.\Delta x=p.A.\Delta x$
$\Delta U= Q =p.dV$
\myparagraph{Druk verandering gedurende proces:}
W= \(\int_{V_{B}}^{V_{A}} p.dV\)
\begin{itemize}
    \item \textbf{Isobaar:} n=0
    \item \textbf{Isotherm:} n=1
    \item \textbf{Isentropisch of adiabatisch:} n=
    \item \textbf{Isochoor:} n=+$\infty$
    \item praktisch $\rightarrow$ polytroop\begin{itemize}
        \item Ligt tussen isotherm en isentroop
        \item Afhnakelijk van polytropische index (n)
    \end{itemize}
\end{itemize}
\myparagraph{Isotherm proces ($\Delta T=0$)}
Hebben we een proces waarbij \textbf{de temperatuur constant} is, dan spreken we van een isotherm proces. Isotherm is de griekgse vertaling voor 'Gelijke temperatuur'. Spreken we van een ideaal gas dan weten we door de wet van Boyle dat druk maal volume constant is bij een gegeven temperatuur T.
\myparagraph{Adiabatisch proces (Q=0)-isentroop}
Dit is een proces waarbij er geen warmte in of uit het proces stroomt. We kunnen \underline{2 situaties} hebben waarbij dit voorkomt:\begin{enumerate}
    \item Ten eerste kan er sprake zijn van een systeem die in die mate geisoleerd is dat er geen warmte kan uitstromen.
    \item Ten tweede kan het verloop fan het proces zo snel gebeuren dat de warmte de tijd niet heeft om weg te stromen.
\end{enumerate}
\myparagraph{Isobaar proces (p=constant)}
Dit is een proces waarbij \textbf{de druk constant} wordt gehouden.
Bij het opwarmen van een gas kan het volumen toenmen,zonder meer.
\myparagraph{Isochoor proces (V=constant)}
Dit is een proces waarbij \textbf{het volume constant} wordt gehouden.
Een voorbeeld hiervan is het opwarmen van een gasfles. De temperatuur zal toenemen maar de systeem grenzen kunnen niet aangepast worden.
%voeg de slide toe.
\subsection{Tweede hoofdwet:}
Vaststelling:\begin{itemize}
    \item \underline{Warmte} stroomt van nature van warme naar koude gebieden en nooit spontaan van koud naar warm.
    \item \underline{Arbeid} Kan wel volledig in warmte, 
\end{itemize}
\myparagraph{Kringprocessen:}
\begin{itemize}
    \item Carnot-proces\begin{itemize}
        \item \textbf{Beschrijving:} Het ideale kringproces methet hoogste theoretische rendement. Het bestaat uit twee isothemre en twee diabetische processen.
        \item \textbf{voorbeeld:} Een Carnot-warmtepomp die warmte van een koude naar een warme omgeving verplaatst met maximale efficientie.
    \end{itemize}
    \item Rankine-proces
    \item Brayton-proces
    \item Otto-proces
    \item Stirling-proces
\end{itemize}
\myparagraph{Carnot}
De Carnot-cylcus is een theoretisch ideaal thermodynamisch kringproces dat bestaat uit vier omkeerbare stappen. Hier is een overzicht van de stappen in de Carnot-cyclus:\begin{enumerate}
    \item %vul verder aan
\end{enumerate}
\myparagraph{Rankine}
Een typisch voorbeeld van de Rankine-cyclus is een kolengestookte elektriciteitscentrale. Hier is een stap-voor-stap beschrijving van hoe de Rankine-cylcus in zo'n centrale werkt:\begin{enumerate}
    \item Isentropische compressie:
    \item Isobare warmte toevoer
\end{enumerate}
\myparagraph{Brayton}
De Brayton-cyclus, ook welbekend als de Joule-cyclus, is een thermodynamisch kringproces dat vaak wordt gebruikt in gasturbines en straalmotoren. Het proces bestaat uit vier hoofdfasen.\begin{enumerate}
    \item compressie
    \item verbranding
    \item Expansie
    \item Uitlaat
\end{enumerate}
\myparagraph{Otto}
De Otto-cylcus bestaat uit vier hoofdprocessen:
\begin{enumerate}
    \item Isentropische compressie:Het lucht-brandstofmengsel wordt adiabetisch en reversibel gecomprimeerd in de cilinder, waardoor de druk en temperatuur
    \item Isochore warmte toevoer:
    \item Isentropische expansie:
    \item Isochore warmte afvoer:
\end{enumerate}
Praktisch:\begin{itemize}
    \item 
\end{itemize}
\myparagraph{Seiliger of Diesel}
De Diesel-cyclus bestaat uit vier hoofdprocessen:
\begin{enumerate}
    \item Isentopische compressie:
    \item Isobare warmte toevoer:
    \item Isentropische expansie:
    \item Isobare Warmte afvoer:
\end{enumerate}
\myparagraph{Stirling}
De Stirling-cyclus is een thermodynamisch proces dat wordt gebruikt in de Stirlingmotoren, die bekend staan om hun efficientie en stille werking. Hier is een overzicht van de Stirling-cyclus:
\begin{enumerate}
    \item Isotherme compressie:
    \item Isovolumetrische (isochorische) verwarming:
\end{enumerate}
\subsection{Open kringprocessen}
\begin{itemize}
    \item Pomp:\begin{itemize}
        \item %vul verder aan
    \end{itemize}\begin{itemize}
        \item %vul verder aan
    \end{itemize}
    \item Smoorvoorziening:
    \item Warmtewisselaar:
    \item ketel:
    \item Compressor:
\end{itemize}
\myparagraph{Open-system Pomp:}
Een pomp is een werktuig dat wordt gebruikt om vloeistoffen of gassen te verplaatsen door er energie aan toe te voegen in de vorm van een drukverhoging of snelhiedsverhoging.\textbf{Dit betekent dat een pomp vloeistoffen 
of gassen van een lager niveau naar een hoger niveau kan verplaatsen}.
\begin{itemize}
    \item $P_{pomp}=$
\end{itemize}
\myparagraph{Smoorvoorziening:}
Een smoorvoorziening wordt gebruikt om de druk te verlagen in een systeem.
\begin{itemize}
    \item[] $h_{uit}=h_{in}$
\end{itemize}
Een expansieventiel is een essentieel onderdeel van koel- en airconditioningsystemen. Het heeft als doel het koudemiddel van een hoge druk (en hoge temperatuur) naar een lage druk (en lage temperatuur) te brengen. Dit proces zorgt ervoor dat het koudemiddel verder kan verdampen en warmte kan opnemen uit de omgeving.
\myparagraph{Warmtewisselaar:}
Een warmtewisselaar is een apparaat dat warmte overdraagt van een medium (zoals een vloeistof of gas) naar een ander medium, zonzder dat de tweede media direct met elkaar in contact komen. Dit proces wordt vaak gebruikt in verwarmings-, ventilatie- en airconditioningsystemen (HVAC), industriele processen, en energieopwekkinssystemen om efficientie te verhogen en energie te besparen.
\myparagraph{Ketel:}
Een verbrandingsketel, ook wel bekend als een cv-ketel of centrale verwarmingsketel, is een apparaat dan wordt gebruikt om warmte te gegenereren door het verbranden van brandstoffen zoals gas,olie,hout of biomassa.
$\phi$
\myparagraph{Compressor:}
Een compressor is een apparaat dat lucht of een ander gas samenperst tot een hogere druk.
Dit samengeperst gas kan vervolgens worden gebruikt voor verschillende toepassingen, zoals het aandrijven  van gereedschappen en machines,spuitwerk van verf, of het oppompen van banden.
\begin{itemize}
    \item[] $P_{Compressor}=\phi_{m}*(h_{uit}-h_{in})$
    \item[] $W_{Compressor}=\phi_{m}*(h_{uit}-h_{in})$
\end{itemize}
$P_{Compressor}$= Compressor vermogen [W]
\section{week 3 (4):}
belangrijke slides:4,5,6,9-31
\subsection{warmteproductie via verbranding:}
Verbranding is een \textbf{chemische reactie} waarbij een oxidatiemiddel snel creeert met een brandstof om opgeslagen energie vrij te maken. Het \textbf{oxidatiemiddel} voor verbranding is \textbf{zuurstof in de lucht}. De oxidatie vindt normaal gesproken plaats met de \textbf{brandstof in dampvorm (gasvorm)}. Een uitzondering is oxidatie van vaste stof, koolstof, die direct met de vaste fase plaatsvindt. 
\subsubsection{Verbrandingschemie:}
\begin{itemize}
    \item \underline{verbrandingswarmte Vb:}\begin{itemize}
        \item Is de \textbf{warmte} die vrijkomt bij het verdampen van vast of vloeibare stoffen:\begin{itemize}
            \item Bij verbranding van 1kg C tot CO2 onststaan 33,9 MJ voor C
            \item Bij verbranding van 1kg tot S tot SO2 ontstaan 10,4 MJ voor S 
            \item Bij verbranding van 1kg tot H2O (vloeibaar) ontstaan 144,@ MJ
        \end{itemize}
    \end{itemize}
    \item \underline{Stookwaarde Vo}:\begin{itemize}
        \item De \textbf{energie} die vrijkomt bij het verbranden van een brandstof.
    \end{itemize}
\end{itemize}
\subsubsection{Overzicht van brandstoffen:}
\begin{itemize}
    \item Calorische bovenwaarde: DIt is de verbrandingswaarde van een brandstof.
    \item Calorische onderwaarde: Dit is de stookwaarde van een brandstof.
\end{itemize}
\subsubsection{Theoretische luchthoeveelheid:}
\underline{Theoretische luchhoeveelheid L0:}
1m$^3$ lucht bevat met andere woorden 0,207m$^3$\begin{itemize}
    \item Zuurstof die we nodig hebben voor een \textbf{volledige verbranding}. Of met andere woorden een volledige verbranding krijgen we pas bij 4,83$m_{n}^3$ $\rightarrow$ 1/0.207 lucht.
\end{itemize}
\subsubsection{Dauwpunts temperatuur van rookgassen:}
\begin{itemize}
    \item dauwpunttemperatuur van rookgassen bij de verbranding van \textbf{stookolie} ligt doorgaans tussen de \textbf{45 en 55 graden celsius}.
    \item dauwpunttemperatuur van rookgassen bij de verbranding van \textbf{aardgas} ligt meestal ronde de \textbf{55 ot 60 graden celsius}.
\end{itemize}
\subsubsection{Verbrandingsproces:}
\begin{itemize}
    \item \underline{Vlamptemperatuur}:\begin{itemize}
        \item is afhankelijk van de soort brandstof die wordt verbrand. Ideaal gebeurt de verbranding volledig, zonder afgifte van warmte aan omliggende wanden. Dit zou dan de adiabatische Vlamptemperatuur heten. Praktische ligt deze waarde enkele honderden graden lager als gevolg van straling naar omgeving.
    \end{itemize}
    \item \underline {Ontstekingstemperatuur}:\begin{itemize}
        \item De onstekingstemperatuur van een brandstof is het laagste punt waarop een brandstof vanzelf zal ontbranden zonder dat er een externe ontstekingsbron nodig is.
        \item Meestal is dit aardagas en diesel. aardgas is tussen de 400-950 graden en diesel rond 450 graden celsius
    \end{itemize}
\end{itemize}
\paragraph{Mazoutbranders:}
\begin{itemize}
    \item Gele vlambrander:\begin{itemize}
        \item klassieke uitvoering van de mazoutbrander, maar onderging een hele evolutie voor een betere werking.
    \end{itemize}
    \item Blauwe vlambrander:\begin{itemize}
        \item Blauwe vlambrander is een populaire mazoutbrander: stoot amper roet of CO uit, de brandstof zo goed als volledig verbruikt wordt. Je herkent dit type brander aan de blauwe vlam. lage vlamtemperatuur wat de vorming van stikstofoxide of Nox tegengaat.
    \end{itemize}
    \item Low-NOx-brander of grijze vlamblander:\begin{itemize}
        \item %vul verder aan
    \end{itemize}
    \item Modulerende olieverdampingsbrander:\begin{itemize}
        \item %vul verder aan
    \end{itemize}
\end{itemize}
\paragraph{Aardgasbranders:}
\begin{itemize}
    \item \underline{Atmosferische branders:}\begin{itemize}
        \item Deze branders mengen gas met lucht voordat het mengsel wordt ontstoken. Ze zijn eenvoudig en betrouwbaar, maar minder efficitent dan andere types.
    \end{itemize}
    \item \underline{Premix branders:}\begin{itemize}
        \item Deze branders mengen gas en lucht in een vooraf bepaalde verhouding voor een efficientere verbranding en lagere uitstoot.
    \item \underline{Condensatiebranders:}\begin{itemize}
        \item Deze branders gebruiken de warmte van de rookgassen om water te verwarmen, wat resulteert in een zeer hoog rendement.
    \end{itemize}
    \end{itemize}
\end{itemize}
\subsubsection{Warmwaterketels-overzicht:}
\begin{itemize}
    \item \underline{Conventionele ketels:}\begin{itemize}
        \item Werking: Verbranden brandstof (zoals aardgas, stookolie) om water te verwarmen.
        \item Voorbeeld: Novum warmwaterketels zijn ontworpen voor commerciële en industriële toepassingen met watertemperaturen tot 100°C
    \end{itemize}
    \item \underline{Hogedruk: Warmwaterketels:}\begin{itemize}
        \item Werking: Werken bij hogere druk en temperatuur, vaak boven 110°C, voor
        industriële toepassingen.
        \item Voorbeeld: Vitomax HW van Viessmann, met een vermogen van 0,35 tot 20,0 MW
        en een efficiëntie van meer dan 95,5\%.
        \item \underline{Biomassaketels:}
        \item \underline{Elektrische ketels:}
        \item \underline{Hybride ketels:}
    \end{itemize}
    \item \underline{Lagedruk-Warmwaterketels:}\begin{itemize}
        \item Werking: Werken bij lagere druk en temperatuur, geschikt voor minder intensieve
        toepassingen.
        \item Voorbeeld: Vitomax LW van Viessmann, met een vermogen van 0,65 tot 22,0 MW
        en geschikt voor diverse brandstoffen.
    \end{itemize}
    \item \underline{Condensatieketels:}\begin{itemize}
        \item Werking: Maken gebruik van de warmte uit de rookgassen door condensatie, wat
        resulteert in een hoger rendement.
        \item Voorbeeld: HR-ketels (Hoog Rendement ketels) die vaak in huishoudens worden
        gebruikt voor hun energie-efficiëntie.
    \end{itemize}
\end{itemize}
\subsubsection{Warmtebalans van een ketel:}
%slides hier toevoegen.

Water moet zeker 1 keer per week +- tot 60 graden opgewarmd worden in uw boiler om legionella tegen te gaan (kan ontstaan 20-30 graden bij stilstaand water).
\subsubsection{Ketelrendement en dauwpunt:}
\begin{itemize}
    \item \underline{HR-ketels:}\begin{itemize}
        \item Maken gebruik van het feit dat rookgassen waterdamp bevatten.
        \item Hoe lager uw Cv-retourtemperatuur hoe hoger uw ketelrendement. Het condensatiepunt is 58.6 graden celsius.
    \end{itemize}
\end{itemize}
\subsubsection{Na warm water komt stoom:}
Stoom wordt gebruikt voor arbeid
\begin{itemize}
    \item Natte stoom:\begin{itemize}
        \item Stoom waarin kleine waterdeeltjes zweven. Door deze waterdeeltjes is het soortelijk gewicht gestegen. Indien natte stoom verhit wordt, dan zullen allereerst deze waterdeeltjes verdampen. Bij voldoende verwarming wordt de stoom dan droog verzadigd.
    \end{itemize}
    \item Verzadigde stoom:\begin{itemize}
        \item Is stoom die condenseert bij verlaging van temperatuur. Deze stoom is ongeschikt voor het aandrijven van een stoomturbine, maar voldoet prima voor het aandrijven van een stoommachine of voor de overdracht van warmte. Er wordt dan wel gebruikgemaakt van verzadigde stoom die bij hogere druk is gegenereerd.
    \end{itemize}
    \item Overhitte stoom:\begin{itemize}
        \item Om overhitte stoom te maken moet de stoom extra worden opgewarmd naar een hogere temperatuur. In een fluitketel kan dat niet omdat daar alleen het water wordt verwarmd. In een stoomketel gebeurt hetzelfde. maar de geproduceerde stoom kan vervolgens door een overhitten worden geleid. De verzadigde stoom stijgt in temperatuur en wordt dan overhitte stoom genoemd.
    \end{itemize}
\end{itemize}
\subsubsection{Stoom in de procestechniek:}
%afbeelding toevoegen.
\paragraph{Water en zijn verschillende vormen:}
Bij 4 graden celsius is water het zwaarste.
\begin{itemize}
    \item \underline{Overgangsfase(s):}\begin{itemize}
        \item Bijgevoegde grafiek is een pT-diagram, die de verschillende fasetoestanden weergeeft van water. Naast vaste stof (ijs), vloeibaar (water), gas (waterdamp) zien we ook de relatie tussen alle drie. Bij 1 wel gemikt, specifiek, punt bestaan ze alle drie in elkaar. Dit in het zogenaamde triple punt. De andere speciale toestand kennen we als "Critical point" of Kritisch punt. Hier bestaan gas en vloeistof in elkaar.
    \end{itemize}
\end{itemize}
Sublimatie is van vast direct naar gas zonder het in vloeibare vorm te moeten omzetten.
\paragraph{Tv diagramma voor water:}
Water onder constante druk en het bijhorende soortelijk volume.
\paragraph{ph-diagram van water:}
Water onder constante druk en het bijhorende soortelijk volume.
\begin{itemize}
    \item X-as: Temperatuur [in graden celsius]
    \item Y-as: Saturatiedruk [bar]
\end{itemize}
\paragraph{Waterpijpketel:}
Oude verwarmingen werkten ook op dit principe.
\begin{itemize}
    \item Constructie: \begin{itemize}
        \item Bestaat uit een voedingswatervat aan de onderzijde, een stoomvat aan de bovenzijde, en waterpijpen die de twee vaten verbinden.
    \end{itemize}
    \item Werkingsprincipe:\begin{itemize}
        \item 
    \end{itemize}
    \item Voordelen
    \item Toepassingen
\end{itemize}
\paragraph{Vlampijpketel:}
Dit wordt voornamelijk in de industrie gebruikt.
\begin{itemize}
    \item Constructie: \begin{itemize}
        \item Bestaat uit een cilindrische ketelwand met een of meer vuurgangen, vlampijpen, een vlamkas, water en stoom.
    \end{itemize}
    \item Werkingsprincipe:\begin{itemize}
        \item 
    \end{itemize}
    \item Voordelen:
    \item Toepassingen:
\end{itemize}
\paragraph{Spiraalketel:}
\begin{itemize}
    \item Constructie: \begin{itemize}
        \item Bestaat uit een spiraalvormige buis waarin water wordt gepompt. De buis wordt
    \end{itemize}
    \item Werkingsprincipe:\begin{itemize}
        \item 
    \end{itemize}
    \item Voordelen
\end{itemize}
\paragraph{De stoomgenerator:}
\begin{itemize}
    \item Compact ontwerp:
    \item Efficientie:
    \item Veiligheid:
    \item Toepassingen:
\end{itemize}
gasbuizen worden meestal in het geel gekleurd (dit is de norm).
\paragraph{Overzicht stoominstallatie - toebehoren:}
\begin{itemize}
    \item watervoorbereiding-hardheindscontrole-doseerinstallatie
    \item Voedingstank-Ontgasser
    \item Voedingspompen
    \item Brander 
    \item LUVO-Luchttoevoer naar brander
    \item Economizer
    \item Rookgaskoeler
    \item Elektrische kast 
    \item Veiligheden 
    \item Collector 
    \item Spuitank - spui op geleidbaarheid - bodemspui
\end{itemize}
\section{Week 5:}
belangrijke slides:4-11,13-17,19,20,21,23-28
\subsection{Pompen en compressoren:}
\subsubsection{Pomp:}
Pomp heeft als hoodfunctie het verplaatsen van vloeistoffen van de ene plaats naar de andere.
\begin{itemize}
    \item Verhogen van druk:\begin{itemize}
        \item Pompen worden vaak gebruikt om de druk van een vloeistof te verhogen, zodat deze door leidingen kan stromen of naar hogere niveaus kan worden getransporteerd.
    \end{itemize}
    \item Verplaatsen van vloeistoffen:\begin{itemize}
        \item Pompen kunnen vloeistoffen verplaatsen van een lager naar een hoger niveau, zoals bij het oppompen van water uit een put.
    \end{itemize}
\end{itemize}
\paragraph{Wet van Bernoulli:}
De bernoullivergelijking is een fundamentele vergelijking in de vloeistofmechanica die het gedrag van een vloeistof beschrijft langs een stroomlijn.
\textbf{formule:}
\paragraph{Manometrische opvoerdruk:}
Manometrische opvoerdruk is de totale druk die een pom moet leveren om een vloeistof te verplaatsen. \begin{itemize}
    \item Statische opvoerdruk (${P_{stat}}$): Dit is de druk die nodig is om het hoogteverschil tussen de zuig- en perszijde van de pomp te overwinnen.
    \item Dynamische druk ({$P_{dyn}$}): Dit is de druk die nodig is om de vloeistof op de gewenste snelheid te brengen.
    \item Leidingweerstand ({$P_w$}): Dit omvat de drukverliezen door wrijving in de leidingen,bochten, ventielen, \dots
\end{itemize}
Dit komt overeen met de wet van Bernoulli:\begin{itemize}
    \item Kinetische energie $\rightarrow$ Dynamische druk
    \item Potentiele energie $\rightarrow$ Statische opvoerdruk
\end{itemize}
\paragraph{Asvermogen:}
Uw asvermogen is altijd kleiner dan uw elektrisch vermogen.
Als we het debiet laten toenemen zal ons vermogen laten stijgen.
Ons debiet is evenredig met de snelheid van onze pomp.
\paragraph{Pomprendement:}
Het rendement van elke machine is de verhouding van het afgegeven vermogen ten opzichte van het opgenomen vermogen.
\begin{itemize}
    \item Motorrendement  (elektrisch en mechanisch)
    \item volumetrisch rendement
    \item Hydraulische rendement
\end{itemize}
\textbf{formule:} $rendement_{tot}$=$rendement_M$.$rendement_{vol}$.$rendement_{hydro}$
\paragraph{Cavitatie en NPSH:}
Als lokaal in de pomp een gebied bestaat waar de statische druk lager is dan de dampdruk van de vloeistof, ontstaan dampbellen. Dit wordt cavitatie genoemd. Het onstaat als gevolg van lokale versnelling van de vloeistof, waarbij statische druk wordt omgezet in kinetische energie. Dit kan bijvoorbeeld gebeuren bij de inlaat van de pomp, waar de vloeistof enigszins versneld wordt alvorens het de waaier binnentreedt. Een andere plaats is dicht in de buurt van het neusprofiel van de waaierschoepen of aan de tip van propellers.
\begin{itemize}
    \item De anti-cavitatie druk-marge of Net Positive Suction Head (NPSH) is gedefineer (in meters):\begin{itemize}
        \item NPSH=$\frac{p_s}{\rho*g}$+$\frac{v_s^2}{2*g}$+$\frac{p_v}{\rho*g}$
        \item $p_s$= statische druk aan pompinlaat
        \item $v_s$=snelheid aan pompinlaat
        \item $p_v$=dampdruk vloeistof
        \item We kunnen hier nog 2 begrippen in onderscheiden:\begin{itemize}
            \item $NPSH_a$
        \end{itemize}
    \end{itemize}
\end{itemize}
\paragraph{Leiding karakteristieken:}
snijpunt pompkarakteristiek en installatiekarakteristiek= werkingspunt.
\begin{itemize}
    \item Dat wil zeggen dat er op dat punt een evenwicht heerst tussen het aangeboden vermogen van de pomp en het verbuirkte vermogen van het buizenstelsel. De pomp-opvoerhoogte is altijd net zo groot als de doostromingsweersand van de installatie.
\end{itemize}
\paragraph{Automatische wijziging van het werkingspunt:}
Dat wil zeggen dat er op een dat punt een evenwicht heerst tussenhet aangeboden vermogen van de pomp en het verbruikte vermogen van het buizenstelsel. De pom-opvoerhoogte is altijd net zo groot als de doorstromingsweerstand van de installatie. Dit resulteert dan in het debie dat door de pomp geleverd kan worden.
\paragraph{Regelstrategien}
\begin{itemize}
    \item Constante verschildruk (dit komt het meest voor): De door de pomp opgewekte verschildruk over het toelaatbare debietbereik constant op de ingestelde waarde Hs tot aan de maximaalkarakteristiek
    \item Variabele verschildruk (komt niet echt voor): De door de pomp aan te houden gewenste verschildruk bijvoorbeeld lineair tussen Hs en 1/2 Hs. De gewenste verschildruk H neemt met het debiet Q af c.q. toe.
    \item Verschildruk constant/variabel: Verschildruk constant/variabel Bij deze regelingsmodus houdt de door de pomp opgewekte verschildruk tot een bepaald debiet constant op de ingestelede verschildruk (Hs 100\%). Daalt het debiet verder, dan verandert de door de pomp aan te houden verschilsdruk lineair.
\end{itemize}
\paragraph{Serie- en parallelschakeling van pompen}
Als in praktijk de benodigde opvoerhoogte of volumestroom hoger is dan een bepaald type pomp kan leveren, is het mogelijk om pompen parallel of in serie te laten werken.
\begin{itemize}
    \item Parallelschakeling:\begin{itemize}
        \item Aangezien bij parallelschakeling van pompen de druk aan de zuigzijde (zuighoogte) voor beide pompen gelijk is, zal de geleverde druk nooit hoger kunnen zijn dan de maximale opvoerhoogte van een van de twee pompen. Om te voorkomen dat de stroomrichting omkeert in deze 'zwakkere' pomp moeten er bij parallelschakeling altijd terugslagkleppen gemonteerd worden.
    \end{itemize}
    \item Serieschakeling:\begin{itemize}
        \item Bij serieschakeling van twee pompen is de perszijde van de eerste pomp gekoppeld aan de zuigzijde van de tweede pomp. Nadat de druk van de vloeistof verhoogd is in de eerste pomp, wordt deze nogmaals verhoogd in de tweede pomp. De volumestroom door beide pompen is gelijk. De gezamenlijke pompkarakteristiek kan dus worden geconstrueerd uit de afzonderlijke karakteristieken door de opvoerhoogten bij gelijke volumestroom op te tellen.
    \end{itemize}
\end{itemize}
\paragraph{Drukval in de leidingen:}
\begin{itemize}
    \item De vloeistofsnelheid
    \item De oppervlaktegesteldheid (ruwheid)
    \item De vloeistofeigenschappen zoals de dichtheid en de viscositeit
    \item[] Formule: $\Delta p$=$f*\frac{L}{D_h}*\frac{1}{2}*\rho*v^2$
    \item $\delta p$= drilva; pver de leiding [Pa]
    \item $D_h$= Hydraulische diameter [m]
    \item $\rho$=Soortelijke massa ($\frac{kg}{m^3}$)
    \item v=bulksnelheid van de stroming ($\frac{m}{s}$)
\end{itemize}
Voor hydraulische diameter van leiding te berekenen: (\textbf{Dit zeker kennen kan mss gevraagd worden op examen!})\begin{itemize}
    \item A=$\pi*r^2$
    \item O=$2*\pi*r$
    \item $D_h$=4*$\frac{A}{O}$
    \item vb:\begin{itemize}
        \item $\frac{4}{2}$=$\frac{\pi*r^2}{2*\pi*r}$$\rightarrow$
        \item $D_h$=$4*2*\frac{\pi*2^2}{2*\pi*2}$=22=diameter
    \end{itemize}
\end{itemize}
\subparagraph{Getal van Reynolds:}
Om te weten of je laminaire of turbulente flow hebt kan je deze formule gebruiken:\begin{itemize}
    \item Laminaire flow heeft slechte warmte overdracht (meeste warmte zal aan buitenkant zijn in midden weinig).
    \item Bij turbulente flow wordt warmte overdracht gelijk verdeelt over alle stromen.
\end{itemize}
\textbf{Formule:}
\begin{itemize}
    \item $Re$=$\frac{\rho*v*D_h}{rendement}$=$\frac{traagheids krachten}{Visceuze krachten}$
    \item Uw rendement=Dynamische viscositeit [$\frac{N_s}{m^3}$]
\end{itemize}
\paragraph{Waaierpompen:}
\begin{itemize}
    \item Hoe groter deze diameter hoe groter de opvoerhoogte
    \item Radiale type (lage volumestroom en hoge opvoerhoogte) tot het axiale type (hoge volumestroom en lage opvoerhoogte)
    \item als je eerder hoge druk verschillen willen aankunnen best radiale type (meerdere radiale pompen in parallel) en voor hoge hoeveelheden axiale type.
\end{itemize}
\paragraph{Eulerse opvoerdruk:}
\textbf{Formule:}$P_e$=$\frac{U_2^2-U_1^2}{2g}$\begin{itemize}
    \item U$_2$= de omtreksnelheid aan de buitenkant van de waaier 
    \item U$_1$= de omtreksnelheid aan de binnenkant van de waaier
    \item g= de zwaartekrachtversnelling
\end{itemize}
De eulerse opvoerdruk wordt gebruikt om de theoretische prestaties van een pomp te evalueren. Het houdt rekening met de ideale omstandigheden waarin alle energie die door de waaier aan de vloeistof wordt overgedragen. In de praktijk zijn er echter verliezen door wrijving, botsingen en andere factoren, waardoor de werkelijke opvoerdruk lager is dan de Eulerse opvoerdruk.
\subsubsection{Compressoren:}
\begin{itemize}
    \item Isentropische compressie ($p_{is}$)
    \item Isentropische vermogen ($P_{is}$)
    \item Werkelijk vermogen ($P_{as}$)
\end{itemize}
Hoe groter het druk verschil tussen p1 en p2 hoe meer het effect zal hebben op uw Isentropische compressie
\paragraph{Isentropisch rendement:}
$rendement_{is}$=$\frac{h_{2s}-h_1}{h_2-h_1}$
\begin{itemize}
    \item Theoretische kringloop
    \item Werkelijke kringloop
    \item Isentropisch rendement: hoe groter het verschil hoe lager uw rendement.
\end{itemize}
\paragraph{Soorten compressoren:}
Compressoren:\begin{itemize}
    \item Verdringingscompressoren:\begin{itemize}
        \item Zuigercompressoren
        \item Roterende compressoren\begin{itemize}
            \item Schroefcompressoren:\begin{itemize}
                \item Monoschroefcompressor
                \item Dubbelschroefcompressor
            \end{itemize}
            \item Scrollcompressoren
            \item Speciale compressoren\begin{itemize}
                \item Rolzuigercompressor
                \item Schoftencompressor
            \end{itemize}
        \end{itemize}
    \end{itemize}
    \item Dynamische compressoren:\begin{itemize}
        \item Centrifugaalcompressoren 
        \item Dampstraalcompressoren
    \end{itemize}
\end{itemize}
\paragraph{Zuigercompressor:}
\begin{itemize}
    \item Positieve verdringing als werkingsprincipe 
    \item Kleppen zorgen voor een afsluiten van zuigdruk of persdruk 
    \item $V_s$ slagvolume van de compressor of mechanische volume zonder dode ruimte.
    \item $V_z$ effectief beschikbaar slagvolume onder gegeven condities.
    \item Waarom? een zuigklep opent pas wanneer de druk in de cilinder lager is dan de heersende zuigerdruk boven de klep
    \item Omgekeerd bij persdruk zal de klep pas opengaan wanneer de druk in de cilinder groter is dan heersende persdruk boven de klep.
    \item Wat met de druk in de dode ruimte na compressie?\begin{itemize}
        \item Druk=zuigdruk
        \item Druk=persdruk
        \item Druk=druk in cilinder
    \end{itemize}
\end{itemize}
Volumetrisch rendement: $\lambda$=$\frac{V_z}{V_s}$=$\frac{V_c-V_Z}{V_c-V_s}$
Het volumetrisch rendement is afhankelijk van de drukverhouding, Langs empirische weg wordt dit voor alle compressoren bepaald: $\pi$=$\frac{Pc}{Po}$ (hier heeft $\pi$ niets met 3.1415 te maken maar wordt het symbool gewoon gebruikt)
\begin{itemize}
    \item[] Maximale drukverhouding voor zuigers in de praktijk is 8 want dan heb je een rendement van ongeveer 40\% kan dit nog juist alles onder 40\% is niet meer interessant.
\end{itemize}
\subparagraph{capaciteitsregeling:}
capaciteitsregeling:\begin{itemize}
    \item kleplichting:\begin{itemize}
        \item allebij de kleppen zijn open en je kan hier wat aanpassingen mee maken. Met kleplichting gaat uw rendement naar beneden
    \end{itemize}
    \item snelheidsregeling
\end{itemize}
Deellast werking:\begin{itemize}
    \item Typische voorbeelden zijn overslag en ateliers.
    \item Ontwerp condities zijn "worst-case":\begin{itemize}
        \item dT van 8 a 10K
        \item Praktisch kunnen we deze gaan verkleinen tot 5K.
        \item gemiddelde toename van de COP met 3,5\% per °C hoger verdampen 
        \item Sutringsgewijs spreken we an een vlottende verndampingstemperatuur.
        \item VSD gestuurde compressor in functie van de gevraagde koelcapaciteit leidt tevens tot een verbetering an de COP voor zuigercompressie.
    \end{itemize}
\end{itemize}
\subparagraph{Elektrische motor frequentiesturing:}
Je kan uw toerental aanpassen via een frequentiesturing.
n=$\frac{60.f}{p}$\begin{itemize}
    \item Hierbij is f uw frequentie (als je gewoon op net steekt 50hz)
    \item p is uw poolparen (2-polen=1 poolpaar,4-polen=2 poolparen,6-polen=3 poolparen)
    \item bvb:\begin{itemize}
        \item f=50hz en p=1 (1 poolpaar) $\rightarrow$ $\frac{60x50hz}{1}=$3000 toeren
        \item 
    \end{itemize}
\end{itemize}
\subsubsection{Monoschroefcompressor:}
\begin{itemize}
    \item aanzuig:\begin{itemize}
        \item Tijdens de omwenteling van de rotor loopt een groef die in open
        verbinding staat met de zuigkamer, door de standen a, b en c en
        wordt geleidelijk gevuld met zuigdamp, waarbij de glijdende tand
        van het rechter tandwiel als een aanzuigende "zuiger" fungeert.
    \end{itemize}
    \item Comperessie:\begin{itemize}
        \item Net na het positon c wordt de groef vergrendeld door een tand
        van het linker tandwiel en wordt de damp volledig omsloten
        door het binnenoppervlak van de cilindrische behuizing, de drie
        zijden van de groef en het bovenoppervlak van de tand.
        Naarmate de rotatie vordert, komt de groef in positie d, waar het
        volume is afgenomen en de damp wordt samengedrukt
    \end{itemize}
    \item Pers:\begin{itemize}
        \item Op het geometrisch gefixeerde punt waar de voorste groefrand
        en de rand van de uitlaatpoort samenvallen, vindt open
        communicatie plaats tussen de groef en de afvoerkamer. Hier
        stopt de compressie en wordt de damp afgevoerd naar de
        afvoerkamer (positie E) totdat het groefvolume tot nul is
        teruggebracht.        
    \end{itemize}
\end{itemize}
\subsubsection{Schroefcompressor-twin:}
In praktijk wordt de twin-schroefcompressor meer gebruikt.
\paragraph{Compressie cyclus:}\begin{itemize}
    \item Zuiggas wordt aangezogen door uit elkaar werkende rotoren
    \item Maximum aangezogen zuigvolume Vs
    \item Start van compressie door in elkaar draaiende rotoren
    \item Verplaatsen van het ingesloten volume richting perszijde
    \item Einde van compressie wordt bepaald door positie van uitlaat
    \item Alle gas is naar HD zijde
\end{itemize}
Bij warmtepompen zijn heeft de as meestal een dikke diameter maar kleine lengte om meer kracht te kunnen zetten.
\subsubsection{Vi in functie van drukverhouding:}
Schroefcompressor heeft geen kleppen (pD>pC)\begin{itemize}
    \item De locatie van de perspoort het maximale persdrukniveau dat wordt bereikt voordat
    het geperste gas in de persleiding wordt geduwd.
    \item De volumeverhouding is een ontwerpkenmerk van alle
    schroefcompressoren, omdat de compressor in wezen een volumereductie-apparaat is.
    \item De vergelijking van het volume ingesloten
    gas bij de aanzuiging (Vs) met het volume
    ingesloten gas dat in de compressiekamer
    achterblijft wanneer deze zich opent voor
    persgas (Vd), definieert de interne
    volumereductieverhouding van de
    compressor. Deze volume-index, of "Vi",
    bepaalt de interne drukverhouding
    \item $V_i=\frac{V_s}{V_d}$\begin{itemize}
        \item $V_i$=Volume ratio or index
        \item $V_s$=Volume at section 
        \item $V_d$=Volume at discharge
    \end{itemize}
\end{itemize}
\paragraph{overdruk en onderdruk:}
Bij overdruk moet je meer kracht geven om genoeg energie te krijgen.
\subsubsection{Scroll compressor:}
Wordt veel in de praktijk gebruikt.
\paragraph{compressie cyclus:}\begin{itemize}
    \item Positieve verdringer 
    \item Vaste spiraal en Variabele spiraal 
    \item Minimale snelheid is 40hz 
    \item Veel toegepast in chiller applicaties met meerdere compressoren in parallel 
    \item Alternatieve manier van capacitietsregeling middels "Digitale scroll" of persgas bypass.
\end{itemize}
\section{Week 6:}
\subsection{Koudemiddelen:}
voor de tweede wereldoorlog:\begin{itemize}
    \item Huishoudkoelkasten:Methychloride
    \item Industrie :Zwavel-Dioxyde
    \item Schepen: niet-giftige Kooldioxyde
\end{itemize}
Na de tweede wereldoorlog:
\subsubsection{Basis-principe:}
Koudemiddelen bezitten de eigenschappen te gaan "koken" onder de atmosfeer druk bij temperaturen ver onder het nulpunt. Koudemiddelen bestaan er in "verschillende maten en gewichten". Maar boven in synthetische of natuurlijke vorm. Natuurlijke koudemiddelen:\begin{itemize}
    \item Ammoniak R717 
    \item Koolstofdioxide R744 
    \item Propaan R290 
    \item Propyleen R1290 
    \item Isobutaan R600a 
    \item Water R718
\end{itemize}
De gene waar er 7 na de R staat is de rest de moleculaire massa.
\paragraph{Lekverlies:}
Wettelijk maximaal 5\% per jaar. Stel een koelinstallatie met R134a met een inhoud van 100kg koudemiddel. Voertuigen met benzine,LPG-, of aargasmotor:78 g/km. Equivalente km's $\rightarrow$ 5kg per jaar $\rightarrow$ 91.670km of 7,63 jaar rond rijden bij een gemiddeld van 12.000 km per jaar.\newline
GWP $\rightarrow$ Global warming potential $\rightarrow$ aaropwarmingselement: Bijdrage van een broeikasgas tot de klimaatopwarming:\begin{itemize}
    \item R717$\rightarrow$ 0
    \item R744 $\rightarrow$ 1 
    \item R290 $\rightarrow$ 3
    \item R600a $\rightarrow$ 4 
    \item R134a $\rightarrow$ 1.430 
    \item R404A $\rightarrow$ 3.922 
    \item R32 $\rightarrow$ 675
\end{itemize}
Europa probeert af te stappen van de synthetische koudemiddelen zoals R134a, R404A, R32.
\subsection{Opdeling volgens type:}
\begin{itemize}
    \item CFK's: zijn berucht voor schadelijke effecten op de ozonlaag ook schadelijk voor opwarming aarde. Door milieubeleid
    \item HCFK's:groep van synthetische koudemiddelen, nog altijd schadelijk voor ozonlaag maar minder schadelijk is ook slecht voor opwarming aarde (broeikaseffect).
    \item HFK's: synthetische koudemiddelen die de ozonlaag niet aantasten heeft wel nog altijd effect op opwarming aarde.
    \item HFO's: Nu zitten we hier, nieuwe generatie synthetische koudemiddelen als milieuvriendelijker alternatief voor HFK's heeft een lager GWP (Global warming potential) en draagt niet bij aan de afbraak van de ozonlaag.
\end{itemize}
% \subsection{Overzicht volgens GWP gewicht:}
% \begin{itemize}
%     \item A1: Non-flammable 
%     \item A2L: Mildly flammable
%     \item A3: Highly flammable 
%     \item B1: Toxic non-flammable 
%     \item B2L: Toxic less-flammable
% \end{itemize}
\subsection{Kringloop:}
\begin{itemize}
    \item Condenseror $\rightarrow$ hoge druk naar verampder $\rightarrow$ lage druk:\begin{itemize}
        \item Expansie 
        \item Via expansieventiel 
        \item Druk verandering
    \end{itemize}
    \item Verdamper $\rightarrow$ lage druk naar condensor $\rightarrow$ hoge druk
\end{itemize}
\subsection{Indeling van koelsystemen:}
\begin{itemize}
    \item Directe koeling:(koude middel loopt overal door)\begin{itemize}
        \item Pomp circulatie systeem 
        \item Natuurlijke circulatie systemen 
        \item DX of directe eXpansie 
    \end{itemize}
    \item Indirecte koeling:(gaat door een specifiek deel)\begin{itemize}
        \item Verdampende koudedrager 
        \item Koudedrager
    \end{itemize}
\end{itemize}
\paragraph{Koudedrager in gebruik bij een Indirect systeem:}
\begin{itemize}
    \item Verdam
\end{itemize}
Voorbeeld: In zieknhuizen is het water 12C/7C $\rightarrow$
\begin{itemize}
    \item Commerciele systemen hebben vooral een delta T van 2k 
    \item Industriele systemen meestal een delta T van 3K 
    \item Q=k.a.$\Delta$T 
\end{itemize}
\paragraph{Voordelen en nadelen van het gebruik van een koudedrager:}
\begin{itemize}
    \item Voordelen:\begin{itemize}
        \item Koudemiddel beperkt zicht tot de machinekamer 
        \item Kant- en klare oplossing $\rightarrow$ reductie van de installatie tijd on-site 
        \item Gebruik van "niet-traditionele" koudemiddelen zoals 
    \end{itemize}
\end{itemize}
\subsection{Koudedragers:}
Boven 0 graden kunnen we water gebruiken.\begin{itemize}
    \item Dynamische viscocity is bijna dubbel bij propyleen glycol dan bij ethylene glycol.
    \item Dus bij propyleen glycol moeten we meer energie gebruiken dan bij ethylene glycol.
    \item De reden waarom we propyleen glycol gebruiken is omdat de meeste propylene glycol veiliger/gezonder is voor de mens dan ethylene glycol.
    \item In warmte toepassingen wordt vooral ethyleen gebruik (bvb bij voertuigen)
    \item Bij omgevingen waar we met voeding werken gebruiken we vooral Propyleen glycol.
\end{itemize}

\end{document}
