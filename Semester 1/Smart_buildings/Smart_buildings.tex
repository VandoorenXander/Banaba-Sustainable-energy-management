
\documentclass[12pt]{article}

\usepackage{amsmath}

\usepackage{microtype}

\usepackage{graphicx}

\usepackage{hyperref}

\usepackage[utf8]{inputenc}

\usepackage{listings}

\usepackage{matlab-prettifier}
% % voor code syntax highlighting
% \usepackage{minted}

% beter font
\usepackage[T1]{fontenc}
\usepackage{helvet}
\renewcommand{\familydefault}{\sfdefault}
\graphicspath{{imagessoft/}}
\begin{document}
\begin{titlepage}
    \author{Xander Vandooren}
    \title{Smart buildings}
\end{titlepage}
\pagenumbering{gobble}
\maketitle
\newpage
\tableofcontents
\newpage
\pagenumbering{arabic}
\section{vak inhoud:}
\begin{itemize}
    \item[] 
\end{itemize}
\subsection{Stappenplan:}
\begin{enumerate}
    \item Meterplan
    \item Sensoren \& platformen
    \item Data protocollen en werkwijze
    \item Centrale database voor visualisatie, EMS, \dots (cloud of lokaal)
    \item Tools (Grafana,Energis,Trikthom,\dots)
\end{enumerate}
\section{Digitale meters (slimme meters)}
\begin{itemize}
    \item Registreert gas of electriciteitsverbruik
    \item Ingebouwde communicatietechnologie\begin{itemize}
        \item Userports S1 en P1
        \item verbonden met 4g via communicatiemodule => netbeheerder
        \item Optionele gasmeter is draadloos verbonden met de elekriciteitsmeter
    \end{itemize}
    \item\textbf{Voordelen}\begin{itemize}
        \item meter aflezen niet meer nodig
        \item meer inzicht in verbruik
    \end{itemize}
    \item\textbf{Slim maken}\begin{itemize}
        \item koppelen aan bepaalde apps
        \item Apparaten laten werken op de voordelige tijdstippen
        \item Eigen opgewekte energie verbruiken
        \item aansturen van batterij
        \item \dots
    \end{itemize}
\end{itemize}
\subsection{gebruikerspoort}
\begin{itemize}
    \item gebruikerspoort P1 => verbruiksgegevens\begin{itemize}
        \item %vul hier wat in
    \end{itemize}
    \item De S1 poort geeft ruwe data terug (Wij gebruiken deze poort niet)
\end{itemize}
\subsection{Besturen van digitale meter}
%vul hier verder aan
\subsection{P1 poort slimme meter hardware}
\begin{itemize}
    \item P1 poort is een seriele poort waarin je via een RJ stekker verbinding kan maken
    \item Werkt op UARTl/TTL niveau => niet staandaard
\end{itemize}
In labo werken met DSMR 4.0/4.2 instellingen
\subsection{message}
\begin{itemize}
    \item Er wordt om de seconde een telegram verstuurd
    \item Het telegram bestaat uit een stream van bytes
    \item Met telkes carriage return
    \item Elk telegram bestaat uit een aantal lijnen
\end{itemize}
\subsection{P1 poort en windows}
\begin{itemize}
    \item De p1 kabel met RJ stekker en USB fiche zorgt ervoor 
\end{itemize}
\subsection{Smart metering:}
\subsubsection{Digital meter:}
\underline{Waarom?}
\begin{itemize}
    \item Technologische evolutie
    \item Ondersteunen van de energietransitie\begin{itemize}
        \item Digitale meter fungeert als \textbf{'enabler'}
    \end{itemize}
    \item Verfijning van het dienstaanbod ondersteunen.\begin{itemize}
        \item Invoering nieuwe tariefstructuren
    \end{itemize}
\end{itemize}
\underline{Wat?}
\begin{itemize}
    \item Digitale meter != slimme meter!
    \item 4 uitleesbare meterstanden\begin{itemize}
        \item Vanaf januari 2020: 1 waarde per dag 
        \item Vanaf kwartaal 3 2020: kwartierdata
    \end{itemize}
    \item 2 gebruikerspoorten met specifiek doel:\begin{itemize}
        \item P1 poort: Gebruikerspoort voor verbruikersinformatie
        \item S1 poort: Gebruikerspoort voor geavanceerde detectie en regelingen\begin{itemize}
            \item Toestelherkenning mogelijk
            \item Poorten zijn standaard gedeactiveerd
        \end{itemize}
    \end{itemize}
\end{itemize}
Bij nieuwste digitale meters zit de S1 poort er \textbf{niet meer op}.
\subsubsection{Smappee:}
Smappee P1/S1 applicatie:
\begin{itemize}
    \item P1 lezer vanuit een fabrikant
    \item Geen eigen code meer nodig (vb. USB\& inlezen via PC of raspberry PI)
    \item Data direct gecapteerd op platform van fabrikant\begin{itemize}
        \item dashboard.smappee.net
    \end{itemize}
\end{itemize}
Smappee Infinity:
\begin{itemize}
    \item Totaaloplossing voor hoofd- en submonitoring van gebouwen
    \item Focus op energiemonitoring
    \item Sturing via eigen systeem/app
\end{itemize}
\section{Batterijopslag Victron:}
\subsubsection{Flex piramide:}
Flexibel energiesysteem wordt heel belangrijk:\begin{itemize}
    \item Om bevoorradingszekerheid en het netevenwicht op elk moment te garanderen, is hogere flexibiliteit nodig
    \item Zowel voor korte periodes (uren,dagen) als lange periodes (weken, maanden)
    \item Flexpiramide vat de belangrijke vormen van flexibiliteit samen.
    \item[]\begin{enumerate}
        \item Technische flexibiliteit: expliciete flexibiliteit, opgelegd door netbeheerder 
        \item Commerciele flexibiliteit: expliciete flexibiliteit, ingekocht op de markten 
        \item Tarieven: Impliciete flexibiliteit
        \item Infrastructuur: Dynamisch beheer van netten 
    \end{enumerate}
\end{itemize}
\subsubsection{Smappee API:}
\subsection{Batterijopslag}
\subsubsection{Introductie:}
\paragraph{Oplossing of uitdagingen?}
\begin{itemize}
    \item Wind \& groeien sneller dan alle andere energiebronnen ooit deden in het verleden.
    \item Zon groeit de laatste jaren veel sneller dan wind!
\end{itemize}
\paragraph{Hernieuwbare energie groeit snel:}
In 2024 was er in Europa (en Belgie) meer opwekking door zon + wind dan fossiel!
\paragraph{opportuniteit voor batterijen:}
\begin{itemize}
    \item 2008 economische crisis:\begin{itemize}
        \item Meer variaties, maar gemmidelde prijzen blijven gelijklopend
        \item Steeds meer negatieve prijzen door hernieuwbare energie
        \item Positieve uitschieters waren vroeger ook al aanwezig 
    \end{itemize}
\end{itemize}
\paragraph{Configuratie batterijopslag:}
\underline{Praktisch ontwerp:}
\begin{itemize}
    \item Verschillende configuraties mogelijk:\begin{itemize}
        \item AC-model:\begin{itemize}
            \item De zonnevormer kan voor of na de omvormer/laadregelaar worden geplaatst, afhankelijk van de dimensioneringscriteria.
            \item Wanneer de zonne-energie na de omvormer/laadregelaar wordt aangesloten, kan het zonnesysteem in eilandbedrijf werken.
            \item Efficientie:\begin{itemize}
                \item Direct gebruikt:97\%
                \item Opgeslagen:71\%
            \end{itemize}
        \end{itemize}
        \item DC-model:\begin{itemize}
            \item Zonne-invertor wordt vervangen door MPP tracker, gekoppeld in het DC-bus van de batterijen.
            \item MPP heeft een hoger rendement dan een omvormer voor zonne-energie
            \item Hoger rendement wanneer er een enorme hoeveelheid energie direct in de batterijen is opgeslagen (afhankelijk van levensstijl).
            \item Efficientie:\begin{itemize}
                \item Direct gebruik: 97\%
                \item Opgeslagen: 71\%
                \end{itemize}
            \end{itemize}
        \end{itemize}
    \end{itemize}
\end{document}
