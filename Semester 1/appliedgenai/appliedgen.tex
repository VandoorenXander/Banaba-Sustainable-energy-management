
\documentclass[12pt]{article}

\usepackage{amsmath}

\usepackage{microtype}

\usepackage{graphicx}

\usepackage{hyperref}

\usepackage[utf8]{inputenc}

\usepackage{listings}

\usepackage{matlab-prettifier}
% % voor code syntax highlighting
% \usepackage{minted}

% beter font
\usepackage[T1]{fontenc}
\usepackage{helvet}
\setcounter{secnumdepth}{5}
\setcounter{tocdepth}{5}
\renewcommand{\familydefault}{\sfdefault}
\graphicspath{{imagessoft/}}
\begin{document}
\begin{titlepage}
    \author{Xander Vandooren}
    \title{Applied \& generative AI}
\end{titlepage}
\pagenumbering{gobble}
\maketitle
\newpage
\tableofcontents
\newpage
\pagenumbering{arabic}
\section{Session 7:}
\subsection{Recommender systems:}
Wat zijn recommender systems?
\begin{itemize}
    \item Traditional media:\begin{itemize}
        \item Broadcasting is time limited
    \end{itemize}
\end{itemize}
Grafiek: met recommenders proberen ze dit recht te trekken.
Voorbeeld van recommender systems zijn spotify,Amazon,Netflix.
\subsubsection{Gevaren:}
\begin{itemize}
    \item Je kan vast zitten in je eigen bubbel.
    \item Op gegeven moment zelfde genre (bvb bij netflix altijd zelfde genre)
    \item Marketing 
\end{itemize}
\subsubsection{Type input data:}
\begin{itemize}
    \item Items data (bvb in films alles van karakteristieken:genre,acteurs,...)
    \item Gebruiker data (Expliciete feedback bvb duim omhoog, 4/5 sterren en impliciete data bvb kijk je een film uit of hoeveel afleveringen van serie kijk je uit etc. Impliciete data is vaak belangrijker dan expliciet)
    \item Interractie items (Hoe reageren we op iets.)
\end{itemize}
Context: bvb dag in de week of feestdag,geslacht,... 
\subsubsection{Basic concepts:}
\begin{itemize}
    \item Algorithms\begin{itemize}
        \item Non-personalized
        \item Personalized\begin{itemize}
            \item Knowledge-based filtering (KBF) (redelijk algemeen bvb gewoon als man die leeftijds groep is zal hij wss deze film graag zien)
            \item Content-based filtering (CBF) (items vergelijken met elkaar)
            \item Collaborative filtering (CF) (wss bekentste systeem)\begin{itemize}
                \item User based 
                \item Item based 
                \item Matrix factorization 
            \end{itemize}
            \item Context-aware (CARS)
            \item Hybrids\begin{itemize}
                \item Deep learning, Factorization machines...
            \end{itemize}
        \end{itemize}
    \end{itemize}
\end{itemize}
\paragraph{Item-content matrix:}
attributen $\rightarrow$strenght of connection
\paragraph{User-rating Matrix (URM):}
\begin{itemize}
    \item Explicit:\begin{itemize}
        \item ratings, thumbs-up...
    \end{itemize} 
    \item Implicit:\begin{itemize}
        \item Did a user finish the movie?
        \item Did he watch the second part?
        \item DId he fast-forward?
    \end{itemize}
\end{itemize}
User-item interaction matrix 
Meestal is er maar 1\% van uw matrix ingevuld (groter dan 0). Bij netflix is de URM density=0.02\%.
\subparagraph{Explicit ratings: Large rating scales vs small rating scales:}
\begin{itemize}
    \item Large rating scale:\begin{itemize}
        \item large user effort 
        \item fewer ratings
    \end{itemize}
    \item Small rating scale:\begin{itemize}
        \item small user effort 
        \item more ratings
    \end{itemize}
\end{itemize}
\subparagraph{Explicit ratings: even vs odd ratings:}
\begin{itemize}
    \item even ratings scale:\begin{itemize}
        \item user forced to express opinion
        \item few ratings
    \end{itemize}
    \item odd ratings scale:\begin{itemize}
        \item trend in giving neutral rating 
        \item more ratings 
        \item user more comfortable
    \end{itemize}
\end{itemize}
\subparagraph{Explicit rating: user bias:}
bvb. enkel negatieve reviews omdat mensen enkel hun tijd nemen om iets te raten als het slecht is.
\subparagraph{Top popular item:}
Grootste hoeveelheid ratings
\subparagraph{Best rated item:}
is eigelijk gewoon formule gemiddelde doen.
\subparagraph{Best rated item with shrink term:}
Items met veel hoge ratings zouden beter moeten zijn dan items met 1 top rating. Hier moet je een Determince C bias toevoegen dit is wel beetje trial en error om uit te zoeken wat juiste C moet zijn.
\subsubsection{Preprocessing: global effects formula:}
\begin{itemize}
    \item Item bias: user will think differently about a movie now than in 20 years. Due to the style, genre, how innovative it is/was.. 
    \item User bias: users can rate a movie differently based on nostalgia, because they rate higher on average, because it reminds them of a book or a special moment... 
\end{itemize}
\begin{enumerate}
    \item Averate ratings:\begin{itemize}
        \item Average ratings for all items and users
    \end{itemize}
    \item Normalized rating:\begin{itemize}
        \item To be computed for eacht user u and item i, only on non-zero ratings
    \end{itemize}
    \item Item bias:\begin{itemize}
        \item N$_i$: Number of users who have rated item i to be computed for each item i
    \end{itemize}
    \item Recompute rating:\begin{itemize}
        \item To be computed for each user u and item i, only on non-zero ratings
    \end{itemize}
    \item User bias:\begin{itemize}
        \item N$_u$: number of items i rated by user u to be computed for each user u
    \end{itemize}
\end{enumerate}
Final formula: Global effects formula to be computed for each user u and item i, only on non-zero ratings
\subsubsection{Evaluating recommender systems:}
Kwaliteit is afhankelijk van dataset $\rightarrow$ algorithm $\rightarrow$ user interface
\paragraph{Quality indicators:}
\begin{itemize}
    \item Relevance 
    \item Coverage 
    \item Novelty 
    \item Diversity 
    \item Consistency 
    \item Confidence 
    \item Serendipity
\end{itemize}
\paragraph{on-line \& off-line evaluation:}
\subparagraph{on-line: direct user feedback:}
Ask for feedback by a rating, questionnaire.
\begin{itemize}
    \item 2 problems:\begin{itemize}
        \item Not always reliable opinions 
        \item Amount of feedback should be large enough
    \end{itemize}
\end{itemize}
\subparagraph{on-line: A/B testing}
Monitor the behaviour:\begin{itemize}
    \item Set A: users with a recommendation
    \item Set B: users without recommendation
\end{itemize}
Will they behave differently? (Buy more products, spend more time on Netflix...)
\begin{itemize}
    \item But:\begin{itemize}
        \item Difficult to set up
        \item Difficult to interpret (if results bad: lack of relevance or lack of diversity?)
    \end{itemize}
\end{itemize}
\subparagraph{on-line: controlled experiments:}
Set up a mock-up application for a group of users\begin{itemize}
    \item Test application for a while and gather feedback
\end{itemize}
But:\begin{itemize}
    \item Not a real application, not the same as using real users
\end{itemize}
\subparagraph{Off-line evaluation 4 steps:}
\begin{itemize}
    \item Defining task\begin{itemize}
        \item Rating prediction (Top-N recommendation)
    \end{itemize}
    \item Dataset\begin{itemize}
        \item URM
    \end{itemize}
    \item Partitioning \begin{itemize}
        \item Model: look for connections in URM
        \item User Profile: what does user like in general?
        \item R: recommendation
        \item Metric: compare recommendation with true opinion
        \item \textbf{Three types of data needed:}\begin{itemize}
            \item Model uses data to detect rules -> model = f(X)
            \item Estimated ratings come from the model and user profile -> estimated ratings = g(model, Y)
            \item Compare estimated ratings with true value -> estimated rating <-> Z
        \end{itemize}
        \item Oppassen voor overfitting!
    \end{itemize}
    \item Metrics:\begin{itemize}
        \item Quality metrics:\begin{itemize}
            \item Error metrics:\begin{itemize}
                \item MAE \& MSE
            \end{itemize}
            \item Classification metrics :\begin{itemize}
                \item True positive 
                \item False positive 
                \item False negative 
                \item True negative 
            \end{itemize}
            \item Ranking metrics:\begin{itemize}
                \item ROC curve:Weten dat er een Area under curve is (AUC).
                \item mAP (mean average precision)
            \end{itemize}
        \end{itemize}
    \end{itemize}
\end{itemize}
\subsubsection{Content-based filtering:}
compare items based on their attributes. A user that expressed a preference for an item is likely to like similar items.
\paragraph{Recommendations based on cosine similarity}
an element of the ICM is 1 if the item has an attribute else 0.
\subparagraph{Dot product:}
many attributes in common $\rightarrow$ the two items are very similar 
\begin{itemize}
    \item[] $\hookrightarrow$ Cosine similarity is the normalization of the dot product. If two items are similair their cosine will be large.
\end{itemize}
\subparagraph{Improving on cosine similarity: shrinking}
Shrinking:\begin{itemize}
    \item Reduce similarity to take into account only most similar with large support
    \item C: Shrink term
\end{itemize}
\subparagraph{Estimating a rating:}
Weighted average of previous given ratings:\begin{itemize}
    \item r$_ui$ = the rating of user u on item i
    \item Choose item with the highest r to recommend 
    \item Or if you want an item of the top-N 
    \item Remove the denominator to save calculation
\end{itemize}
\paragraph{K-nearest neighbours (KNN):}
Similarity matrix problems:\begin{itemize}
    \item Very dense - not many 0-values: computational heavy
    \item A lot of small similarity values:\begin{itemize}
        \item A lot of very similar small values 
        \item Hard to distinguish between
        \item Noise \& low quality
    \end{itemize}
\end{itemize}
Solution: keep K most similar items\begin{itemize}
    \item K influences the quality of the recommendations:
    \item Small K: not enough data for a reliable estimation
    \item Big K: too much noise in the data
\end{itemize}
\paragraph{TF-IDF}
\subparagraph{ICM improvements: non-binary attributes:}
\subparagraph{ICM improvements: attribute weights:}
\subparagraph{ICM improvements: TF-IDF:}
Do we have to manually figure out the attribute weights and non-binary attribute values?
No! We can use a technique to do this automatically.
\begin{itemize}
    \item TF-IDF: Term Frequency - Inverse Document Frequency
    \item TF-IDF = TF * IDF $\rightarrow$ The weight of each attribute depends on its frequency
    \item Replace Item-Content Matrix by TF-IDF matrix
\end{itemize}
\begin{itemize}
    \item TF: Term frequency:\begin{itemize}
        \item Higher if attribute appears more for an item
    \end{itemize}
    \item IDF: Inverse document frequency:\begin{itemize}
        \item Higher if attribute appears more over all items
    \end{itemize}
\end{itemize}
TF-IDF: higher value to terms that are frequent for an item, but rare in the whole collection of items. One value per item - attribute pair.
\begin{enumerate}
    \item Apply TF-IDF on the Item-Content matrix to obtain a TF-IDF matrix
    \item onstruct the similarity matrix by applying cosine similarity on the TF-IDF matrix. Also use a shrink term if necessary.
    \item pply K-Nearest Neighbours on the similarity matrix
\end{enumerate}
\subparagraph{Pros and cons content-based filtering:}
\begin{itemize}
    \item Pros:\begin{itemize}
        \item 
    \end{itemize}
\end{itemize}
\subsubsection{Collaborative filtering:}
Often better quality recommendations compared to other methods.Large number of different techniques: user-based and item-based are seen here. Ratings that a user has not explicitly given to an item, can be inferred because therating is correlated across various users and items.
\paragraph{User-rating matrix:}
Matrix containing past interactions between users and
items\begin{itemize}
    \item Explicit ratings: for example rating from 1 to 5 (0 if
    missing)
    \item Implicit ratings: for example 1 if there is interaction
    with item and 0 otherwise
\end{itemize}
Implicit information is often more interesting than explicit. Most simple algorithms choose between explicit and implicit.
\begin{itemize}
    \item Search for similar users and recommend items they like:\begin{itemize}
        \item If two users give similar ratings to several items, we assume that they share the same opinion. 
        \item Better to have a number between 0 and 1
    \end{itemize}
\end{itemize}
\subsubsection{Model based vs memory based systems:}
Model based:when a new user is added, the similarity
matrix does not change the model significantly (Item based).
Memory based: when a new user is added, the
similarity matrix changes (User based).
\end{document}
