
\documentclass[12pt]{article}

\usepackage{amsmath}

\usepackage{microtype}

\usepackage{graphicx}

\usepackage{hyperref}

\usepackage[utf8]{inputenc}

\usepackage{listings}

\usepackage{matlab-prettifier}
% % voor code syntax highlighting
% \usepackage{minted}

% beter font
\usepackage[T1]{fontenc}
\usepackage{helvet}
\setcounter{secnumdepth}{5}
\setcounter{tocdepth}{5}
\renewcommand{\familydefault}{\sfdefault}
\graphicspath{{imagessoft/}}
\begin{document}
\begin{titlepage}
    \author{Xander Vandooren}
    \title{Sustainable business management}
\end{titlepage}
\pagenumbering{gobble}
\maketitle
\newpage
\tableofcontents
\newpage
\pagenumbering{arabic}
\section{Week 1:}
\subsection{MDG's (Millenium development goals):}
eerder gefocused op ontwikkelings landen ivgm westerse landen.
\begin{itemize}
    \item 2000-2015
\end{itemize}
\subsection{SDG's (Sustainable development goals):}
Wereldwijde doelstellingen voor duurzame ontwikkeling
\begin{itemize}
    \item 2015-2030
\end{itemize}
\subsection{Inhoud:}
\subsubsection{Klimaatakkoord van Parijs (2015 COP 21):}
Om de temperatuurstijging ruim onder de 2 graden celsius te houden (t.o.v de pre-industriele periode). Liefst maximum 1.5 graden celsius.
\paragraph{Sustainable Finance Action Plan (2018)}
Wetgeving met 3 primaire doelstellingen kapitaal laten stromen richting duurzame investeringen voor meer transparantie en LT-denken en mainstream maken van duurzaamheid in financieel risicio management.
\subsubsection{Green Deal (2019)}
Een antwoord op de uitdagingen van het akkoord van Parijs met als finale doelstelling om het eerste klimaatneutrale continent te worden tegen 2050.
\begin{itemize}
    \item De non-financial reporting directive herbekijken. (was een richtlijn uit 2014 om bedrijven hun duurzaamheids maatregelen te laten publiceren).
\end{itemize}
\paragraph{EU taxonomy (2020):}
Een classificatiesysteem met als doelstellingen om te bepalen welke economische activiteiten bijdrage aan de milieudoelstellingen. Dit heeft tot doel om kapitaal richting duurzame investringen de duwen, investeerders keuzes laten maken die in lijn liggen met het parijsakkoord en greenwashing vermijden.
\begin{itemize}
    \item[] Alligned met taxonomy als;
    \item Eligble (=voorkomen in de lijst)
    \item Een substantiele bijdrage tot minstens 1/6 milieu doelstellingen 
    \item Geen schade toebrengen aan 1 van de andere doelstellingen (= Do no significant harm)
    \item Repsecteren van minimale sociale grenzen
\end{itemize}
Aandeel 'environmentally sustainable' activiteiten:\begin{itemize}
    \item Omzet (\%)
    \item Capex (\%)
    \item Opex (\%)
\end{itemize}
\paragraph{Corporate sustainability reporting directive (CSRD):}
Herziening van de non-financial reporting directive:
\begin{itemize}
    \item $\hookrightarrow$ Akkoord van Parijs (2015)
    \item $\hookrightarrow$ Publicatie Actieplan Duurzame Financiering (maart 2018)- Publicatie Europese Green Deal (December 2019)
    \item[]\begin{itemize}
        \item Sustainable Finance Disclosure regulation-verordering (EU) 2019/2088
    \end{itemize}
\end{itemize}
Wie moet er daar aan voldoen?
\begin{itemize}
    \item $\hookrightarrow$ Alle EU - beursgenoteerde ondernemingen
    \item Bedrijven die minimaal aan 2 van de 3 criteria voldoen:\begin{itemize}
        \item Vanaf meer dan 250 werknemers 
        \item Bedrijven met meer dan 50 miljoen euro in omzet 
        \item Bedrijven met meer dan 25 miljoen euro balanstotaal
    \end{itemize}
\end{itemize}
Verplichte audit
\paragraph{ Corporate social due dilligence directive (CSDDD) (2024):}
= transparantie wordt verwacht van bedrijven met betrekking tot het naleven van mensenrechten en duurzaamheidsvereisten. Wat? Passende maatregelen nemen (een inspanningsverplichting) in overeenstemming zijn met de ernst en de waarschijnlijkheid van de diverse effecten.
Concreet:\begin{enumerate}
    \item De integratie van de zorgvuldigheidseisen in hun beleid 
\end{enumerate}
\paragraph{EUDR (anti-ontbossingswetgeving):}
\paragraph{Green claims directive:}
Greenwashing:\begin{itemize}
    \item $\hookrightarrow$ De praktijk van het geven van een valse indruk van de milieueffecten of voordelen van een product, wat consumenten kan misleiden.\begin{itemize}
        \item Generieke Claims die slechts over een deel van het product gaan (vb. een dop is recycleerbaar, maar de rest niet en je claim slaat op de hele verpakking).
        \item Claims dat een product een beter effect heeft op het milieu omdat de producenten emissies compenseert.
        \item Duurzaamheidslabels die niet zijn goedgekeurd. (eigen 'eco' labels)
    \end{itemize}
\end{itemize}
Beleidsmatige aanpassingen:\begin{itemize}
    \item Belgische aanpassingen (2021);
    \item Europese wijzigingen (2024);\begin{itemize}
        \item In lijn met de Circular Economy action plan (2019)
        \item Empowering consumers for the green transition directive. (ECGT): \begin{itemize}
            \item Zorgt er voor dat consumenten beschermt worden tegen misleidende marketingpraktijken, en meer informeren hoe lang produceren meegaan. (Tegen 2026 om te zetten in nationale wetgeving.)
        \end{itemize}
        \item Voorstel Groene claims (2023) voor een transparant gebruik van milieuclaims die betrouwbaar, vergelijkbaar en controleerbaar zijn. Er dient duidelijke onderbouwing te zijn, verificatie door een onafhankelijke partij en transparante conmmunicatie. Ecolabels dienen goedgekeurd te worden door de E. Commissie.
    \end{itemize}
\end{itemize}
\subsection{ESG cyclus:}
\section{Week 2:}
\subsection{Carbon Footprint:}
\subsubsection{Klimaatverandering:}
Inkomende zonne energie die dan grotendeels terug naar buiten gaat.
Vandaag onbalans Omdat we veel uitgaande energie kwijt spelen als greenhouse gases. 91\% van deze energie zit in de oceaan. Onze oceanen zijn soort van warmtebuffers. 
\paragraph{Grenzen van onze planeet:}
Een oefening die om de zoveel jaar gedaan wordt om te kijken hoever we van de grenzen van onze planeet zitten voor specifieke stoffen etc. Novel entities zijn stoffen die standaard nie bestaan die door de mens gemaakt zijn die een impact hebben op het klimaat en het milieu.
\paragraph{Stijging van de gemiddelde temperatuur:}
Vanaf de jaren 70 begint de opwarming van de aarde snel omhoog te gaan. Dit is door de start van de echte industrialisatie van alles waardoor er veel vervuiling gebeurde. We kunnen zien dat er een rechtstreekse link is tussen de hoeveelheid broeikasgassen en de opwarming van de aarde. De opwarming gebeurdt door het broekaseffect (broeikasgassen). klein stukje van de zon wordt door de wolken geabsorbeerd deel door de ozon laag een groot deel door de aardoppervlak en klein deel gaat terug weg van aarde.
\paragraph{Oefening:}
\begin{itemize}
    \item Welke zijn de broeikasgassen:\begin{itemize}
        \item Koolstofdioxide
        \item Methaan
        \item Distikstofoxide
        \item Fluorkoolwaterstoffen
        \item Perfluorkoolstoffen
        \item Zwavelhexafluoride
        \item Stikstoftrifluoride
    \end{itemize}
\end{itemize}
\paragraph{Broeikasgassen:}
\begin{itemize}
    \item Broeikasgassen zijn gassen in de atmosfeer met het vermogen om warmtestraling te absorberen en geleidelijk in alle richtingen weer af te geven:\begin{itemize}
        \item H$_2$O: is het belangrijkste van nature aanwezige broeikasgas 
        \item CO$_2$: ontstaat bij de natuurlijke afbraak van plantaardig of dierlijk materiaal, maar wordt tevens opgenomen door planten in de fotosynthese. Komt ook vrij bij:\begin{itemize}
            \item Opwekking van energie door verbranding van fossiele bradnstoffen 
            \item Ontbossing, vooral in de tropen voor omschakeling naar landbouw
            \item Industriele processen zoals de productie van cement en kalk 
            \item Activiteiten in de petrochemie
        \end{itemize}
        \item CH$_4$ (Methaan): Ontstaat bij de ontbinding van plantaardig materiaal in vochtige gebieden komt ook wel vrij bij:\begin{itemize}
            \item De landbouw (rijstvelden, door darmgisting bij herkauwers, gebruik van mesthopen)
            \item De behandeling van huishoudelijk afval (Storten en compostering)
            \item De epxploitatie en distributie van aardgas (lekken, onvolledige of niet verbrand gas).
        \end{itemize}
        \item N$_2$O (lachgas): komt voor bij:\begin{itemize}
            \item De landbouw 
            \item De chemische industrie 
            \item De verbranding van fossiele brandstoffen voor huisverwarming en transport
        \end{itemize}
        \item Stikstof (N) zit in veel verschillende broeikasgassen.
        \item O$_3$ (Ozon):\begin{itemize}
            \item Ozon is van nature aanwezig in de atmosfeer
        \end{itemize}
        \item Industriele broeikasgassen (gemaakt door de mens):\begin{itemize}
            \item Gefluoreerde koolwaterstoffen (CFK's, HCFK's ,HFE's , HFK's , PFK's)\begin{itemize}
                \item Doen dienst als koelvloeistof (koelkasten en airco), solvent (o.m voor de schoonmaak van elektronica), brandblusmiddel en worden gebruikt in de productie van aluminium en kunststofschuim.
                \item CFK's en HCFK's zijn verantwoordelijk voor de afbraak van de stratosferische ozon (op grote hoogte) en zijn of worden verboden door het Protocol van Montreal (1987).
            \end{itemize}
            \item SF$_6$ Zwavelhexafluoride:\begin{itemize}
                \item Wordt gebruikt in transformatoren en dubbel glas (geluidsisolatie)
            \end{itemize}
        \end{itemize}
    \end{itemize}
\end{itemize}
GWP (global warming potential) zeker kennen. We vergelijken de GWP van gassen met de GWP van CO$_2$ die 1 is.
De reden waarom we vaak GWP in 100 jaar bekijken is omdat het 50-100 jaar duurt tegen dat CO$_2$ weg is uit onze atmosfeer. 74\% van de broeikasgassen is CO$_2$ en 17\% is methaan. De rest zijn de andere gassen.
\paragraph{Oefening:}
\begin{itemize}
    \item Hoe ziet de verdeling van broeikasgassen er uit in belgie?
    \item Te vinden op klimaat.be (wordt uitgebaat door de federale overheidsdienst.)\begin{itemize}
        \item CO$_2$ 86\% 
        \item Methaan CH$_4$ 7,4\%
        \item N$_2$O 4,3\%
        \item Gefluoreerde gassen 2,3\%
    \end{itemize}
    \item welke sectoren hebben de grootste uitstoot:\begin{itemize}
        \item Industrie (verbranding,energie,processen elk apart)
        \item Transport 
        \item Landbouw 
        \item Residentiele verwarming 
        \item tertiaire verwarming
    \end{itemize}
\end{itemize}
In belgie is Industrie de grootste sector waar uitstoot vrij komt. (Kalek en cement wegen waarschijnlijk redelijk zwaar door in de industrie). We kunnen wel zien dat we sedert de jaren 90 een neatieve groei in broeikasgassen hebben.
\subsubsection{Effecten van klimaatverandering:}
\begin{itemize}
    \item Klimaat=weer?\begin{itemize}
        \item Het klimaat is het geheel van weersomstandigheden (oppverlaktevariabelen zoals temperatuur,wind en neerslag) die een bepaalde plaats gedurende een bepaalde periode karakteriseren. (milieu is vaak lokaal terwijl klimaat vaak eerder globaal is)
    \end{itemize}
\end{itemize}
\paragraph{The intergovernmental Panel on Climate Change (IPCC)}
\begin{itemize}
    \item Opgericht in 1988 door de VN-milieuprogramma (UNEP) en de wereld meteorologische organisatie (WMO)
    \item Evalueert de risico's van klimaatverandering 
    \item Publiceert hier regelmatig rapporten over 
    \item Momenteel zitten we aan het 6de rapport (2021)
\end{itemize}
IPPC houd zich vooral bezig met de wetgeving rond milieu.
\begin{enumerate}
    \item Loss of animal species
    \item Heat humidity risks to human health
    \item food production impacts
\end{enumerate}
\subsubsection{De carbon footprint:}
Wat is CO$_2$-neutraal ondernemen?\begin{itemize}
    \item Om CO$_2$-neutraal te zijn moet er een evenwicht zijn tussen de CO$_2$ uit de atmosfeer.
\end{itemize}
Er zijn koplopers binnen de bedrijfswereld:\begin{itemize}
    \item Zijn reeds lang bezig met klimaat en CO$_2$
    \item Hebben een ambitieuze doelstelling op korte termijn 
    \item Zowel op energie als op CO$_2$
\end{itemize}
Google is een van de eerste bedrijven die effectief bezig is met carbon neutraal worden.De rest volgt:\begin{itemize}
    \item Steeds meer en meer bedrijven zetten doelstellingen:\begin{itemize}
        \item CO$_2$-neutraliteit
        \item 100\% hernieuwbaar
        \item SBTI (Science based targets initiative)
    \end{itemize}
    \item Corporate sustainability reporting (+500 werknemers)
    \item Andere bedrijven voelen druk als leverancier (scope 3)
\end{itemize}
Binnen de carbon footprint bestaan 3 standaarden:\begin{itemize}
    \item PAS 2050 
    \item Greenhouse Gas Protocol 
    \item ISO 14067
\end{itemize}
Lijken allen sterk op elkaar
\paragraph{Scope's}
Over het algemeen wordt gesproken van 3 scope's waarbinnen de CO$_2$-emissies vallen:\begin{itemize}
    \item Scope 1 (Direct):\begin{itemize}
        \item Company facilities
        \item Company vehicles 
        \item Processes
    \end{itemize}
    \item Scope 2 (Indirect):\begin{itemize}
        \item Purchased electricity, steam, heating and cooling for own use
    \end{itemize}
    \item Scope 3 (Indirect):\begin{itemize}
        \item Upstream activities 
        \item Downstream activities
    \end{itemize}
\end{itemize}
Voor we opsplitsen in verschillende scopes kijken we eerst naar het niveau van de organisatie. In praktijk niet altijd even eenvoudig want een bedrijf is niet steeds 1 op 1.
\paragraph{Verdeling volgens aandelen:}
\begin{itemize}
    \item Hierbij is het bedrijf verantwoordelijk voor de emissies volgens zijn hoeveelheid in aandelen.
    \item Veelal wordt er gekeken naar het \% eigendom
    \item 
\end{itemize}
\paragraph{Verdeling volgens bestuur (controle)}
\begin{itemize}
    \item Hierbij is het bedrijf 100\% verantwoordelijk voor de emissies waar ze zeggenschap over heeft
    \item Activiteiten waar ze eigenaar zijn maar geen operationele controle over hebben, worden niet meegerekend. bvb. Dus als er niemand van Colruyt operationele controle heeft over Kriket moeten zij niet betalen, indien ze wel iemand hebben met operationele controle moeten ze 100\% betalen.
\end{itemize}
als op basis van benadering van aandelen minder dan 50\% is dan heb je geen operationele controle. Uw financiele controle kan wel nog altijd bvb 50\% zijn.
\paragraph{Opdracht:}
\paragraph{Scope 1:}
Emissies uit processen:\begin{itemize}
    \item Afkomstig zijn van biologische, mechanische, chemische of andere activiteiten die
    verband houden met een industrieel proces. Enkel brandstoffen die op de site zelf
    worden gebruikt.
    \item De optimale activiteitsgegevens voor dit item zijn gebaseerd op metingen van de
    hoeveelheden broeikasgassen die rechtstreeks door de processen worden uitgestoten.
    \item Emissies gerelateerd aan chemische reacties kunnen worden afgeleid uit de
    stoichiometrische formule van de reacties.
    \item Verbranding voor processen / verwarming van gebouwen / noodgroepen
    \item Enkel brandstoffen die op de site zelf worden gebruikt 
    \item Omvat verschillende brandstoffen\begin{itemize}
        \item Aardgas 
        \item Stookolie 
        \item \dots
    \end{itemize}  
    \item Bepaling van de hoeveelheid op basis van de factuur\begin{itemize}
        \item Verbruik in kWh/MWh of liter/kg
        \item Som per maand
        \item Verdeling van de jaarafrekening
    \end{itemize}
    \item Bepaling van de hoeveelheid op basis van de factuur\begin{itemize}
        \item Verbruik in kWh/MWh of liter/kg
        \item Som per maand
        \item Verdeling van de jaarafrekening
    \end{itemize}
\end{itemize}
Transport:\begin{itemize}
    \item (voornamelijk) verbranding van brandstoffen van bedrijfswagens of transport in
    eigendom van het bedrijf.
    \item omvat verschillende brandstoffen\begin{itemize}
        \item Diesel
        \item Benzine 
        \item LPG 
        \item CNG 
        \item Elektriciteit
        \item \dots
    \end{itemize}
    \item Bepaling van de hoeveelheid op basis van de factuur\begin{itemize}
        \item Verbruik in kWh/MWh of liter/kg 
        \item Som per maand
    \end{itemize}
    \item Gegevens opvragen bij leasingmaatschappij
\end{itemize}
Koelgassen:\begin{itemize}
    \item Vluchtige emissies van koelgassen
    \item Iedere koelinstallatie heeft een bepaald lekdebiet\begin{itemize}
        \item (Grote) koelinstallaties worden regelmatig onderhouden
        \item Tijdens onderhoud wordt het koelmiddel bijgevuld
    \end{itemize}
    \item Bepaling van hoeveelheid op basis van de factuur\begin{itemize}
        \item Verbruik in kg 
        \item Totaal van het jaar/type
    \end{itemize}
    \item Vaak wordt het verbruik van koelmiddelen niet bijgehouden, alternatief is de berekening van het lekdebiet van een installatie
    \item Eerst wordt de inhoud aan koelmiddelen bepaald (technische gegevens) 
    \item Daarna wordt het theoretisch lekdebiet bepaald\begin{itemize}
        \item Industrieële koelinstallatie = 15\%
        \item Koelinstallatie tertiair (water) = 15\%
        \item Koelinstallatie tertiair (lucht) = 10\%
        \item \dots
    \end{itemize}
    \item Finaal wordt nog een onzekerheid van 50\% ingerekend
\end{itemize}
bij Scope 1: verbranding van brandstoffen nemen we normaal HHV voor de kosten. ipv LHV, bij emissiefactoren transport minder fluctuatie dan bij verbranding brandstoffen. Bij koelgassen schakelen ze meer over naar Ammoniak installaties vooral bij grote koelings installaties.
\paragraph{Scope 2:}
\begin{itemize}
    \item Indirecte uitstoot (behoort dus niet inherent tot de onderneming)
    \item Emissies die ontstaan door de opwekking van\begin{itemize}
        \item Elektriciteit 
        \item Warmte 
        \item Koude
    \end{itemize}
\end{itemize}
Elektriciteit:\begin{itemize}
    \item Bepaling van de hoeveelheid op basis van de factuur\begin{itemize}
        \item Verbruik in kWh/MWh
        \item Som per maand
        \item Verdeling van de jaarafrekening
    \end{itemize}
    \item Je kan levering van groene stroom controleren via \url{https://www.vreg.be/nl/controleren-hoe-groen-uw-stroom-groencheck} 
    \item Hiervoor heb je je EAN code nodig (staat op factuur)
\end{itemize}
\subparagraph{Market based vs location based:}
Groene of grijze energie?
\begin{itemize}
    \item Market:\begin{itemize}
        \item Wat zegt het energie contract\begin{itemize}
            \item stel 100\% groen = 0.024 kg CO2/kWh 
            \item Mix = mix leveren of market based rapporteren
        \end{itemize}
    \end{itemize}
    \item location based:\begin{itemize}
        \item Belgie Co2-EF (167,49 gc02/kWh)
    \end{itemize}
\end{itemize}
Dus als we gaan rapporteren dan gaan we onze location based en market-based rapporteren.4

Warmte/koude\begin{itemize}
    \item Warmte en koude
    \item Bepaling van de hoeveelheid op basis van de factuur\begin{itemize}
        \item Verbruik in kWh/MWh of GJ (warmte)
        \item Som per maand
        \item Verdeling van de jaarafrekening
    \end{itemize}
\end{itemize}
$((E_afval*EF_afval)+(E_fossiel*EF_fossiel))/stoom_uit (in kWh of GWJ)=EF_stoom warmte (CO2/GJ)$
% de 3 variabelen hier zijn uw E_afval,stoom_uit en E_fossiel. want uw EF_afval is redelijk vast en uw EF_fossiel is ook redelijk vast.
% Bij afval heeft bio de zwaardste verbrandingsfactor.
% De scope 1 \&2 dingen kunnen zeker gevraagd worden op examen! (Sustainable_Energy_Management_Les_3_Carbon_Footprint vanaf pagina 30).
\section{Week 4:}
\subsubsection{Scope 3:}
Duiding Upstream-Downstream:\begin{itemize}
    \item Upstream:Dit zijn alle activiteiten die plaats hebben gevonden voordat het product bij jou
    terechtkomt.
    \item Downstream:Dit zijn alle activiteiten die plaatsvinden nadat het product bij jou terecht is
    gekomen.    
\end{itemize}
Indirecte emissies in scope 3:\begin{itemize}
    \item Aankoop van grondstoffen/diensten/verpakkingen
    \item Transport van goederen (upstream/downstream)
    \item Afval
    \item Professionele verplaatsingen
    \item Woon-werk-verkeer
    \item Investeringen
    \item Geleasde activa (upstream/downstream)
    \item Franchises
    \item Gebruik van product
    \item Einde levensduur product (end-of-life)
\end{itemize}
\paragraph{grondstoffen:}
Mogelijke grondstoffen in uw productieproces:
\begin{itemize}
    \item Metalen (nieuwe/gerecycleerd)
    \item Kunststof (nieuw)
    \item Andere 
    \item Landbouw 
\end{itemize}
\paragraph{Aankopen/diensten:}
opzich alles dat bij boekhouden passeerd moet opgenomen worden.
\begin{itemize}
    \item "Aankopen" in uw kantoor
    \item "Diensten" vor uw kantoor (vb. poetsdienst,internet,telefonie,webservices)
    \item Mogelijke verpakking van afgewerkte producten
\end{itemize}
\paragraph{Transport van goederen:}
\begin{itemize}
    \item Indien u upstream en/of downstream transport in rekening brengt, kunnen volgende
    waarden worden toegepast
    \item Breng het totaal aantal kilometers in kaart ieder product:\begin{itemize}
        \item Bepaal de afstand van leverancier tot klant of van je bedrijf tot de klant
        \item Wat is het gewicht van de levering?
        \item Breng het aantal ritten in kaart
        \item Vermenigvuldigen de factoren met elkaar afstand x gewicht x aantal bewegingen
        \item Omrekenen voor ieder type transport
    \end{itemize}
\end{itemize}
\paragraph{afvalstoffen}
\begin{itemize}
    \item Data terug te vinden in het IMJV of in het afvalstoffenregisterindien van toepassing voor
    uw bedrijf
    \item Afvalstoffen worden er geregistreerd met bijhorende tonnages
    \item Mogelijke afvalstoffen:\begin{itemize}
        \item PMD
        \item Metalen 
        \item Restafval 
        \item Papierafval 
        \item Industrieel gevaarlijk afval
    \end{itemize}
    \item Lozing van water:\begin{itemize}
        \item Het gebruik van water wordt niet opgenomen als “grondstof”
        maar de lozing en zuivering ervan wordt meegenomen in de
        afvalstoffen 
        \item Wanneer er door het bedrijf zelf afvalwater wordt gezuiverd dan
        zit dit reeds vervat in het verbruik van het bedrijf
        \item Afvalwater = 0,262 kg CO2/m³
    \end{itemize}
    \item Belangrijke noot (nevenstromen):\begin{itemize}
        \item de CO2-reductie door materiaal (dat voor u afval is in uw
        productieproces) naderhand te laten recycleren of
        hergebruiken door een ander bedrijf, kan niet van jouw
        footprint worden afgetrokken. Het vertaalt zich wel minimaal
        door de andere eindverwerking van de grondstof. En het is heel
        waardevol en dus zeker het vermelden waard in jullie
        communicatie.
        \item Bijvoorbeeld: Een voedingsbedrijf is producent van
        pastaproducten. De pasta die tijdens het proces uitvalt wegens
        kwaliteit of opstart,... wordt door een verwerker periodiek
        opgehaald om deze reststroom te verwerken in dierenvoeder.
        Het voedingsbedrijf moet de uitval van pasta rekenen als afval
        en mag dus in de berekening van hun CO2-voetafdruk geen
        rekening houden met de valorisatie van dit product bij de
        verwerker.        
    \end{itemize}
\end{itemize}
\paragraph{Professionele verplaatsingen:}
\begin{itemize}
    \item Vliegreizen:\begin{itemize}
        \item U kan de vliegreizen opsplitsen in 2 (of meer) categorieën:\begin{itemize}
            \item Tot 3.500km per vlucht (en minder dan 50 zitjes): 0,519 kg CO2/passagier.km 
            \item Meer dan 3.500km per vlucht (en meer dan 100 zitjes): 0,210 kg
            CO2/passagier.km
        \end{itemize}
        \item Bepaal het totaal aantal kilometers per categorie en bereken de uitstoot
    \end{itemize}
    \item wagen (zie woon-werk verkeer) 
    \item Trein- en busreizen:\begin{itemize}
        \item Bepaal het aantal passagier kilometers per categorie en bereken de uitstoot 
        \item Metro: 5,7 kg CO2/1.000 passagier.km
        \item Trein: 48,4 kg CO2/1.000 passagier.km
        \item TGV: 3,69 kg CO2/1.000 passagier.km
        \item Bus: 0,1812 kg CO2/passagier.km
    \end{itemize}
\end{itemize}
\paragraph{woon-werkverkeer:}
\begin{itemize}
    \item Info woon-werkverkeer van werknemers te bekomen via HR of via een bevraging.
    \item De verplaatsingen van werknemers met een bedrijfswagen werden al in rekening
    gebracht in scope 1.
    \item Breng het totaal aantal kilometers in kaart voor benzine en diesel - stappenplan:\begin{itemize}
        \item 'Afstand enkele route' x 2 x 'aantal keer weg-en-weer per dag' = aantal kilometers
        per dag 
        \item Daarna alle kilometers voor diesel optellen, alsook voor benzine
        \item Vermenigvuldigen met aantal werkdagen (doorgaans 220 per jaar)
        \item Omrekenen naar verbruik in liters (reken gemiddeld verbruik van 5 liter/100km)\begin{itemize}
            \item Benzine = 2,724 kg CO2/liter
            \item Diesel = 3,099 kg CO2/liter
        \end{itemize}
    \end{itemize}  
    \item Opmerking:\begin{itemize}
        \item In feite kan een bedrijf ook de verplaatsingen van bezoekers meerekenen in scope 3
        indien gewenst. 
        \item Dit is bijvoorbeeld relevant voor de event-sector
    \end{itemize}
\end{itemize}
\paragraph{Investeringen}
\begin{itemize}
    \item Gebouwen worden in rekening gebracht in de CO2-footprint
    gedurende de periode van afschrijving stappenplan:\begin{itemize}
        \item Bepaal het aantal m² 
        \item Emissiefactor gebouwen:\begin{itemize}
            \item Industrieel gebouw (betonstructuur) = 825 kg CO2/m²
            \item Kantoren en garage (betonstructuur) = 650 kg CO2/m²
        \end{itemize}
        \item Ga na of het gebouw nog steeds in een afschrijving zit en
        over welke periode deze loopt
        \item Deel de voetafdruk door het aantal jaren van de afschrijving 
        \item Voorbeeld: Industrieel gebouw van 1.000 m2 werd gebouwd
        in 2005 en wordt op 30 jaar afgeschreven. De CO2-uitstoot bedraagt:
        1.000 m2 x 825 kg CO2/m2 ÷ 30 jaar = 27,5 ton CO2
    \end{itemize}
    \item Alternatief 0,360 kg CO2/€
    \item Aankoop transportmiddelen:\begin{itemize}
        \item Voertuigen: 5.500 kg CO2/ton 
        \item Elektrische fiets: 261 kg CO2/stuk
    \end{itemize}
    \item Aankoop machines:\begin{itemize}
        \item Kleine machines: 0,7 kg CO2/€ 
        \item Grote productiemachines op afschrijving verdeeld over de afschrijvingstermijn: 5.500
        kg CO2/ton
    \end{itemize}
    \item Parking:\begin{itemize}
        \item Volgens hetzelfde principe met afschrijvingstermijn: 319 kg CO2/m²
    \end{itemize}
\end{itemize}
\paragraph{Leasing:}
\begin{itemize}
    \item Afhankelijk van de voorwaarden van de leasing, wordt er
    anders gerekend.
    \item Voorbeelden:\begin{itemize}
        \item Je huurt een gebouw maar hebt geen invloed op de
        aankoop van energie en het zit mee vervat in de
        huurprijs: inrekenen in scope 3
        \item Je huurt een gebouw, maar sluit zelf een contract aan
        voor stroom: gebouw in scope 3, stroom in scope 2
        \item Je least een bedrijfswagen, maar moet zelf de
        tankbeurten bekostigen: leasing van de wagen in scope
        3, brandstof in scope 1
    \end{itemize}
    \item Gebouwen huren: 0,360 kgCO2/€
    \item Personenwagens huren: 0,700 kgCO2/€
\end{itemize}
\paragraph{Franchises:}
\begin{itemize}
    \item Franchisegevers dienen in deze categorie rekening te houden met emissies die
    ontstaan door de exploitatie van franchises (d.w.z. de scope 1- en scope 2-emissies
    van franchisenemers) 
    \item Indien er veel franchises zijn kan men er ook
    voor kiezen om te werken met een benadering
    van samples
\end{itemize}
\paragraph{Gebruik van het product:}
\begin{itemize}
    \item Directe emissies bij het gebruik:\begin{itemize}
        \item Elektriciteit (vb. wasmachine) 
        \item Brandstoffen (vb. heftruck) 
        \item Broeikasgassen (vb. warmtepompen)
    \end{itemize}
    \item Indirecte emissies bij het gebruik:\begin{itemize}
        \item Kledij (dient gewassen te worden) 
        \item Voeding (dient opgewarmd te worden)
    \end{itemize}
    \item De berekening omvat het gebruik van alle producten verkocht in het berekende jaartal.
\end{itemize}
\subsection{Climate game:}
\begin{itemize}
    \item Adviser: David Deals
    \item Questions:\begin{itemize}
        \item Question 1:Coal makes up three-quarters of the CO2 produced by electricity.
        \item Answer 1: Phase out coal plants in wealthy countries over 10 to 20 years
        \item Question 2:Cutting emissions in all areas needs new technologies. Each innovation costs five points and you will double your investment if your choices take off. Select one or more of the following:
        \item Answer 2:Better electric vehicle batteries,Carbon capture from factories
        \item Question 3: traffic congestion:
        \item Answer 3: ban of all normal cars in 2035 investing more in electric cars
        \item Question 4:Buildings and construction make up almost one-third of energy use. Cities are growing fast in developing countries. What will you do?
        \item Answer 4: Ban global coal and oil boiler sales by 2025 and roll out heat pumps cap thermostats at 20C in winter
        \item Question 5: Whether cement, steel or glass, manufacturing causes lots of pollution. Industry alone makes up about a quarter of energy-related CO2 emissions. What do you want to do first?
        \item Answer 5: Demand that 40\% of steel and 15\% of plastic come from recycled materials by 2030
        \item Question 6:Your goal is to keep global warming to 1.5C, and this means dealing with greenhouse gases other than CO2. Global methane emissions account for about one-third of human-caused warming. Do you want to deal with methane now?
        \item Answer 6: Focus on methane
        \item Question 7:Agriculture and the loss of trees is the largest cause of greenhouse gas emissions after energy. How will you tackle those and improve nature's ability to absorb CO2?
        \item Answer 7: reduce deforestation by two-thirds by 2050 and plant hundreds of millions of hectares of native forest.
    \end{itemize}
    \item Round two:\begin{itemize}
        \item Question 1: Polling shows some voters think you're moving too fast and they are worried electricity bills will soar. How will you get them onside?
        \item Answer 1: Start a global advertising campaign on climate change and green jobs.
        \item Question 2: You've decided how you want to deal with coal. What will you do next about electricity?
        \item Answer 2: Invest \$2tn a year in renewable electricity (about 1\% of projected global GDP in 2030) 
        \item Question 3:Despite your efforts so far, carbon already in the atmosphere is causing the world to get hotter. How many points would you like to spend on adaptation, to help improve people's lives and protect them from heatwaves, floods, wildfires and rising seas in the future?
        \item Answer 3:Adaptation is really important, build infrastructure that will help people live on a hotter planet.
        \item Question 4: Transport is not only about cars — what will you do to address emissions from planes?
        \item Answer 4: Mandate 20\% low-carbon fuel for aviation. Cap the number of long-haul flights.Introduce a frequent flyer tax
        \item Question 5: What will you do next to tackle emissions from buildings?
        \item Answer 5: Deal with both old and new buildings: retrofit 20\% of the old ones and declare all new buildings must be zero carbon
        \item Question 6: Economic incentives can also help cut carbon emissions. Polluters in Europe are paying about \$100 for every tonne that is released. Do you want to set a global carbon price?
        \item Answer 6: Yes, set a carbon price now of \$1000 per tonne
        \item Question 7: Improving recycling was important for cutting industrial emissions. What's next?
        \item Answer 7: Commision a research project into innovative low-carbon steelmaking
    \end{itemize}
    \item Round three:\begin{itemize}
        \item Question 1: Electricity must be the core of the new energy system. How do you want it to be generated in 2050? Renewables,nuclear,Fossil fuels* *includes traditional fossil energy, as well as low emissions fossil energy through carbon capture and storage. Some answer options exclude energy sources that make relatively small contributions.
        \item Answer 1: 71\%,8\%,20\%
        \item Question 2: Population and economic growth mean that road transport is on the rise. How will you deal with it from 2030 to 2050?
        \item Answer 2: Push for electric vehicles to make up half of cars on the road by 2050
        \item Question 3: Policy changes and new technologies won't be enough — human behaviour will have to adapt too. How will you reduce emissions from the food that people eat?
        \item Answer 3: Invest in research to improve crop production
        \item Question 4:Building emissions have fallen, but not by enough. How will you get them to zero by 2050?
        \item Answer 4: Launch a mass training programme for builders to retrofit.
        \item Question 5:There is a shortage of critical minerals such as copper, lithium and cobalt because of high demand for their use in batteries and wind turbines. What do you do?
        \item Answer 5: Invest into research into new and recycled materials
        \item Question 6:Industry emissions are the hardest to get to zero. What will you do to cut them by 2050?
        \item Answer 6: Set national goals to gradually increase hydrogen use, and use of carbon capture and storage.
    \end{itemize}
\end{itemize}
\section{Gastles:}
Duurzaamheid \& ecodesign:\begin{itemize}
    \item Waarom?
    \item Wat?
    \item Wie?
    \item Hoe?
\end{itemize}
\subsection{Waarom:}
\subsubsection{Idealisme \& Noodzaak:}
\begin{itemize}
    \item Geschat wordt dat we in 2030 twee planeten nodig hebben
    \item Geen commerciele visserij meer in 2048 | 1000 ton solid waste elke 5 secondne
\end{itemize}
\subsubsection{Materialen schaarste:}
Hoeveel materialen zijn er nog over voor x aantal jaar?\begin{itemize}
    \item Sommige kritieke materialen raken SNEL op
    \item de geglobaliseerde winning van grondstoffen kan grote problemen met zich meebrengen 
\end{itemize}
\subsubsection{EU regelgeving:}
Duidelijke en meetbare doelen: Klimaatneutraal tegen 2050:\begin{itemize}
    \item er komt beleid:\begin{itemize}
        \item Informatief (bvb Ecolabel,product pasport)
        \item Economisch (bvb belastingen,subsidies)
        \item Regelgevend (bvb verbod op PFAS)
    \end{itemize}
\end{itemize}
\subsubsection{Stakeholder management:}
\subsection{Wat:}
\begin{itemize}
    \item Lineiare economie
    \item Circulaire economie:\begin{itemize}
        \item Biologische cyclus:(wordt nog niet super veel mee gedaan)\begin{itemize}
            \item hout 
            \item \dots
        \end{itemize}
        \item Technische cyclus: (waar we nu vooral mee bezig zijn)\begin{itemize}
            \item liefst in zo klein mogelijke stappen te hergebruiken en proberen af te stappen van dingen te vernietigen.
        \end{itemize}
    \end{itemize}
\end{itemize}
\subsubsection{Waardebehoud:}
lineaire economie:
waarde toevoegen $\rightarrow$ use case $\rightarrow$ vernietigen
De circulaire economie:
Waarde toevoegen $\rightarrow$ use case (reparatie onderhoud) $\rightarrow$ waarde behouden (hergebruik, refurbish, her-fabricage, recycle)
\paragraph{R-strategien:}
\subparagraph{Rethink:}
\begin{itemize}
    \item Nieuwe business modellen (e.g. Product-as-a-service)
    \item Nieuwe consumptie patronen 
    \item Nieuwe stakeholders in de waardeketen
\end{itemize}
Fabrikant blijft eigenaar van de producten:
\begin{itemize}
    \item Verhoogde productkwaliteit+ verbeterd onderhoud 
    \item Meer mogelijkheden
\end{itemize}
Personalisere/Upcycling/deeleconomie: (zul je minder snel wegsmijten)\begin{itemize}
    \item Gepersonaliseerde producten 
    \item Unieke items
\end{itemize}
\subparagraph{Reduce:}
Verminder maeriaal gebruik:\begin{itemize}
    \item Nudging: iemand aanmoedigen of overtuigen om iets te doen op een manier die zachtaardig is in plaats van krachtig of direct
\end{itemize}
\subsubsection{Ecodesign in de circulaire economie:}
Ecodesign= Duurzaam productontwerp: Alle mogelijke Producten / services / verpakking\begin{itemize}
    \item Ecodesign in de circulaire economie:\begin{enumerate}
        \item levenscyclus denken.
        \item Systeem denken:\begin{itemize}
            \item Verpakkingssysteem: product + verpakking + logistieke keten 
            \item Voorbeeld: Shampoo, 80\% van vloeibare zeep is water. oplossing hiervoor is iets wat we vroeger gebruikten, een bar.
        \end{itemize}
        \item Ketensamenwerking:
    \end{enumerate} 
\end{itemize}
\subsection{Wie?}
Samenwerking binnen de waardeketen:\begin{itemize}
    \item Leveranciers $\rightarrow$ ken je materialen 
    \item Afvalverzamelaars 
    \item Afvalverwerkers $\rightarrow$ bezoek eens een recyclingbedrijf 
    \item Distributeurs 
    \item Andere bedrijven $\rightarrow$ zoek synergieen: bv het delen van afvalstromen:
    \item bvb Tomato masters \& Omegabaars 
\end{itemize}
Samenwerking met sociale actoren:\begin{itemize}
    \item Overheid
    \item Federaties
    \item Kennis instituties 
    \item Innovators 
    \item Banken en verzekeringen 
    \item Sociale economie
    \item Sociaal ondernemen zal zeer belangrijk zijn in de circulaire economie van vandaag en morgen $\rightarrow$ maatwerkbedrijven kunnen vele taken aanleren, op lokaal niveau, zijn flexibel en zijn kosten efficient.\begin{itemize}
        \item Groep Maatwerk  (Colruyt, Coca Cola, Pidpa,\dots)
        \item Herwin (kringwinkel,Oxfam,blue bike,\dots)
        \item Andere voorbeelden: MolWol,Wash-it
    \end{itemize}
\end{itemize}
\subsection{Hoe?}
Ecodesign guidelines= Design for X 
\begin{itemize}
    \item Design for longevity:\begin{itemize}
        \item Waarom gooien consumenten producten weg?\begin{itemize}
            \item Ze zijn het beu? Maak het product personaliseerbaar, product - consumer band kan sterk zijn 
            \item Uit de mode? Modulaire constructie en upgradebaar  
            \item Is het kapot? Maak een robuust product, verbeter de zwakke plekken en bied reparatie- en onderhoudsmogelijkheden aan.
        \end{itemize}
    \end{itemize}
    \item Deisgn for sustainable behaviour (Nudging):\begin{itemize}
        \item Hoe kun je de consument ertoe aanzetten zich duurzaam te gedragen?\begin{itemize}
            \item Make it fun 
            \item Make it easy 
            \item Make it rewarding
        \end{itemize}
    \end{itemize}
    \item Design for reuse:\begin{itemize}
        \item Maak het gemakkelijk om het product opnieuw te gebruiken?\begin{itemize}
            \item Een product kan vele producten voor eenmalig gebruik vervangen 
            \item Pas op voor te grote tegenreacties 
            \item Producten kunnen een tweede leven/dubbel gebruik krijgen 
        \end{itemize}
    \end{itemize}
    \item Design from waste:\begin{itemize}
        \item Nieuwe producten gemaakt uit afval:\begin{itemize}
            \item Geef werkkleding een nieuw leven:
            \item Als je afval gebruikt als grondstof:\begin{itemize}
                \item Let op met risico op contaminatie: Je moet 100\% op de hoogte zijn van de chemische structuur van het afvalmateriaal en het risico op besmetting voor het milieu en de consument.
            \end{itemize}
        \end{itemize}
    \end{itemize}
    \item Design for the sharing economy:\begin{itemize}
        \item Nieuwe sharing concepten vragen om nieuwe producten vereisten:\begin{itemize}
            \item Robuste producten, denk aan het gedrag van 'alle' gebruikers
            \item Producten die gemakkelijk te kuisen en onderhouden zijn 
            \item Producten upgradebaar en repareerbaar
        \end{itemize}
    \end{itemize}
    \item Design for assembly/ disassembly:\begin{itemize}
        \item Elimineer, simplificeer en standardizeer:\begin{itemize}
            \item Maak producten modulair, eenvoudig te (de)monteren $\rightarrow$ tijd-/geldwinst in het productieproces 
            \item Kies het laagste aantal onderdelen 
            \item Beperk soorten verbindingstechnieken 
            \item Gebruik klikverbindingen (afneembaar/niet afneembaar) 
            \item Kies indien mogelijk standaard materialen/afmetingen (schroeven,bouten etc.) Gemakkelijk te gebruiken met standaard gereedschap (Schroevendraaier etc.)
            \item Geen lijm!
        \end{itemize}
    \end{itemize}
    \item Design for repair \& maintenance:\begin{itemize}
        \item Maak producten die gemakkelijk schoon te maken zijn 
        \item Onderhoudsintervallen moeten duidelijk worden 'gepushed' of vastgelegd in onderhoudscontacten $\rightarrow$ slimme producten 
        \item Modulair ontwerp helpt ook bij onderhoud en reparatie 
        \item Gemakkelijk toegang tot alle beschikbaarheid van onderhoudsonderdelen en componenten 
        \item Voorbeelden:\begin{itemize}
            \item Fruitpers 
            \item Koffie machine 
            \item Luchtcompressor onderhoud
        \end{itemize}
        Eu right to repair act:\begin{itemize}
            \item Implementatie binnen de 2 jaar 
            \item Nauwe product scope (merendeel electronische producten) 
            \item Reasonable price and access to repair or original parts
            \item[] De huidige reparatie incentives in Oostenrijk en Frankrijk zijn succesvol $\rightarrow$ tot 200 €/jaar
        \end{itemize}
    \end{itemize}
    \item Design for remanufacturaing:\begin{itemize}
        \item Product-dienstcombinaties maken.
        \item Ontwerp een kwaliteitscontrolesysteem om de producten en componenten na gebrik te testen. $\rightarrow$ Mogelijkheden voor digitalisering 
        \item Ontwerp robuuste onderlelen/componenten van het product die tijd- en technologie-onafhankelijk zijn.
        \item Cluster productcomponenten met dezelfde verwachte gemiddelde levensduur. 
        \item voorbeeld Caterpillar
    \end{itemize}
    \item Design for recycling/composting:\begin{itemize}
        \item Gebruik van recyleerbare/composteerbare materialen 
        \item Houd er rekening mee dat bestaande logistieke recyclingketens in elk land/regio anders zijn.
    \end{itemize}
    \item Design for composting:\begin{itemize}
        \item Biobased plastics <=> Fossil based plastics 
        \item Biodegradable plastics <=> non-Biodegradable plastics 
        \item Compostable plastics => beide biobased \& Biodegradable plastics
    \end{itemize}
\end{itemize}
\subsection{Materiaal selectie}
Een duurzaam materiaal bestaat niet, het draait allemaal om het duurzaam gebruik van materialen. Coated papier is zelf slechter dan mono-plastic. (papieren rietjes hebben een plastieken coating). er is 10 liter water nodig om 1 blad A4 te printen, papier is zeer water intensief. voor de productie van 1 kilo papier is er 2-3 kilo boom nodig. Papier is nog altijd 26\% van het afval op stortplaatsen. Glas,metaal en keramiek zijn de beste materialen om te hergebruiken om dat ze super lang kunnen hergebruikt/gerecycled kan worden. Papier die voor voedsel wordt gebruikt moet verplicht nieuw papier zijn, mag niet recycled zijn.
\subsubsection{Recycleren van metaal:}
\begin{itemize}
    \item Geen degradatie van metalen 
    \item echter wel oxidatie => verlies van materiaal 
    \item Oppervlaktebehandelinge
\end{itemize}
\subsubsection{Recycleren van glas:}
Sooer per soort zonder contaminatie= geen kwaliteitsverlies
Glasscherven in een glasoven zorgen voor ongeveer 30\% minder CO2-uitstoot.
\subsubsection{Recycleren van electronica:}
Urban mining wordt steeds belangrijker: uit 1 ton GSM's kan je meer koper,zilver en goud halen dan uit 1 ton erts.
\section{Week 6:}
\subsection{EU ETS:}
European Union Emissions Trading System:\begin{itemize}
    \item Startdatum: 1 januari 2005 
    \item Doel: op een kostenefficiënte wijzen de uitstoot in energie-intensieve industriële sectoren en
    fossiele elektriciteitscentrales verminderen
    \item Wat: Emissiehandel is de handel in emissierechten: het recht om broeikasgassen uit te stoten. Met één emissierecht mag een bedrijf één ton koolstofdioxide (CO2) uitstoten.
    \item Toepassingsgebied:\begin{itemize}
        \item +/- 40\% van de uitstoot van broeikasgassen in de EU
        \item meer dan 8.000 installaties (toen UK erbij was: +/- 10.000 installaties) 
        \item 27 EU-lidstaten, IJsland, Noorwegen en Liechtenstein (\& koppeling met ETS van Zwitserland) 
        \item Alleen directe emissies (op de site) dus enkel scope 1 
        \item Broeikasgassen: CO2, N2O, PFC 
    \end{itemize}
    \item In Vlaanderen vallen momenteel ongeveer 200 BKG-installaties onder het EU ETS, die samen goed zijn voor 32Mt CO2eq., wat neerkomt op ongeveer 40 \% van de Vlaamse broeikasgasuitstoot.
    \item EU-Directive 2003/87/EG (EU ETS-richtlijn):\begin{itemize}
        \item Beschrijft het emissierechtensysteem (EU ETS)
        \item Annex I: geeft aan welke activiteiten onder het toepassingsgebied van het EU ETS vallen 
        \item 'Guidance on Interpretation of Annex I of the EU ETS Directive' verduidelijkt verder
    \end{itemize}
    \item Vanaf 2012:\begin{itemize}
        \item Toevoeging luchtvaart 
    \end{itemize}
    \item Vanaf 2024:\begin{itemize}
        \item Toevoeging scheepsvaart
        \item Afvalverbranding (enkel rapportage)
    \end{itemize}
    \item Bedrijven kunnen emissierechten kopen (o.a. via veilingplatform) en verkopen op de EU ETS-
    markten. Als een bedrijf meer uitstoot dan het aantal ontvangen emissierechten, moet het extra
    rechten kopen.:\begin{itemize}
        \item Product benchmark:\begin{itemize}
            \item baksteen,staal,\dots
        \end{itemize}
        \item Heatbenchmark:\begin{itemize}
            \item GJ warmte = meetbare warmte (stoom,warm water,thermische olie,\dots)
        \end{itemize}
        \item Fuelbenchmark:\begin{itemize}
            \item GJ brandstof
        \end{itemize}
        \item deze resultaten worden vergeleken met de andere bedrijven die deze benchmarks opgegeven hebben en als je in de top 10 zit van laagste verbruik/emissies krijg je meer rechten.
    \end{itemize}
    \item Bedrijven krijgen kosteloos emissierechten toegewezen op basis van combinatie van:\begin{itemize}
        \item Historische activiteitsgegevens 
        \item Blootstelling van de sector aan koolstoflekkage 
        \item Relevante benchmarks, gelieerd aan evolutie v/d gemiddelde uitstoot v/d 10\% meest efficiënte installaties
    \end{itemize}    
    \item Bedrijven die onder hun toewijzing blijven, kunnen ongebruikte rechten verkopen.
    \item Het aantal gratis rechten wordt iedere periode verlaagd
    \item Elke exploitant van een BKG-installatie moet zijn emissies monitoren o.b.v. een geverifieerd en goedgekeurd monitoringplan (zie presentatie VBBV)
    \item Na elk kalenderjaar moeten de emissies zoals gemeten en berekend o.b.v. dit monitoringplan
    worden gerapporteerd aan het VEKA via een geverifieerd emissiejaarrapport (EJR).
    \item Deadline: steeds 14 maart van het jaar X+1 (EJR + VR).
    \item Het jaarlijks emissiejaarrapport moet vooraf worden geverifieerd door een geaccrediteerde en
    onafhankelijke verificateur.
    \item Deze externe verificateur zal nagaan of de methodologie zoals opgenomen in het goedgekeurde
    monitoringplan correct is toegepast en of alle data in het emissiejaarrapport correct en accuraat zijn.
    \item Na verificatie zal deze externe verificateur een verificatierapport afleveren, dat samen met het emissiejaarrapport door de exploitant moet ingediend worden.
    \item In Vlaanderen gebeurt de opstelling en indiening van het geverifieerd emissiejaarrapport
    momenteel via de webapplicatie ETS: Emissiejaarrapport e-loket
    \item Het VEKA controleert steekproefsgewijs de geverifieerde emissies zoals opgenomen in het
    emissiejaarrapport, en keurt deze uiterlijk op 15 april goed (zal vermoedelijk verlaat worden naar 15 juni).
    \item De geverifieerde emissiecijfers kunnen dus nog aangepast/gecorrigeerd worden vooraleer ze
    worden goedgekeurd
    \item Na indiening van de geverifieerde emissies, bezorgt VEKA deze aan de nationale
    registeradministrateur.
    \item Jaarlijks worden de reeds beschikbare geverifieerde emissies van alle installaties die in Europa onder het EU ETS vallen, gelijktijdig op 1 april vrijgegeven in het Europees register (EUTL).:\begin{itemize}
        \item Deze kunnen geraadpleegd worden op de website van de europese comissie
    \end{itemize}
    \item CO$_2$-prijs:$\rightarrow$ Bepaald door de marktwerking
\end{itemize}
\subsubsection{Reductiedoelstelling:}
\begin{itemize}
    \item Het plafond van totaal beschikbare emissierechten (uitgedrukt in CO2-eq.) heeft een jaarlijkse reductiefactor waardoor de jaarlijkse hoeveelheid beschikbare emissierechten stelselmatig daalt
    \item De daling van beschikbare emissierechten komt overeen met een reductie van 62\% tegen 2030 tgo. 2005
\end{itemize}
\subsubsection{EU ETS 2:}
\begin{itemize}
    \item Voor de gebouwde omgeving, transport en overige sectoren (voornamelijk kleinere industriële bedrijven die niet onder het bestaande ETS vallen) is een nieuw ETS 2
    \item Start in 2027
    \item Doelstelling: 42\% in 2030 ten opzichte van de uitstoot in 2005
    \item Verplichte monitoring van uitstoot en betalen van een CO₂-prijs door de brandstofleveranciers
    \item Monitoring vanaf 2025
    \item Indienen emissierechten vanaf 2027
    \item Tot 2030 is de prijs beperkt
\end{itemize}
\subsubsection{CBAM (Carbon Border Adjustment Mechanism)}
\begin{itemize}
    \item CO$_2$-prijs voor geïmporteerde goederen uit landen die niet onder EU-ETS vallen
    \item Koolstoflekkage tegengaan - gelijk speelveld voor EU versus Non-EU 
    \item Momenteel zes sectoren
    \item Overgangsfase (Oktober 2023 - December 2025):\begin{itemize}
        \item Importeurs moeten per kwartaal de CO$_2$-emissies rapporteren van de geïmporteerde goederen.
        \item Geen CBAM-certificaten in te dienen
    \end{itemize}
    \item Definitieve fase (Vanaf 1 januari 2026):\begin{itemize}
        \item Jaarlijkse rapportering van de CO$_2$-emissies van de geïmporteerde goederen en indienen van CBAM-certificaten
        \item Prijs rechtstreeks gelinkt aan ETS-emissierechtenmarkt
        \item Graduele toepassing in combinatie met afname van gratis emissierechten binnen deze 6 sectoren
    \end{itemize}
    \item Welke emissies worden gerapporteerd?\begin{itemize}
        \item Scope 1 \& 2 van productieproces en grondstoffen (indien deze ook onder CBAM vallen)
    \end{itemize}
\end{itemize}
\section{laatste les:}
\subsection{CSRD:}
Waarom?\begin{itemize}
    \item Nood aan regelgeving die:\begin{itemize}
        \item verplichte 
        \item Transparante 
        \item Vergelijkbare
    \end{itemize}
    \item[]rapportage omtrent ESG-topics mogelijk maakt
    \item In het verleden reeds vrijblijvende initiatieven (GRI,\dots) en verplichte regelgeving (NFRD)\begin{itemize}
        \item Maar boden onvoldoende structuur \& dekten de scope niet
    \end{itemize}
\end{itemize}
Vertaling in het beleid:\begin{itemize}
    \item Budgetten: de helft van het geld nodig CO2 neutraal tegen 2050 komt van de europese Unie (€503 miljard voor CLimate and environment) de rest moet eigelijk komen van bedrijven/mensen 
    \item Landschap:\begin{itemize}
        \item European green deal 
        \item $\hookrightarrow$ SFAP (Sustainable finance action plan)
        \item EU taxonomie: (weten dat er groen en zwarte activiteiten zijn)\begin{itemize}
            \item SFDR 
            \item CSRD ESRS
        \end{itemize}
    \end{itemize}
\end{itemize}
Bestaand framework:\begin{itemize}
    \item Verplichte rapportering zonder framework:\begin{itemize}
        \item NFRD 
    \end{itemize}
    \item Niet verplicht framework:\begin{itemize}
        \item GRI 
        \item TCFD 
        \item SASB 
        \item GHG Protocol 
    \end{itemize}
\end{itemize}
Nieuw framework:\begin{itemize}
    \item Verplichte rapportering volgens framework:\begin{itemize}
        \item EU:\begin{itemize}
            \item CSRD 
            \item EU Taxonomy 
            \item CSDD 
            \item SFDR
        \end{itemize} 
        \item Nationaal:\begin{itemize}
            \item LKSG 
            \item Transparenct act
        \end{itemize}
        \item Niet-EU:\begin{itemize}
            \item ISSB 
            \item SEC (USA) 
            \item SSRC (CN)
        \end{itemize}
    \end{itemize}
\end{itemize}
Beperkingen:\begin{itemize}
    \item Beperkte impact op bedrijfsstrategie en gedrag 
    \item Greenwashing blijft mogelijk 
    \item Weinig tot geen controle
\end{itemize}
Voordelen:\begin{itemize}
    \item Toename beschikbaarheid data 
    \item Gestandaardiseerd framework 
    \item Audit assurance
\end{itemize}
Verplichte duurzaamheidsrapportage:\begin{itemize}
    \item CSRD (Corporate Sustainability Reporting Directive):\begin{itemize}
        \item Een verplichte Europese richtlijn over duurzaamheidsrapportering4
        \item De rapportage is geen doel op zich, maar een middel om bedrijven te
        doen groeien naar duurzaamheid.
    \end{itemize}
    \item Eerst voor grote ondernemingen met +500 werknemers (NFRD): (voor boekjaar 2024)\begin{itemize}
        \item Grote beursgenoteerde bedrijven 
        \item Banken 
        \item Verzekeringsmaatschappijen 
        \item Andere
    \end{itemize}
    \item Tweede groep rapporteert vanaf 2026 (FY 2025): (voor boekjaar 2025)\begin{itemize}
        \item Bedrijven die voldoen aan 2 van de 3 criteria:\begin{itemize}
            \item +250 werknemers 
            \item +€50 miljoen onzet 
            \item +€25 miljoen balans
        \end{itemize}
    \end{itemize}
    \item vanaf 2027 moeten beursgenoteerde KMO's die aan 2/3 criteria voldoen (voor boekjaar 2026)
    \item Vanaf 2029 Non-EU ondernemingen met EU-dochterondernemingen $\rightarrow$ €150 miljoen euro omzet in EU. (voor boekjaar 2028)
\end{itemize}
CSRD structuur:\begin{itemize}
    \item CSRD - Corporate Sustainability Reporting Directive: Rapporteringsrichtlijn die dienst doet als framework voor de bijhorende standaarden $\downarrow$ 
    \item ESRS - European Sustainability Reporting Standards: Uitgewerkte rapporteringsstandaarden om te kunnen voldoen aan de CSRD\begin{itemize}
        \item Algemene richtlijnen Verplicht voor alle ondernemingen(Governance; Strategie; Impact, Risico's \& Opportuniteiten; Maatstaven (KPI's) en doelen)
        \item Specifieke richtlijnen Sector- en Bedrijfsspecifiek Enkel indien materieel (ESG-topics)
    \end{itemize} 
\end{itemize}
ESRS: 2 algemene \& 10 thematische standaarden:\begin{itemize}
    \item Verplichte principes \& dubbele materialiteits-analyse $\rightarrow$ ESRS 1: algemene vereisten (0 datapunten)
    \item Rapportage gebieden: governance, Strategie, Impact, Risico's \& opportuniteiten maatstaven en doelen $\rightarrow$ ESRS 2: algemene toelichtingen (195 datapunten)
    \item In totaal 1144 datapunten
\end{itemize}
\subsubsection{Stappenplan duurzaamheidsrapportage:}
De dubbele materialiteitsanalyse:\begin{itemize}
    \item Dubbele Materialiteitsmatrix - de impact op en van het bedrijf\begin{enumerate}
        \item Interne materialiteitsoefening\begin{itemize}
            \item Selectie relevante thema's en uitzetting materialiteitsmatrix
        \end{itemize}
        \item Externe materialiteitsoefening
    \end{enumerate}
    \item Financiele impact (Outside-in) $\rightarrow$ Impact op het bedrijf\begin{itemize}
        \item Financiele impact 
        \item Outside-in het perspectief 
        \item Het effect van het milieu en de maatschappelijk op de organisatie 
        \item Risico's en opportuniteiten
    \end{itemize}
    \item Materiele impact (Inside-out) Impact van het bedrijf:\begin{itemize}
        \item Materiele impacten 
        \item Inside-out perspectief 
        \item Het effect van de organisatie op het milieu en de maatschappij
    \end{itemize}
\end{itemize}
\end{document}
