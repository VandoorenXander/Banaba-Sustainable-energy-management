
\documentclass[12pt]{article}

\usepackage{amsmath}

\usepackage{microtype}

\usepackage{graphicx}

\usepackage{hyperref}

\usepackage[utf8]{inputenc}

\usepackage{listings}

\usepackage{matlab-prettifier}
% % voor code syntax highlighting
% \usepackage{minted}

% beter font
\usepackage[T1]{fontenc}
\usepackage{helvet}
\setcounter{secnumdepth}{5}
\setcounter{tocdepth}{5}
\renewcommand{\familydefault}{\sfdefault}
\graphicspath{{Imageselek/}}
\begin{document}
\begin{titlepage}
    \author{Xander Vandooren}
    \title{Elektriciteitsproductie en distributie}
\end{titlepage}
\pagenumbering{gobble}
\maketitle
\newpage
\tableofcontents
\newpage
\pagenumbering{arabic}
\section{Hoofdstuk 1:}
\subsection{Wat is energie:}
\textbf{De mogelijkheid om arbeid te verrichten of warmte te genereren.}
\begin{itemize}
    \item Kinetische energie (bewegingsenergie bv auto die rijdt)
    \item Potentiele energie (opgeslagen energie bv een opgetilde bal of opgespannen veer)
    \item Thermische energie (zoals warmte)
    \item Stralingenergie (zonlicht,microgolf)
    \item Nucleaire energie (kernenergie)
    \item Chemische energie (opgeslaan in atomen, bv fossiele brandstoffen, batterijen)
    \item Elektrische energie (energie van bewegende ladingen om elektronische apparaten aan te drijven)
\end{itemize}
\textbf{Energie kan niet gecreeerd of vernietigd worden, alleen omgezet van de ene vorm naar de andere (Wet van behoud van energie)}
\subsection{Elektrische stroom:}
\subsubsection{\textbf{Atomen:}}
\begin{itemize}
    \item Elektrische ladingen = fysieke eigenschap van deeltjes:\begin{itemize}
        \item Protonen zijn positief geladen
        \item Elektronen zijn negatief geladen
        \item Neuronen zijn niet geladen
    \end{itemize}
    \item Gelijke ladingen stoten af
    \item Tegengestelde ladingen trekken aan
\end{itemize}
voorbeeld: Een koolstofatoom is niet geladen omdat het 6 protonen en 6 elektronen heeft.

Elke attoom streeft naar stabiliteit (buitenste schil gevuld met $e^-$)
\subsubsection{\textbf{Edelgassen:}}
\begin{itemize}
    \item Standaard buitenste schil volledig gevuld
    \item Stabiel op zichzelf
    \item Zal niet binden
\end{itemize}
\subsubsection{\textbf{Bindingen:}}
\begin{itemize}
    \item Geen volle buitenste schil?\begin{itemize}
    \item $\Rightarrow$ ofwel Elektronen delen $\Rightarrow$ covalente binding
    \item $\Rightarrow$ Ofwel/volledig afstaan/opnemen $\Rightarrow$ ionen
\end{itemize}
\end{itemize}
\subsubsection{\textbf{Elektrisch geleidend materiaal:}}
Minstens 1 elektron dat 'los' zit (buitenste schil) = vrij elektron.
Bij bindingen tussen zo een atomen worden overschotten losgelaten $\rightarrow$ vliegt wat in het rond.

Natrium en koper hebben allebij 1 valentie $e^-$ toch geleidt koper beter.
\subsubsection{\textbf{Geleiding:}}
\begin{itemize}
    \item Stroom onstaat wanneer er zich aan de ene kant $e^-$ bevinden dan aan de andere kant in een gesloten circuit = potentiaalverschil.
    \item De kant met meer $e^-$ 'duwen' de elektronen naar de andere kant met minder elektronen= stroom.
    %hier afbeeldingen toevoegen.
\end{itemize}
\subsubsection{\textbf{Geleiders vs isolators vs halfgeleiders:}}
\begin{itemize}
    \item Geleider geleidt de stroom gemakkelijk = vrije elektronen op de buitenste schil.\begin{itemize}
        \item alle metalen zijn goeie Geleiders
        \item ook de aarde is een goeie geleider
        \item supergeleiders bieden geen enkele weerstand tegen geleiding\begin{itemize}
            \item vooralsnog enkel mogelijk bij abosolute nulpunt (-273 graden celsius)
        \end{itemize}
    \end{itemize}
    \item Isolatoren geleiden de stroom niet, of weinig.\begin{itemize}
        \item Bijna of geen vrije elektronen op de buitenste schil 
        \item Rubber,glas,olie,\dots
        \item Beschermt mens tegen elektrocutie
    \end{itemize}
    \item Halfgeleiders situeren zicht tussen geleiders en isolatoren\begin{itemize}
        \item In normale toestand niet geleidend (Silicium)
        \item Door dopering (toevoeging vreemde atomen ) wel geleidend.
    \end{itemize} 
\end{itemize}
\subsection{Spanning:}
\begin{itemize}
    \item Potentiaalverschil = spanningsverschil = spanning (U of E) in Volt (V)
    \item \textbf{Altijd} tussen 2 punten (Richting pijl naar grootste potentiaal)
    \item We zeggen dus altijd spanning over \dots .
    \begin{figure}[ht]
        \centering
        \includegraphics[scale=0.6]{Meten_Spanning}
    \caption{hoe spanning te meten met multimeter}
    \end{figure}
    %slide tonen
\end{itemize}
\subsection{Elekrische lading en stroom:}
\begin{itemize}
    \item De hoeveelheid elektrische lading (Q) in een systeem wordt uitgedrukt in Coulomb (C)
    \item 1 C = $\pm$6,24 * $10^18$ aantal elektronen
    \item Elektrische stroom ontstaat wanneer deze ladingen beginnen te 'lopen'.
    \item Elektrische stroom is dus de hoeveelheid lading dat per seconde door een bepaald punt passeert = C/s = Ampere (A).
    \item We zeggen dus altijd stroom door \dots .
    %foto multimeter stroom
    \begin{figure}[ht]
        \centering
        \includegraphics[scale=0.6]{Meten_Stroom}
    \caption{hoe stroom te meten met multimeter}
    \end{figure}
\end{itemize}
\subsubsection{\textbf{Stroomkring:}}
Is een gesloten pad waarlangs elektrische ladingen kunnen bewegen (dus stroom voeren).
\subsection{Weerstand:}
In een stroomkring \textbf{moet} dus een verbruiker aanwezig zijn, iets dat de energie dan de bron omzet in een andere energiesoort $\rightarrow$ Anders \textbf{kortsluiting}.
Elektrische weerstand (R) is de tegenstand die stoffen bieden aan de stroom ($\Omega$)
Ook elektrische verbruikers bieden een weerstand tegen de stroom.
$V = I*R$ hierbij is V uw spanning, I uw stroom en R uw weerstand.
Je kan dit ook omvormen naar $R= U/I$ en $I=U/R$.
\subsubsection{\textbf{Serieschakeling van weerstanden:}}
Het potentiaal verschil zal zich verdelen over de weerstanden.
De stroom door alle weerstanden is dezelfde $\rightarrow$ I1=I2=I3.
%afbeelding toevoegen van circuit
U1=I1*R1, 
U2=I2*R2, 
U3=I3*R3 
\subsubsection{\textbf{parallelschakeling van weerstanden:}}
potentiaal over alle weerstanden is dezelfde.
stroom zal zich verdelen.
I1=U/R1,
I2=U/R2,
I3=U/R3
Er kan geen stroom verdwijnen dus I=I1+I2+I3.
\subsubsection{\textbf{bij te grote spanning voor led:}}
Indien je een led hebt met een maximum spanning van 1.2v en een maximum stroom van 0.03A maar je hebt een voeding (batterij van 5V) dan moet je een \textbf{voorschakelweerstand} plaatsen.
In dit geval weten we dat er 1.2V over de led gaat dus is er nog 3.8V over die over onze voorschakelweerstand staat. We weten ook dat er maximum 0.03A mag staan op onze led dus we kunnen berekenen hoeveel ohm onze voorschakelweerstand zal zijn door $R=U/I$ te doen
dus 3.8/0.03=126.7 (127 ohm afgerond).
\subsection{Energie,verbruik en vermogen:}
Wanneer er stroom loopt door weerstand $\Rightarrow$ warmte-ontwikkeling door botsing van elektronen. De hoeveelheid elektrische energie die daarmee verbruikt wordt is evenredig met de stroom, de weerstand en de tijd. $W=I^2.R.t$ Hierbij is W ons verbruik, I onze stroom, R onze weerstand en t onze tijd.

Een omgevormde versie van deze formule is $W= U.I.t$ hierbij gebruiken we de spanning door de wet van ohm om te vormen.
Elektrische energie wordt dus uitgedrukt in $V*A*s=Ws$ $Ws$ (Watt* seconde)= Joule.
Dit betekent dat een elektrische energie van 1Ws (1J) verbruikt wordt, wanneer een spanning van 1V gedurende 1s een stroom van 1A door de keten drijft.
\subsubsection{Het elektrisch verbruik:}
Het elektrisch verbruik zoals je ze op de energiefactuur terugvindt, is eigenlijk hetzelfde als de verbruikte elektrische energie.
Echter wordt deze altijd in kWh uitgedrukt ipv in Ws.
\subsubsection{Elektrisch vermogen:}
Elektrisch vermogen (P) is de elektrische energie die per tijdseenheid wordt verricht (J/ s=W)
\begin{itemize}
    \item $P=W/t$ of $P=U*I$ dus $P=I^2*R$=$U^2/R$
\end{itemize}
%wat verder aanvullen van slides
\subsection{Soorten Spanning en stroom:}
\begin{itemize}
    \item Gelijkstroom-/spanning (DC)\begin{itemize}
        \item Dezelfde richting (polariteit)
        \item niet noodzakelijk constante waarde\begin{itemize}
            \item constant,varieren in tijd,periodiek herhalen
            \item vb. batterij
        \end{itemize}
    \end{itemize}
    \item Tijdsafhankelijke stroom-/spanning (AC)\begin{itemize}
        \item Waarde afhankelijk van de tijd\begin{itemize}
            \item vb. wisselstroom-/wisselende stroom
        \end{itemize}
    \end{itemize}
\end{itemize}
\subsubsection{Alternating Current (AC)}
Wisselstroom-/spanning
\begin{itemize}
    \item Periodische Tijdsafhankelijke grootheid
    \item Belangrijk kenmerk: rekenkundig gemiddelde=0\begin{itemize}
        \item Vorm speelt geen rol!
        \item Rekenkundig gemiddelde != 0? Gemengde gesupersoneerde stroom samengesteld uit gelijk- en wisselstroom
    \end{itemize}
    \item Piekwaarde (=amplitude)\begin{itemize}
        \item Positief als negatief
        \item Hoogste waarde (IRMS x $\sqrt{2}$)
    \item Ogenblikkelijke waarde\begin{itemize}
        \item Op een bepaald ogenblik 
        \item $\sqrt{2}$x IRMS x sin()
    \item RMS-waarde (=effectieve waarde)\begin{itemize}
        \item Gelijkstroomwaarde van het signaal die in dezelfde tijd en in dezelfde weerstand dezelfde hoeveelheid warmte ontwikkelt
        \item Root Mean Square:Wortel uit het gemiddelde van het kwadraat
    \end{itemize}
    \end{itemize}
    \end{itemize}
    \item Gemiddelde waarde\begin{itemize}
        \item Nooit over volledige periode bekijken!
        \item De constante gelijkstroom-/ of spanning die er zou moeten vloeien om in dezelfde tijd van een halve periode in dezelfde weerstand dezelfde hoeveelheid elektriciteit te verplaatsen als de beschouwde wisselstroom.
    \end{itemize}
\paragraph{Periode}
\begin{itemize}
    \item Tijdsduur in seconden tussen 2 opeenvolgende tijdstippen waarop de wisselgrootheid op dezelfde wijze door 0 gaat of minimale,maximale waarde bereikt (20ms=1 periode).
\paragraph{Frequentie}
\begin{itemize}
    \item Aantal perioden per seconde f=1/t (Hz)
\end{itemize}
\paragraph{Cirkelfrequentie}
\begin{itemize}
    \item Aantal radialen dat per seconde doorlopen wordt (1 periode = 360\textdegree{}=2$\pi$ radialen)
\end{itemize}
\end{itemize}
\end{itemize}
\paragraph{Soorten vermogen:}
\begin{itemize}
    \item \textbf{Actief vermogen:}\begin{itemize}
        \item vermogen dat over een volledige periode verbruikt wordt
        \item nuttig vermogen\begin{itemize}
            \item Warmte van elektrische verwarming
            \item licht van gloeilamp
        \end{itemize}
        \item  Gewenste output van elektrisch toestel
    \end{itemize}
\end{itemize}
\begin{itemize}
    \item \textbf{Reactief vermogen:}\begin{itemize}
        \item Vermogen wordt niet gebruikt over een periode, toch moet installatie stroom kunnen dragen.
        \item Nodig voor magnetisme in elekrtische machines, transformatoren, smoorspoelen,\dots
        \item  Extra belasting voor de installatie 
        \item Geen eenduidige formule (Budeanu, Fryze, Sharon)
    \end{itemize}
\end{itemize}
\begin{itemize}
    \item \textbf{Schijnbaar vermogen:}\begin{itemize}
        \item Product van RMS-waarde van de spanning en stroom
        \item Vermogen dat effectief dient geinstalleerd te worden\begin{itemize}
            \item Bestaat zowel uit actief als reactief vermogen.
        \end{itemize}
        %vermogen driehoek
    \end{itemize}
\end{itemize}
Bij particulier moeten we meestal actief vermogen betalen, bij bedrijven is dat meestal het Schijnbaar vermogen.
\paragraph{Belang van cos($\phi$) bij netverbruikers (230V AC)}
\begin{itemize}
    \item P=100W met cos($\phi$)=1 (dus $\phi$=0, of geen faseverschuiving)
    P=U*I*cos($\phi$)=U*I$\Rightarrow$ I=P/U=100W/230V=0.43A
    \item P=100W met cos($\phi$)=0.6 (dus $\phi$=53.13\textdegree)
    P=U*I*cos($\phi$)$\Rightarrow$ I=P/U*cos($\phi$)=0.72A
    \item Om het zelfde actief (of nuttig) vermogen te kunnen leveren moet er bijna 70\% meer stroom lopen=meer opwarmen van de kabels= dikkere kabels nodig!
    \item cos($\phi$) wordt ook wel de arbeidsfactor genoemd.
\end{itemize}
\paragraph{Lineaire belasting:}
\begin{itemize}
    \item Lineaire belasting onttrekken een \textbf{stroom die dezelfde vorm} heeft als de aangelegde spanning. Als de spanning vervormd is zal de stroom dit volgen!
    \item zowel resistief, capacitief als inductief of een combinatie van de 3.
\end{itemize}
\paragraph{Niet-lineaire belastingen:}
\begin{itemize}
    \item Niet-lineaire belastingen onttrekken een \textbf{een niet sinusoidale stroom} bij een sinusoidale spanning.
\end{itemize}
\subsection{3-fase systemen}
Drifasige systemen\begin{itemize}
    \item Grote vermogens- elektrische energie overbrengen met meerfasig spanningssystemen.
\end{itemize}
\subsubsection{Evenwichtig en symmetrisch}
\begin{itemize}
    \item \textbf{Evenwichtig} driefasig lineaire systemen:\begin{itemize}
        \item Spanningen en stromen hebben dezelfde amplitude en vorm IA=IB=IC
    \end{itemize}
    \item \textbf{symmetrisch} driefasig lineair systeem:\begin{itemize}
        \item Fasehoeken gelijk aan elkaar (120\textdegree{} faseverschuiving onderling)
    \end{itemize}
\end{itemize}
\subsubsection{Niet-symmetrisch en niet-Evenwichtig}
\begin{itemize}
    \item Driefasig systemen- lineaire last\begin{itemize}
        \item Symmetrisch en evenwichtig driefasig lineaire systemen: nulgeleiderstroom = 0
    \end{itemize}
    \item Niet- symmetrische en evenwichtig driefasig lineaire systemen:
    %afbeeldingen toevoegen
\end{itemize}
\subsubsection{1-fase samenvatting:}
\begin{enumerate}
    \item Schakelaar open:\begin{itemize}
        \item spanning 230V over Schakelaar
        \item Geen stroom $\rightarrow$ lamp brandt niet
    \end{itemize}
    \item Schakelaar uit:\begin{itemize}
        \item Spanning weg over schakelaar (0V)
        \item Spanning over lamp=230V
        \item Stroom door cicrcuit=lamp brandt
        \item Stroom door bruine draad=stroom door blauwe
        \item Spanning bruin=230V,spanning blauw=0V $\Rightarrow$ vermogen door bruin=230*I,door blauw=0W (geen spanning)
        \item energie van net naar lamp
    \end{itemize}
\end{enumerate}
\subsection{Hoe komt het bij ons thuis?}
voor grote vermogingstoepassingen in 3-fasen.
\begin{enumerate}
    \item \textbf{Hoogspanningsnet:} +25KV 3-fasig.
    \item[] $\downarrow$ 
    \item \textbf{Middenspanningsnet:} 10KV-25kV 3-fasig. (vaak ondergronds)
    \item[] $\downarrow$
    \item \textbf{Laagspanningsnet:} 400V 3-fasen
\end{enumerate}
\subsubsection{Middenspanning naar laagspanning:}
Dit is in ster.
Als je tussen 2 lijnen meet krijg je 400V, als je fasen meet zal je 230V hebben.
1 fase met de nullijn is 230V.
$U_{fase}=U_{lijn}/\sqrt{3}$
\subsubsection{Ster vs driehoek:}
\begin{itemize}
    \item Ster configuratie - Bronzijde\begin{itemize}
        \item Mogelijkheid tot nulgeleider (er is altijd gemeenschappelijk nulpunt)$\Rightarrow$ IN=I1+I2+I3
        \item symmetrisch evenwichtig systeem IN=0 $\Rightarrow$ Afhankelijk van lineaire of niet-lineaire systemen.
        \item Verband tussen fase- en lijnspanning?
        \item verband tussen fase- en lijnstroom?
    \end{itemize}
\end{itemize}
Formules:$Ul=\sqrt{3}$ x $Uf$ In deze formule toont het aan dat $Ul$ groter is dan uw fasespanning * $\sqrt3$ de vierkantswortel 3 toont aan dat we in 3 fasen bezig zijn
en $If=Il$.
\begin{itemize}
    \item Driehoek configuratie: Bronzijde\begin{itemize}
    \item Geen gemeenschappelijk nulpunt.
    \item Verband tussen fase- en lijnspanning?
    \item Verband tussen fase- en lijnstroom?
    \end{itemize}
\end{itemize}
Bij driehoek zal $Uf$=$Ul$ maar de stroom zal dus $Il=\sqrt{3}$ x $If$.
\subsection{Condensatoren en batterijen:}
We hoeven niet te weten wat Tau inhoud/ 5 Tau voor deze les.
\subsubsection{Wat zijn condensatoren?}
\begin{itemize}
    \item Elektrische component dat elektrische lading kan opslaan.
    \item 2 parallele platen gescheiden door een dielektricum (isolator met doorslagspanning)
    \item Als condensator lading opneemt, stijgt de spanning\begin{itemize}
        \item Vermogen om lading op te nemen= de capaciteit [C]=C/V=Farad [F]. $Coulomb= A*s$ hier is A uw stroom en s uw tijd in seconden.
    \item Voornamelijk gebruikt in elektronische filters en energieopslag (klein).
    \end{itemize}
\end{itemize}
\subsubsection{Condensator op DC:}
Alle stroom kan in het begin door de condensator omdat er geen weerstand is over de condensator met dat er geen elektrische energie opgeslagen is.
Als je de spanning van de bron afsluit als uw condensator vol is wordt uw condensator de bron tot uw condensator volledig is ontladen.
\subsubsection{Condensator op AC:}
In wisselspanning is uw condensator eerder een weerstand die constant op en ontlaad. Hier wordt de condensator voornamelijk gebruikt om de stroom wat voor te laten lopen op de spanning. \textbf{De stroom ijlt hier voor}. De interne weerstand kan je berekenen met De formule hiervan is $X_C=1/(2\pi*f*c)$ hierbij is f de frequency van uw netwerk (bij ons 50hz) en c de capaciteit van uw condensator in Farad.
\subsubsection{gebruik condensatoren:}
\begin{itemize}
    \item Onstoringgscondensator
    \item Tijdschakelaar
    \item Afvlakkingen
    \item verbetering arbeidsfactor
\end{itemize}
\paragraph{afvlakkingcondensator:}
Bij een voeding gebruik je na uw transfo een bruggelijkrichter om de spanning naar DC om te vormen. Hierna gebruik je een afvlakkingcondensator om er voor te zorgen dat uw spanning niet constant wisselt.
\subsubsection{Batterijen:}
\begin{itemize}
    \item Vaak door elkaar gebruikt met condensator.
    \item Levert energie dmv galvanische reactie (chemisch).
    \item Levert dus spanning die verminderd naarmate batterij ontlaadt.
    \item capaciteit is uitgedrukt in Ah (ampere-uur)$\Rightarrow$ 1Ah= batterij kan 1 ampere voor 1 uur leveren.
\end{itemize}
\paragraph{Schakelen van batterijen:}
\begin{itemize}
    \item Serieschakelen:\begin{itemize}
        \item 1 cel $\rightarrow$ 1 tot 4V 
        \item Hybride omvormer heeft +- 50V ingangsspanning nodig.
        \item Bij serie schakeling worden de spanningen opgeteld
        \item Ah blijft gelijk
    \end{itemize}
    \item Parallelschakelen:\begin{itemize}
        \item Capaciteiten optellen
        \item Spanning blijft gelijk
        \item Opletten! 'Zwakste' batterij wordt opgeladen door 'sterkste'.
        Hiervoor gebruiken we een Battery managment system (BMS) om te controleren/voorkomen dat dit gebeurdt.
    \end{itemize}
\end{itemize}
\subsection{Spoelen en magnetisme:}
\subsubsection{Wat is magnetisme:}
\begin{itemize}
    \item Eigenschap van stoffen om bepaalde legereingen aan te trekken:\begin{itemize}
        \item Natuurmagneten
        \item Kunstmagneten of permanentmagneten
        \item Elektromagneten
    \end{itemize}
    \item Bestaan altijd uit een noord- en zuidpool\begin{itemize}
        \item Gelijknamige polen stoten elkaar af (N-N en Z-Z).
        \item Ongelijknamige polen trekken elkaar aan (N en Z).
    \end{itemize}
\end{itemize}
\subsubsection{Het magnetisch veld en veldlijnen:}
\begin{itemize}
    \item Magnetisch veld (H) is een veld dat de ruimte doordringt en een magnetische kracht op magnetische dipolen uitoefent of op elektrische ladingen.
    \item Eigenschappen van veldlijnen:\begin{itemize}
        \item Gesloten baan!! van N naar Z 
        \item Veldlijnen kunnen elkaar niet snijden
        \item Veldsterkte (H) overal verschillend
        \item Sterkst aan de polen
    \end{itemize}
\end{itemize}
\subsubsection{Magnetische permeabiliteit:}
\begin{itemize}
    \item Best te vergelijken met stroom: magnetsiche veldlijnen volgen de weg met de minste magnetische weerstand (reluctantie <> weerstand)
\end{itemize}
Vaak zullen we voor elektronica die niet mag verstoord worden door magnetische golven een metalen ring rond zodat de golven door de ring gaan ipv door het elektronisch circuit.
\subsubsection{Magnetische inductie:}
\begin{itemize}
    \item Magnetische inductie (beinvloeding) is het magnetisch maken van een stuk staal door het bij een magneet te brengen.
Remanent magnetisme (rest magnetisme) is het magnetisme dat in een stuk staal overblijft nadat het in een magnetisch veld is gebracht en terug verwijderd is.
\end{itemize}
\subsubsection{elektromagnetisme:}
\begin{itemize}
    \item We kunnen een magnetisch veld opwekken dmv stroom
    \item De veldlijnen zijn in dit geval concentrische cirkels rondom de geleider\begin{itemize}
        \item Rechterhand regel
    \end{itemize}
    \item Verdwijnen als we de stroom terug stoppen.
\end{itemize}
\subsubsection{Het EM veld rond een spoel:}
\begin{itemize}
    \item Luchtspoel (geleider ombuigen tot een lus)
\end{itemize}
$\rightarrow$ EM veld wordt sterker in de lus!
\subsubsection{Elektromagneet:}
\begin{itemize}
    \item Elektromagneet = spoel met een zachtstalen kern, dit is een tijdelijke magneet $\rightarrow$ klein remanent magnetisme
\end{itemize}
\subsubsection{elektromagnetische schakelaar:}
\begin{itemize}
    \item Relais = elektromagneet + beweegbaar anker
\end{itemize}
\subsubsection{Stroomklem:}
Denk aan de CT's die we in smart building gebruiken die we over de kabel plaatsen. We kunnen de stroom meten zonder de kring te onderbreken door een magnetisch veld te meten die door de kabel opgewekt wordt. (hoe groter de stroom hoe sterker het magnetisch veld).
\subsubsection{Spoel op DC:}
\begin{itemize}
    \item $P=I^2*R_w$: omzetting van elektrische energie in warmte ten gevolge van de wikkelingsweerstand.
    \item $W=(1/2)*L*I^2$: Opgeslagen energie in het magnetisch veld
\end{itemize}

\subsubsection{Spoel op AC:}
Zorgt er voor dat onze stroom zal na ijlen op onze spanning. We kunnen condensatoren plaatsen om de stroom voor te laten ijlen. Dit is belangrijk zodat we een goeie cos($\Phi$) krijgen.
\section{Hoofdstuk 2:}
\subsection{energieproductie:}
\subsubsection{Welke soort centrales bestaan er?}
\begin{itemize}
    \item Kerncentrale\begin{itemize}
        \item 50\% komt van kerncentrales
    \end{itemize}
    \item Thermische centrale
    \item STEG centrale (Steam and gas)\begin{itemize}
        \item 30\% komt van stoom en gas.
    \end{itemize}
    \item Hydro centrale
    \item \dots
    \item decentrale productie:\begin{itemize}
        \item 15\% ongeveer
    \end{itemize}
\end{itemize}
Elektriciteitsverbruik blijft nog altijd hoog door de elektrificatie $\rightarrow$ warmte pompen, elektrische voertuigen,\dots 
\subsubsection{Belastingsfactor:}
gas heeft een relatief lage belastingsfactor omdat we ze makkelijk kunnen uitschakelen.
\subsubsection{Black-out:}
\begin{itemize}
    \item Volledige onderbreking van het vermogen in een bepaald leveringsgebied.
    \item Zonder waarschuwing en is van onbepaalde duur.
    \item Typisch veroorzaakt door een plots onevenwicht in belasting en productie.
\end{itemize}
\subsubsection{Rolling Black-out:}
%vul verder aan
\subsubsection{Brownout:}
%vul verder aan
\subsection{Centrale energieproductie:}
\begin{itemize}
    \item Vraag != aanbod (ogenblikkelijk)
    \item Eis = stabiel net:\begin{itemize}
        \item Primaire regeling\begin{itemize}
            \item Gevraagd vermogen gelijk aan geleverd vermogen
            \item Regeling output van centrales
            \item Activering na enkele seconden tot uiterlijk 30 seconden
            \item Tekorten of uitval stuk primaire regeling $\Rightarrow$ meer leveren door units.
            \item Frequentieregeling net
        \end{itemize}
        \item Secundaire regeling (Turbojet, Coo)\begin{itemize}
            \item Extra vermogen door onvoorziene uitval van primaire regeling of een groot tekort zodat frequentie 50hz bedraagt.
            \item 
        \end{itemize}
    \end{itemize}
\end{itemize}
\subsubsection{Kerncentrales:}
\begin{itemize}
    \item Verschillende types kerncentrales:\begin{itemize}
        \item PWR: Drukwaterreactor of hogedrukreactor (merendeel centrales)
        \item BWR: Kokendwaterreactor (Fukushima)
        \item RMBK: lichtwater-gekoelde grafiet gemodereerde reactor (Tsjernobyl)
    \end{itemize}
    \item Thorium als groene vervanger voor Uranium?\begin{itemize}
        \item % vul verder aan
    \end{itemize}
\end{itemize}
\paragraph{Werking:}
\begin{itemize}
    \item Gebaseerd op kernsplitsing (Splitsen van zware atomen zoals uranium)
    \item Ontstaan van warmte die water omzet in stoom onder hoge druk en temperatuur (thermische energie)
    \item De stoom kan een turbine aandrijven (mechanische energie)
    \item Alternator zal mechanische energie omzetten in elektrische energie.
\end{itemize}
\textbf{Kan is groot dat we een van de soorten Centrales als examen vraag krijgen!}
\begin{itemize}
    \item Voordelen:\begin{itemize}
        \item Continue elektriciteitsproductie (24/24)
        \item Niet afhankelijk van eindige fossiele brandstoffen
        \item 1 van de goedkoopste manieren om elektriciteit te produceren
        \item geen CO2 uitstoot
        \item Grote reserves Uranium, veel productie met weinig Uranium mogelijk
    \end{itemize}
    \item Nadelen:\begin{itemize}
        \item Niet flexibel (Trage opstart en niet snel regelbaar)
        \item radioactief afval 
        \item Redelijk laag rendement (30 a 36\%)
    \end{itemize}
\end{itemize}
\subsubsection{Thermische centrale}
Gevoed van fossiele grondstoffen zoals kolen,aardolie of gas\begin{itemize}
    \item Ook mogelijk met biomassa (CO2 neutraal).
\end{itemize}
Oudste centrale in het energielandschap van Belgie\begin{itemize}
    \item Basis van andere centrales
\end{itemize}
\paragraph{Werking:}
Rankine-Cyclus principe\begin{itemize}
    \item Verbranding van fossiele grondstoffen in de stoomketel
    \item Hete rookgassen
    \item %vul verder aan
\end{itemize}
\subsubsection{STEG centrale:}
\begin{itemize}
    \item[Werking]:Gasturbine bestaat uit compressor,verbradingskamer en turbine:\begin{itemize}
        \item Compressor zuigt gefilterde lucht aan, perst deze samen onder hoge druk en stuurt deze naar de verbrandingskamer.
        \item Aardgas toevoegen aan de gecomprimeerde lucht (gasmengsel) voor ontbranding mogelijk te maken.
        \item Bij verbrandingsproces komt er een gasstroom vrij onder hoge druk (>10 bar) en een hoge temperatuur (>1000C)
    \end{itemize}
\end{itemize}
% vul verder aan
\subsubsection{Hydro centrale}
\begin{itemize}
    \item[Werking:]\begin{itemize}
        \item IN belgie enkel Centrale van Coo
    \end{itemize}
\end{itemize}
\subsubsection{andere centrales:}
\subsection{Transport en distributie:}
Centrale opwekking bestaat uit verschillende units:\begin{itemize}
    \item Voeden gevraagd lastprofiel:\begin{itemize}
        \item Profielen van last worden zo goed als mogelijk nagebouwd door verschillende technieken
        \item Reserve inbouwen
    \end{itemize}
    \item Transporteren met zo weinig mogelijk verlies (driefasige AC)\begin{itemize}
        \item Joule verliezen (weerstand van de geleider)
        \item Spanningsval
    \end{itemize}
Skins-effecten $\rightarrow$ alles wordt in de kabels naar buiten geduwd (AC heeft een groter verlies dan DC)
\end{itemize}
De reden waarom we AC gebruiken is omdat we gemakkelijk van 400V AC naar 400kV AC kunnen gaan met een transformator (transfo heeft altijd al hoog rendement gehad) nu is rendement 99\% en vroeger was dat al 95\%. Bij DC was het moeilijker vroeger om dit omtevormen omdat het rendement veel lager was voor zo een generators bvb max 50\% nu is dat ook wel ongeveer 95\% rendement voor DC.
\begin{itemize}
    \item Nemo-link: Eerste langeafstands HVDC verbinding tussen Belgie en Verenigd-Koninkrijk.
\end{itemize}
\begin{itemize}
    \item Hoe dichter bij de klant, hoe lager het spanningsverschil\begin{itemize}
        \item Afhankelijk van het type klant (industrieel of particulier)
    \end{itemize}
\end{itemize}
\section{Hoofdstuk 3:}
\subsection{Leidingen:}
\subsubsection{inleiding:}
\begin{itemize}
    \item Elektrische verbinding tussen verschillende delen van de elektrische installatie.
    \item Kiezen van de juiste geleideres beschreven in AREI en gebaseerd op internationale norm. 
    \item Toelaatbare stroom in een leiding wordt bepaald door:\begin{itemize}
        \item Opwarming van de geleiders leidt tot een temperatuur van de isolatie die niet groter wordt dan deze die de isolatie gedurende onbepaalde tijd kan verdragen zonder haar eigenschappen in gevaar te brengen.
    \end{itemize}
\end{itemize}
\begin{itemize}
    \item Leidingen:\begin{itemize}
        \item Eisen keuze juiste geleider:\begin{itemize}
            \item Toelaatbare stroom >= Bedrijfsstroom (stroom verreist voor toepassing).
            \item Spanningsval onder normale omstandigheden verenigbaar met de bedrijfszekere werking van de toestellen.
            \item Elektrodynamische invloeden die kunnen voorkomen in geval van kortsluiting en andere mechanische invloeden bregen de veiligheid niet in gevaar.
            \item Waarde van de impedantie moet verenigbaar zijn met de werking van de beveiligingen.
        \end{itemize}
    \end{itemize}
\end{itemize}
\begin{itemize}
    \item Voorbeeld opbouw kabel:\begin{enumerate}
        \item Kern 
        \item Isolerend omhulsel 
        \item Vulling (om de doorsnede rond te maken)
        \item Isolerende mantel 
        \item Metalen mantel\begin{itemize}
            \item Staal:mechanische bescherming
            \item Lood:waterdichting
        \end{itemize}
        \item Isolerende mantel tegen corrosie en potentiaalverspreiding
    \end{enumerate}
\end{itemize}
\paragraph{Isolatie:}
\begin{itemize}
    \item Zowel de elektrische, thermische als fysische eigenschappen zijn van belang.
    \item Levensduur installatie wordt in hoofdzaak bepaald door kwaliteit isolatie.
    \item Beperkende factor voor stroombelastbaarheid kabel $\rightarrow$ Tmax.
\end{itemize}
\begin{itemize}
    \item Buisleidingen:\begin{itemize}
        \item 
    \end{itemize}
    \item StroomRails:\begin{itemize}
        \item 
    \end{itemize}
\end{itemize}
\paragraph{Codes \& classificaties:}
\begin{itemize}
    \item Internationale aanduiding vereist fabricage kabels in verschillende landen
    \item Code met 9 aanwijzingen:\begin{itemize}
        \item enkel de laatste 4 moeten wij kennen (6,7,8,9)\begin{itemize}
            \item 6: welk soort kabel heb je (S=soepel,R=rigide) is bvb of ze goed flexibel zijn
            \item 7: het aantal geleiders
            \item 8: Beschermingsgeleider
            \item 9: Geleiderdoorsnede (dikte van kabel). bvb 1,5mm ofzo
        \end{itemize}
    \end{itemize}
\end{itemize}
%belgische versie moeten we niet kennen voor examen
\subsubsection{Plaatsingswijze:}
Gewoon weten dat regelgeving voor is.
\subsubsection{Bedrijfstroom:}
Stroom die toepassing nodig heeft, Ib\begin{itemize}
    \item Doorlopend bedrijf:\begin{itemize}
        \item Grootste stroomsterkte die in normale omstandigheden in de stroombaan vloeit
    \end{itemize}
    \item Onderbroken bedrijf:\begin{itemize}
        \item Thermisch gelijkwaardige stgroom die bij doorlopend bedrijf de samenstellende delen van de stroombaan op dezelfde temperatuur zou brengen (RMS-stroom)
        RMS is om aan te tonen hoeveel DC stroom/spanning je moet hebben voor zelfde resultaat en staat voor root mean square ($\sqrt{2}$)
    \end{itemize}
    \item Om installatie te dimensioneren moet een beel kunnen gevormd worden van de totale bedrijfsstroom die zal vloeien in een installatie.\begin{itemize}
        \item Som van alle bedrijfsstromen zou leiden tot een zware overdimensionering.
        \item Rekening houden met:\begin{itemize}
            \item Gebruiksfactor Kfactor
            \item Gelijktijdigheidsfactor: Kgelijktijdig
            \item Uitbreidingsfactor: Kuitbreiding
        \end{itemize}
    \end{itemize}
\end{itemize}
\paragraph{Gebruiksfactor:}
gebruiksfactor Kgebruik:\begin{itemize}
    \item kgebruik=$\frac{gebruikt vermogen aangekoppeld toestel}{maximaal vermogen}$
    \item Kan varieren van 0,3 tot 1\begin{itemize}
        \item Vb. van een lamp wordt er verwacht dat hij altijd op zijn nominaal vermogen werkt.
    \end{itemize}
\end{itemize}
\paragraph{Gelijktijdigheidsfactor:}
Gelijktijdigheidsfactor Kgelijktijdig:\begin{itemize}
    \item 
\end{itemize}
\begin{itemize}
    \item Monofasige kringen:$I_b=(P_el/(U_f *cos(\Phi)))*kgebruik*kgelijktijdig*kuitbreiding$
    \item Driefasige kringen:$I_b=(P_el/(\sqrt{3}*U_f *cos(\Phi)))*kgebruik*kgelijktijdig*kuitbreiding$
\end{itemize}
\subsubsection{Thermische belasting van een leiding:}
%mss verder aanvullen
Er zal wss een oefening op het examen zijn van het kiezen van dikte kabel met uw stroom met specifieke factoren.
\begin{itemize}
    \item correctiefactor thermische weerstand van de grond 
    \item correctiefactor harmonische stroom.
\end{itemize}
\paragraph{voorbeeldoefeninge:}
\begin{itemize}
    \item Gegeven:\begin{itemize}
        \item Bedrijfstroom (Ib)=100A
        \item 3G koper (dus 3 geleiders)
        \item multipolaire kabel PVC
        \item Enkele laag op kabelladder met 5 andere kabels
        \item Omgevingstemperatuur 40 graden.
    \end{itemize}
    \item oefening:\begin{itemize}
        \item We zitten in factor E voor kabels en de rechter kolom. We weten dat we een Ib van 100A hebben dus moeten we een kabel van 25 $mm^2$ hebben. dus S=25 $mm^2$ 
        \item onder standaardcondities is dit genoeg maar de standaarcondities zijn niet voldoende!
        \item er zijn in totaal 6 kabels (5+1) bij enkele laag op de ladder heb je 0,79 dus correctiefactor 1 is 0,79 $\rightarrow$ k1. 
        \item Correctiefactor is de Omgevingstemperatuur van 40 graden. omdat we een PVC kabel hebben Bovengronds wordt onze tweede correctiefactor 0.87 $\rightarrow$ k2
        \item nu zal onze Ib'=$\frac{I_b}{k_1*k_2}$ $\Rightarrow$ 153A dus nu moeten we zeker een dikkere kabel nemen ondat onze Ib groter is.
        \item als we terug naar de tabel gaan zien we nu dat onze S=50$mm^2$
    \end{itemize}
\end{itemize}
Oppassen met piekstroom. stroom-tijd diagram verstaan.
\subsection{Investingsniveau en energiebesparing:}
Leiding met grote stroomdichtheid: $\rightarrow$ \textbf{Leiding kan veilig zijn uit thermisch standpunt}, maar tevens zeer energieverslindend door grote stroomdichtheid. Als je op een grensgeval zit bij een kabel is het beter om de dikkere kabel te nemen zodat je minder verlies hebt.
\section{Beveiligingen:}
\subsection{Algemeen:}
\begin{itemize}
    \item[] Beveiligen van leidingen: Het vermijden van te grote opwarming bij:\begin{itemize}
        \item Overstromen:\begin{itemize}
            \item Groter dan nominale stroom: maar die gedurende niet al te lange tijd toegelaten kunnen worden (vb. Opstarten inductiemotor).
            \item Overstroom te lang aanwezig: installatie moet afschakelen omdat levensduur van de leidingen sterk verminderd wordt.
        \end{itemize}
        \item Kortsluitstromen:\begin{itemize}
            \item Dit zijn heel grote stromen die snel moeten onderbroken worden. Ontoelaatbare warmte veroorzaakt in erg korte tijd.
        \end{itemize}
        \item Een leiding die een te hoge temperatuur heeft gehad moet vervangen worden:\begin{itemize}
            \item Gepaard met veel manuren = hoge kostprijs.
        \end{itemize}
    \end{itemize}
\end{itemize}
Elektrische leidingen en circuits moeten op een veilige manier in en uit bedrijf te nemen zijn, hiervoor bestaan verschillende soorten schakelaars:\begin{itemize}
    \item Scheiders:\begin{itemize}
        \item zichtbare scheiding van een stroombaan 
        \item Enkel openen zonder stroom, een nullaststroom van een transfo is reets 
    \end{itemize}
\end{itemize}
\begin{itemize}
    \item Magneto thermische automaten:\begin{itemize}
        \item kunnen zowel lage als Kortsluitstromen onderbreken
        \item Uitschakeldrempels zijn niet instelbaar
    \end{itemize}
    \item Vermogenschakkelaars:\begin{itemize}
        \item Kunnen zowel lage als kortsluitstromen onderbreken 
        \item Bevatten 1 of meerdere instelbare Uitschakeldrempels
    \end{itemize}
    \item Smeltveiligheden in smeltpatroonhouders:\begin{itemize}
        \item Kunnen overbelastingstromen en/of kortsluitstromen onderbreken, afhankelijk van het soort smeltveiligheid.
    \end{itemize}
\end{itemize}
Magnetische schakelaars worden meestal gebruikt voor kortsluitstromen.
\subsection{Differentieel schakelaar en plong/automaat}
\subsubsection{Differentieel schakelaar:} uw differentiaal schakelaar is om verlies te detecteren (voornamelijk om mensen te beschermen).werkt op 300mA en in vochtige ruimtes 30mA.
1 differentiaal is genoeg voor al uw stopcontacten etc in uw huis. enkel voor uw badkamer heb je een apparte differentiaal nodig.
\paragraph{opbouw en werking:}
\begin{itemize}
    \item Isolatiefout\begin{itemize}
        \item Fout tussen geleidend deel en aarde (foutstroom via ongewenst pad)
    \end{itemize}
    \item Stromen moeten gedetecteerd en uitgeschakeld worden
    \item Synoniemen als verliesstroomschakelaar of aardlekschakelaar
    \item\textbf{Opgelet:}\begin{itemize}
        \item Differentieelinrichting: Detecteert een lekstroom
        \item Differentieelschakelaar: Detecteert een lekstroom en onderbreekt
        \item Differentieelautomaat: Detecteert lekstroom, overbelasting, kortsluitingen en
        onderbreekt
    \end{itemize}
    \item Nettostroom door spoelen met m windingen gelijk aan 0\begin{itemize}
        \item IU= IN , er loopt geen flux in de ringkern
    \end{itemize}
    \item Stroom naar de aarde, flux wordt veroorzaakt
    \item Resulterende flux induceert spanning in een spoeltje dat gewikkeld is rond dezelfde ringkern\begin{itemize}
        \item Voedt de bekrachtigingsspoel van de elektromagneet
        \item Elektromagneet is in staat om ontgrendelingsmechanisme te activeren
    \end{itemize}
\end{itemize}
\paragraph{Kenmerken:}
\begin{itemize}
    \item Nominale stroom In\begin{itemize}
        \item Maximale stroom die de differentieel continue mag voeren
    \end{itemize}
    \item Afschakelstroom I$\Delta$n\begin{itemize}
        \item Is de waarde van de kleinste aardlekstroom waarbij de differentieel de stroombaan onderbreekt
        \item De vereiste gevoeligheid is afhankelijk van de aardingsweerstand van de massa's, de aardingsweerstand van de transfo en de omgevingstoestand
        \item 30, 100 en 300 mA zijn de meest voorkomende waarden bij residentiële woningen
    \end{itemize}
    \item Onderbrekingsvermogen Imax\begin{itemize}
        \item Heel wat lager dan bij zekeringen of automaten. Differentieelschakelaars worden meestal voorzien in combinatie met vermogensschakelaars die wel grote stromen kunnen onderbreken
    \end{itemize}
    \item Nominale spanning Un\begin{itemize}
        \item Hoogste nominale spanning van het net waarop de schakelaar mag gebruikt worden
    \end{itemize}
    \item Tijd-stroomkarakteristiek:\begin{itemize}
        \item Enige karakteristiek die gedefinieerd wordt
        \item Uitzetten van tijd waarin de differentieelschakelaar afschakelt in functie van de verliesstroom
        \item Vanaf de gevoeligheidsgrens geldt een bepaalde maximale onderbrekingstijd, vastgelegd in de norm IEC 60364.
        \item De differentieel kent ook een minimale reactietijd, typisch 40 ms.
        \item Bij tijdsvertragende differentieelschakelaars kan de tijd-stroomkarakteristiek aangepast worden
    \end{itemize}
\end{itemize}
Uitvoering residentieel:
\begin{itemize}
    \item 300mA voor algemene installaties
    \item 30mA voor alle vochtige ruimtes
    \item 100mA voor circuits voor vloerverwarming
\end{itemize}
\subsubsection{Automaat/plong:}
Is bedoeld om uw kabels en uw toestellen te beschermen. werkt op bvb 21A. Automaat kijkt of dat je stroom niet te hoog is.
\paragraph{Fuse/zekering:}
Oudere primitievere manier van automaat die doorsmelt als de stroom te hoog wordt. Dit wordt wel nog veel gebruikt bij elektronica. Je kan checken dat uw fuse doorgesmolten is door met een multimeter te checken voor weerstand, als het piept werkt uw fuse nog anders moet je hem vervangen.
\subparagraph{opbouw en werkingsprincipe:}
\begin{itemize}
    \item Beveiligingen tegen overstroom en kortsluitstroom:\begin{itemize}
        \item Gekalibreerd element (smeltdraad of -strip) onderbreekt de stroom bij overschrijding stroomwaarde ifv tijd
    \end{itemize}
    \item werkingsprincipe:\begin{itemize}
        \item zwakste schakel van de kring
        \item Te hoge stroom zal de temperatuur van de smeltdraad opwarmen tot het smeltpunt (>1000 graden)= smelttijd.
        \item Wanneer smeltlichaam in 2 delen is gesmolten, begint de boogtijd
    \end{itemize}
\end{itemize}
\textbf{Totale schakeltijd = smelttijd + boogtijd}
Zekeringen liefst niet al te hoge weerstand want een weerstand vermindert uw stroom dus zal je verlies hebben.
\subsubsection{Vermogenschakelaars:}
Automatisch werkende schakelaar die het elektrisch circuit \textbf{beveiligt} tegen schade door overstroom of kortsluiting.\begin{itemize}
    \item Moet niet vervangen worden na onderbreking
    \item Kan meerpolig onderbreken, terwijl de smeltveiligheid maar 1 fase kan onderbreken
    \item Duurder en bevatten onderdelen die onderhevig zijn aan slijtage (aantal afschakelingen is gelimiteerd).
\end{itemize}
Thermisch afschakelprincipe (=traag)\begin{itemize}
    \item Plooi bimetaal B dat in serie staat met de te beveiligen stroomkring\begin{itemize}
        \item Een bimetaal is een zoals de naam zegt een metaal die uit 2 metalen bestaat, dus de 2 metalen reageren anders op verandering in temp etc. 
    \end{itemize}
    \item Door de temperatuursverandering veroorzaakt door de stroomdoorgang plooit het bimetaal, waardoor een hefboom wordt ontgrendelt
    \item Veer duwt met grote kracht het contact open
    \item Gebruikt voor trage afschakeling (>0.1s) van kleine overstromen (bv. overbelasting)
\end{itemize}
Elektromagnetisch afschakelprincipe (=snel)\begin{itemize}
    \item Spoel in serie met de te beveiligen stroomkring
    \item Weekijzeren kern is beweegelijk en zit slechts gedeeltelijk in de spoel 
    \item Hoe sterker de stroom in het circuit, hoe meer de kern in de spoel wordt gezogen en het staafje S, A omhoog duwt 
    \item Stroomkring kan onderbroken worden,wordt gebruikt bij grote overstroom:\begin{itemize}
        \item aantrekking is op reluctantie gebaseerd en vrijwel kwadratisch met de stroom
        \item Fysische tijd nodig om te onderbreken
    \end{itemize}
\end{itemize}
Kenmerken:\begin{itemize}
    \item Nominale stroom [A]\begin{itemize}
        \item Stroom die de vermogenschakelaars oneindig lang kan voeren
    \end{itemize}
    \item Nominale spanning Ue [V]\begin{itemize}
        \item Is de hoogste nominale spanning van het net waarop de schakelaar mag gebruikt worden
    \end{itemize}
    \item Isolatiespanning Ui [V]\begin{itemize}
        \item Is de spanning die de schakelaar gedurende lange tijd kan verdragen 
    \end{itemize}
    \item Stootspanning Uimp [V]\begin{itemize}
        \item Is de spanning die schakelaar gedurende korte tijd kan verdragen (vb. Blikseminslag)
    \end{itemize}
    \item I\^2 t-waarde\begin{itemize}
        \item Maat voor de doorlaatenergie die stroomafwaarts gelegen toestellen moeten kunnen doorstaan, alvorens de vermogenschakkelaar het circuit onderbreekt (we moeten zorgen dat deze waarde niet te hoog is want eenmaal als de schakelaar of fuse etc werkt kan dit nog altijd schade aan uw kabels doen).
    \end{itemize}
    \item Conventionele niet-aanspreekstroom Inf [A]\begin{itemize}
        \item Stroom die die vermogenschakelaar gedurende bepaalde tijd kan weerstaan
    \end{itemize}
    \item Conventionele aanspreekstroom If [A]\begin{itemize}
        \item Stroom die die vermogenschakelaar binnen bepaalde tijd moet weerstaan
    \end{itemize}
    \item Magnetische uitschakelstroom Im [A]\begin{itemize}
        \item Stroom waarbij de automaat elektromagnetisch uitschakelt. Meestal uitgedrukt als een veelvoud van In.
        \item Stromen groter dan dit worden beschouwd als kortsluitstroom en worden zeer snel afgeschakeld (10ms)
    \end{itemize}
    \item Ultiem onderbrekinsvermogen [kA]\begin{itemize}
        \item Grootste kortsluitstroom die de vermogenschakelaar minstens 1 keer veilig kan onderbreken (kan zijn dat hij na die ene keer kapot is).
    \end{itemize}
    \item Bedrijfsonderbrekingsvermogen Ics\begin{itemize}
        \item moet meermaals na elkaar veilig onderbroken kunnen worden
        \item Voor nieuwe toestellen geldt meer en meer dat het bedrijfsonderbrekingsvermogen=ultiem onderbrekingsvermogen
    \end{itemize}
    \item Korteduurstroom Icw\begin{itemize}
        \item Stroom die kan gevoerd worden gedurende bepaalde tijd zonder dat de goede werking van het toestel verstoord wordt.
    \end{itemize}
    \item Normering\begin{itemize}
        \item Belangrijk verschil tussen residentiele en industriele norm:\begin{itemize}
            \item IEC 60898-1 legt bepalingen vast voor overstroombeveiliging in huishoudelijke of aanverwante installaties zoals tertiare gebouwen
            \item IEC 60947-2 legt bepalingen vast voor een industriële installatie.
        \end{itemize}
    \end{itemize}
    \item Tijd-stroomkarakteristiek (Thermisch gebied)\begin{itemize}
        \item Onderbrekingsduur in functie van de stroom
        \item Deel van het thermisch relais en deel van het elektromagnetische afschakeling
        \item Vanaf een stroom hoger dan de magnetische drempel Im schakelt het toestel magnetisch uit
        \item Eveneens bepaalde onzekerheid op de uitschakelcurve (Inf vs. If)
    \end{itemize}
    \item Tijd-stroomkarakteristiek (magnetisch gebied)\begin{itemize}
        \item Residentiële automaten (IEC 60898)\begin{itemize}
            \item Verschillende types, naargelang de magnetische uitschakeldrempel
            \item Thermisch gedrag is hetzelfde voor zowel curve B, C of D
            \item Residentiële kringen worden meestal beveiligd met een type C automaat
            \item D wordt meestal in de industrie gebruikt voor zware machines apparaten B wordt amper gebruikt.
        \end{itemize}
        \item Industriële automaten (IEC60947-2)\begin{itemize}
            \item Marge voor magnetische uitschakeling wordt gedefinieerd als Im ± 20\%
            \item Magnetische uitschakelstroom varieert naargelang het type van de automaat
        \end{itemize}
    \end{itemize}
    \item Doorlaatkarakteristiek\begin{itemize}
        \item Geeft de stroombegrenzende werking weer van vermogenschakelaars.
        \item Optredende kortsluitstroom beperken (kleiner dan werkelijk zou vloeien)
        \item IEC 60898-1 definieert de maximale doorgelaten energie I²t per energiebeperkingsklasse\begin{itemize}
            \item Waarden vermenigvuldigen met 120\% voor vermogenschakelaars van 40A en groter
        \end{itemize}
        \item Norm stelt dat in een huishoudelijke installatie alle vermogensschakelaars van klasse 3 moeten zijn.
    \end{itemize}
    \item Energiekarakteristiek\begin{itemize}
        \item Ook wel I²t karakteristiek genaamd\begin{itemize}
            \item Voor overbelasting kent deze eerst een dalende kromme, veroorzaakt door het thermisch relais
            \item Stroom groter dan de magnetische drempel Im daalt de doorgelaten energie aanzienlijk
            \item Nadien stijgt de stroom terug omdat er een minimale tijd nodig is om de contacten te openen, onafhankelijk van de grootte van de stroom
            $\Rightarrow$Bij niet-stroombegrenzende schakelaars neemt de energie toe met het kwadraat van de stroom
            $\Rightarrow$Bij stroombegrenzende schakelaars is dit niet zo, de maximale doorgelaten stroom warmteimpuls wordt door constructeurs weergegeven als functie van de kortsluitstroom en de netspanning
        \end{itemize}
    \end{itemize}
\end{itemize}
\subsubsection{Isolatiewachters:}
\begin{itemize}
    \item IT-net: Hier hebben we bij de grond van de transfo een grote weerstand (impedantie) die de lekstroom beperkt die binnen komt (zorgt ervoor dat uw differentieel niet direct uitvalt) als er toevallig nog een toestel is die ook een lekstroom heeft zal uw lekstroom gaan waar het minste weerstand zitten dus dan zal uw differentieel wel uitvallan (Dit wordt meestal gebruikt in ziekenhuizen ofzo).
    \item TN-net: hierbij kan het zijn dat je PEN kabel hebt die dan de lek stroom van uw machines opneemt en via je nullijn terug stuurt naar uw transfo. (dit komt meer voor bij industrie)
    \item TT-net: als er lekstroom is zal het naar de aarding gaan en dan zal de stroom van ons residentiele net via de aarding terug gaan door de aarding van de van de transfo (groene kabine) terug in uw ster systeem. (dit komt meestal voor bij residentiele gebouwen) lekstroom zal hier redelijk groot zijn.
\end{itemize}
\begin{itemize}
    \item Isolatiefout in een IT-net\begin{itemize}
        \item IT-netten worden slechts erg lokaal aangewend op die plaatsen waar het eventueel wegvallen van de spanning niet tolereerbaar is.\begin{itemize}
            \item Bij een eerste fout is er geen mogelijkheid tot vormen van een foutlus -transformator geïsoleerd of geaard is over een impedantie die voldoende groot is.
            \item Opgelet, bij 2de fout ontstaat wel een grote foutstroom waardoor de installatie uitgeschakeld zal moeten worden. Dit gebeurt via, automaten of zekeringen al dan niet in combinatie met differentieelschakelaars.
        \end{itemize}
        \item Isolatiewachter waakt over het net en geeft een melding bij een 1ste fout. Zo is er de tijd om de fout te vinden en continuïteit te garanderen.
    \end{itemize}
\end{itemize}
\subsubsection{Lastscheiders:}
Wordt vooral gebruikt om er voor te zorgen dat specifieke schakelaars niet te warm worden als ze aan/af gezet moeten worden.
\subsection{Beveiliging tegen overstroom:}
Beveiliging van leidingen tegen overbelasting\begin{itemize}
    \item AREI Onderafdeling 4.4.3.2 - 3 voorwaarden waaraan gelijktijdig moet voldaan worden\begin{itemize}
        \item Ib <= In <= Iz
        \item nf <= 1,15 x Iz\begin{itemize}
            \item Beveiliging kan niet 100\% zeker reageren op een bepaalde overstroom binnen een zekere tijd, vandaar de begrippen Inf en If
            \item If <= Iz (totale beveiliging, veel te streng)\begin{itemize}
                \item Elke overbelasting van de leiding zou onmogelijk zijn
            \end{itemize}
            \item Inf < Iz (minder streng)
            \item Betekent dat de leiding niet permanent mag overbelast worden, toch te streng, mits 10 tot 20\% van de tijd de leidingen onderbenut
            \item Inf =1,15 Iz (slechtste geval)\begin{itemize}
                \item Temperatuurstijging kan oplopen tot 32\% boven de toegestane opwarming, en dit voor onbepaalde duur
            \end{itemize}
            \item If <= 1,45 x Iz\begin{itemize}
                \item Worst case zal de leiding gedurende de conventionele tijd 45\% overbelast worden.
            \end{itemize}
        \end{itemize}
    \end{itemize}
\end{itemize}
Beveiliging van leidingen tegen kortsluitstroom\begin{itemize}
\item Toestel moet voldoen aan 2 voorwaarden:\begin{itemize}
    \item Icu (ultiem onderbrekingsvermogen) >= Ik,max (maximale kortsluitstroom)
    \item $Ik^2t <=k^2Sl^2$\begin{itemize}
        \item Afschakeling vereist vooraleer de maximale thermische belasting van de leiding bereikt is.\begin{itemize}
            \item Sl = doorsnede geleider
            \item Ik = kortsluitstroom
            \item k= constante die afhangt van het type leiding
        \end{itemize}
    \end{itemize}
\end{itemize}
\end{itemize}
Beveiliging met zekering: overbelasting:\begin{itemize}
    \item Moet voldoen aan de 3 eerder besproken voorwaarden
    \item Keuze door karakteristiek van de leiding uit te zetten ten opzichte van de eigenschappen zekering\begin{itemize}
        \item Zekering zwarte curve, leiding rode curve
        \item Smeltkarakteristiek zekering iets steiler dan de leidingskarakteristiek en deze snijden elkaar bij Ia
    \end{itemize}
    \item rekenvoorbeeld:\begin{itemize}
        \item gG zekering van 25A (=In) wordt gebruikt om een leiding te beveiligen met een doorsnede van 6$mm^2$ en toelaatbare stroom van 30A\begin{itemize}
            \item Inf = 1.4 x In = 35A (k-waarde uit tabel)
            \item If = 1.75 x In = 43,75A (k-waarde uit tabel)
            \item[]Inf (35A) <= 1.15 x Iz (34,5A)
            \item[]If (43,75) <= 1.45 x Iz (43,5A)
            \item[] We zullen een grotere zekering moeten nemen.
        \end{itemize}
    \end{itemize}
\end{itemize}
Beveiligen met zekeringen: Kortsluiting\begin{itemize}
    \item Om een leiding te beveiligen tegen kortsluiting, moet voldaan zijn aan de 2 eerder besproken voorwaarden.\begin{itemize}
        \item Het maximaal onderbrekingsvermogen van de zekering moet groter zijn dan het maximaal kortsluitvermogen op de plaats van de beveiliging.\begin{itemize}
            \item (Icu >= Ik,max)
        \end{itemize}
        \item De doorgelaten energie zekering moet lager zijn dan de energie om de leiding op een ontoelaatbaar hoge temperatuur te brengen.\begin{itemize}
            \item ($Ik^2t =< k^2Sl^2$)
        \end{itemize}
        \item De kleinste kortsluitstroom Ik,min is de gevaarlijkste omdat zijn onderbreking het langst duurt.\begin{itemize}
            \item Om te kunnen voldoen aan de 2de voorwaarde, moet bovendien de tijd waarin de kortsluiting vloeit, kleiner zijn dan 5 seconden.
        \end{itemize}
    \end{itemize} 
\end{itemize}
\subsection{Coordinatie:}
Gedefinieerd als het correct gebruik van 2 of meer schakel- of controle inrichtingen in serie of parallel, teneinde de veilige en functionele werking van de installatie te
verzekeren.\textbf{We moeten vooral kunnen uitleggen wat dit is verder moeten we hier niet echt iets van weten.}
\subsection{Selectiviteit:}
\begin{itemize}
    \item Selectiviteit of discriminatie is een coördinatie van de werkingskarakteristieken van 2 of meer beveiligingstoestellen.
    \item Beveiliging die zich onmiddellijk stroomopwaarts van de fout bevindt, zal deze fout onderbreken, terwijl de overige toestellen
    ingeschakeld blijven.
\end{itemize}
2 Soorten selectiviteit:\begin{itemize}
    \item Totale selectiviteit:\begin{itemize}
        \item Wanneer het stroomafwaarts beveiligingstoestel (OCPD2) reageert op gelijk welke overstroom tot aan het onderbrekingsvermogen, het stroomopwaartse (OCPD1) toestel niet reageert
    \end{itemize}
    \item Partiële selectiviteit: \begin{itemize}
        \item Ontstaat wanneer het stroomafwaarts (OCPD2) beveiligingstoestel reageert op gelijk welke overstroom tot een bepaalde waarde vastgelegd door de fabrikant, terwijl de stroomopwaartse (OCPD1) toestel niet reageert. Bij het overschrijden van de vastgelegde waarde is het niet zeker welk toestel eerst reageert (OCPD 1 of 2)
    \end{itemize}
    \item Selectiviteit is belangrijk omdat het gepaard gaat met grote bedrijfscontinuïteit
\end{itemize}
\section{Verlichting (week 5):}
\subsection{Inleiding en begrippen:}
In industrie nog altijd veel verouderde, niet efficiente technologie.
\underline{Waarom investeren in LED?}
\begin{itemize}
    \item efficient licht (Lm/W)
    \item Eenvoudig dimbaar
    \item Regelbare kleur/kleurtemperatuur
    \item Lange levensduur
    \item Goede trilling bestendigheid
    \item klein en lichtkan eventueel werken op laagspanning
    \item met hoge prijzen snel terug verdiend
\end{itemize}
\subsubsection{Straling:}
Is niets meer dan het overbrengen van energie door middel van elektromagnetische golven (of deeltjes). Bepaald door\begin{itemize}
    \item Frequentie
    \item Amplitude
\end{itemize}
\paragraph{Stralingsflux of stralingsvermogen:}
Hoeveelheid energie per tijdseenheid door een oppervlak [W]\begin{itemize}
    \item != vermogen van de lamp!
    \item Straling van de lamp niet monochromatisch gedefineerd\begin{itemize}
        \item Niet discreet (continues): Kan ook infrarood en ultraviolet uitstralen.
        \item discreet: Enkel de energie die nuttig is voor ons (energie/licht die wij kunnen zien).
    \end{itemize}
    \item Zegt ons enkel iets over de energie in de golf, ons oog kan dit niet waarnemen.
\end{itemize}
\paragraph{Daarom invoeren lichtstroom:}\begin{itemize}
    \item Grootheid die het voor het menselijk oog waarneembare vermogen aan uitgestraald licht aangeeft.
    \item Dus maat voor helderheid die waarneembaar is voor het oog
    \item Lumen (lm)
\end{itemize}
Stralingsrendement:\begin{itemize}
    \item Verhouding tussen uitgestraalde vermogen als EM straling en het elektrische vermogen van de bron.
    \item Niet al het vermogen wordt omgezet in EM straling bv door warmte
\end{itemize}
Visueel rendement:\begin{itemize}
    \item Verhouding tussen lichtstroom en het stralingsvermogen.
    \item Infrarood bron = hoog stralingsvermogen, lage (0) lichtstroom
\end{itemize}
Specifieke lichtstroom:\begin{itemize}
    \item Verhouding tussen lichtstroom en elektrisch vermogen.
\end{itemize}
\paragraph{Verlichtingssterkte (van lumen naar lux):}\begin{itemize}
    \item Lux = lumen per vierkante meter oppervlak
    \item Te meten met luxmeter
\end{itemize}
Bij een licht blijft uw lumen hetzelfde maar uw Lux kan varieren afhankelijk van afstand van licht etc. Dus voor normen van licht etc in bvb een ziekenhuis kijken ze eerder naar uw Lux.
\paragraph{Lichtsterkte:}
\begin{itemize}
    \item Meet hoe helder een lichtbron in een bepaalde richting schijnt
    \item Candela
\end{itemize}
\subsection{CCT en CRI:}
\subsubsection{CCT:}\begin{itemize}
    \item Correlated color temperature:\begin{itemize}
        \item Uitgedrukt in Kelvin (lichaam dat opwarmt zal licht uitzenden)
        \item CCT hoog= koud licht
        \item CCT laag= warm licht
        \item Eigen aan constructie
        \item CCT ledstrips? het verkleuren werkt door meerdere verschillende soorten led lichtjes van bvb 2700K,3500K,5500K naast elkaar te zetten en de ene meer intensiteit geven of wat minder laten schijnen voor warmere of meer heler licht te krijgen.
    \end{itemize}
\end{itemize}
\subsubsection{CRI:}\begin{itemize}
    \item Color Rendering Index of kleurweergave index
    \item Van lumen naar CRI
    \item Vergelijken van kleurweergave tov referentiebron (CRI 100=halogeenlamp)
    \item Hoe goed kan de lamp de echte kleuren weergeven 
    \item Bv. straatlamp dikwijls CRI van +- 70\%, spuitcabines +90\%
\end{itemize}
\subsubsection{Toepassingen:}
CCT en CRI niet enkel voor sfeer en gezelligheid\begin{itemize}
    \item Detecteren van rottend fruit door machines $\rightarrow$ verlichting enorm belangrijk
    \item Perovskite LEDs $\rightarrow$ rot fruit zichtbaar voor het zichtbaar is voo menselijk oog.
    \item Medische wereld, kleuren moeten correct zijn om correcte conclusies te kunnen trekken.
    \item Kunst en musea
    \item Koplampen/interieur verlichting voor veiligheid (en uiteraard ook esthetiek)
\end{itemize}
\subsection{Lenzen en reflectoren}
\subsection{Verlichting in de praktijk:}
\begin{itemize}
    \item Verlichtingssterkte (E)= afhankelijk van het taakgebied en visuele eisen.
    \item 3 zones:\begin{itemize}
        \item taakgebied
        \item Directe omgeving 
        \item Achtergond
    \end{itemize}
    \item Gemiddelde verlichtingssterkte mag nooit lager zijn dan wettelijke norm
    \item onderhoudsfactor
    \item Schaalverdeling: [20-30-50-75-150-200-300-500-750-1000-1500-2000-3000-5000 Lux]
\end{itemize}
Gelijkmatige lichtverdeling = uniformiteit = Emin/Egem\begin{itemize}
    \item Minimaal tussen 0,4 en 0,7 in taakgebied 
    \item Directe omgeving 0,4
    \item Achtergrond 0,1
\end{itemize}
Reflectiewaarde:\begin{itemize}
    \item Plafond (30 lux), muren (50 lux), meubels,...
\end{itemize}
UGR = verblindingsfactor = bepaald gebied grotere helderheid\begin{itemize}
    \item Rechtstreeks of onrechtstreeks
    \item Voor elke type taak omschreven
    \item Schaalverdeling loopt van 10 tot 28, hoe lager de waarde, hoe minder de verblinding\begin{itemize}
        \item UGR < 15: geen merkbare verblinding
        \item UGR > 22: storend
        \item UGR > 28: ondraaglijk, belemmerend
    \end{itemize}
\end{itemize}
Normering EN 12464-1:\begin{itemize}
    \item E = Verlichtingssterkte
    \item U = gelijkmatige lichtverdeling
    \item UGR = verblindingsgraad
    \item Ra = kleurweergave-index
\end{itemize}
\subsection{Investeringen:}\begin{itemize}
    \item Veel oude technologie in de industrie (75\%)
    \item Comfort parameters:\begin{itemize}
        \item E, U, UGR, Ra, onderhoud, ..
    \end{itemize}
    \item Relamping, retrofitting en relighting
    \item Hoe aanpakken?\begin{itemize}
        \item Energieaudit + metingen:\begin{itemize}
            \item Elektrische en comfort
        \end{itemize}
        \item Wensen klant:\begin{itemize}
            \item Ruimtes, lichtinval, werkingsgebieden, aard van de ruimte, wetten, budget,... $\rightarrow$ Lichtplan (Dialux)
        \end{itemize}
    \item Monitoringsystemen
    \item Slimme sturingen:\begin{itemize}
        \item Aanwezigheidsdetectie, daglichttoetreding, dimmen, ..
    \end{itemize}
    \end{itemize}
\end{itemize}
\begin{itemize}
    \item Relamping\begin{itemize}
        \item Minst ingrijpende manier
        \item Enkel lampen vervangen door nieuwer model
        \item Vb gloeilampen vervangen door ... ?
    \end{itemize}
    \item Retrofitting\begin{itemize}
        \item Lampen vervangen door ander model
        \item Fitting blijft behouden
    \end{itemize}
    \item Relighten\begin{itemize}
        \item Meest ingrijpende manier, grootste besparing mogelijk
        \item Lampen + armatuur vervangen
        \item Duurder investering en installatiekost
    \end{itemize}
\end{itemize}
\subsubsection{Financiele analyse:}\begin{itemize}
    \item Elke situatie is anders $\rightarrow$ Algemeen plan 
    \item Armatuur?
    \item Infrastructurele aanpassingen?
    \item Intern uitvoeren = tijdskost
    \item Extern uitvoeren = verschillende offertes aanvragen
    \item Onderhoudskost
    \item Comfortparameters
\end{itemize}
\paragraph{Opdracht:}
\textbf{Gegeven:}\begin{itemize}
    \item 7 TL lampen die 10 uur per dag aanliggen. twijfelt om TL lampen te vervangen met LED. het is 30 cent per KWH
\end{itemize}
\textbf{Gevraagd:}\begin{itemize}
    \item Verminderingsfactor oude TL-lampen = 0.85
    \item Wat moet de eigenaar doen? Oude lampen behouden, nieuwe TL lampen of LED?
\end{itemize}
\textbf{Uitwerking:}\begin{itemize}
    \item Oude lampen behouden?\begin{itemize}
        \item 7x30W=210W
        \item 10x210=2100Wh=2.1kWh per dag 
        \item 2.1kwH*365=766.5kWh per jaar 
        \item €0.3*766.5=€229.95
    \end{itemize}
    \item Nieuwe TL lampen:\begin{itemize}
        \item ook €229.95 per jaar
        \item maar dan ook nog de 150 en 14 voor de TL lampen.
    \end{itemize}
    \item Led verlichting:\begin{itemize}
        \item 7x€7=€49
        \item 7x14W=98W
        \item 98W*10=980wh per dag
        \item 980*365=357.7kWh 
        \item €0.3*357.7= €107.3 per jaar
        \item 49+150=€199 bij plaatsen
    \end{itemize}
\end{itemize}
na 2 jaar zijn de leds goedkoper dan de oude TL lampen en achter het eerste jaar zijn de leds als goedkoper dan de nieuwe TL lampen.
\subsection{Types:}
\subsubsection{Gloeilamp:}
\begin{itemize}
    \item Elektrische weerstand
    \item Experiment Edison 
    \item Meer verwarmingselement dan lichtelement
    \item Wolframdraad ipv katoendraad
    \item Inert gas (N of Kr)
    \item 95\% warmte
    \item Meestal technische gegevens op lampvoet\begin{itemize}
        \item Spanning, vermogen en evt lichtstroom (Lumen)
    \end{itemize}
    \item Goedkoop
    \item Goede kleurweergave
    \item Korte levensduur (+- 1000u)
    \item Mag niet meer gemaakt worden (in Europa)
\end{itemize}
Dimmen:\begin{itemize}
    \item Regelbare weerstand (vermogensverlies = ?)
    \item Regelbare transformator\begin{itemize}
        \item Veel minder warmteverlies
        \item Nadelen?\begin{itemize}
            \item Duur voor een lampje aan te sturen.
            \item Rendement zakt veel als je voor 1 lamp weinig spanning nodig hebt.
            \item lawaai (trillingen transfo)
        \end{itemize}
    \end{itemize}
    \item Chopper:\begin{itemize}
        \item Faseaansnijding\begin{itemize}
            \item Met welke vermogenscomponent kunnen we dit doen?
            \item Vermogen?\begin{itemize}
                \item veel minder vermogen verlies omdat er delen zijn waar je geen vermogen verlies hebt.
            \end{itemize}
        \end{itemize}
    \item Kan PWM?
    \end{itemize}
\end{itemize}
\subsubsection{Halogeenlamp:}\begin{itemize}
    \item soort gloeilamp:\begin{itemize}
        \item Halogeengas (l of Br)\begin{itemize}
            \item 25 atm (2500 kPa)
        \end{itemize}
    \end{itemize}
    \item 3000K tov 2700K bij gloeilamp 
    \item langere levensduur (+-2000u)
    \item Dimmen zelfde als gloeilamp\begin{itemize}
        \item om de 4u volle spanning (Anders kapot) $\rightarrow$ Halogeencyclus
    \end{itemize}
\end{itemize}
\subsubsection{TL lamp:}
\begin{itemize}
    \item Gasontladingslamp\begin{itemize}
        \item Ioniseren van gas $\rightarrow$ beter stroom geleiden
    \end{itemize}
    \item Elektronica om buis te ontsteken\begin{itemize}
        \item Starter
        \item Ballast
    \end{itemize}
    Gewone TL lamp:\begin{itemize}
        \item Ballast of voorschakelapparaat\begin{itemize}
            \item Spoel
        \end{itemize}
        \item Starter\begin{itemize}
            \item Lampje 
            \item Bimetaal 
            \item Condensator 
        \end{itemize}
        \item Gloeidraad 
        \item Fluorescentiepoeders
    \end{itemize}
\end{itemize}
Werking conventionele TL lamp:\begin{itemize}
    \item Ballast:\begin{itemize}
        \item Hoge spanningspiek 
        \item Beperken stroom
    \end{itemize}
    \item Starter:\begin{itemize}
        \item Aansluiten netspanning:\begin{itemize}
            \item $\rightarrow$ Lamp 
            \item $\rightarrow$ Bimetaal trekt krom (=sluiten)
            \item $\rightarrow$ Grote stroom:\begin{itemize}
                \item gloeidraden worden warm 
                \item Lampje gaat uit
            \end{itemize}
            \item $\rightarrow$Bimetaal opent
        \end{itemize}
    \end{itemize}
    \item Spoel verzet zich tegen stroomverandering\begin{itemize}
        \item Spanningspiek met reeds opgewarmde gloeidraden $\rightarrow$ Tl lamp ontsteekt
    \end{itemize}
    \item Lage spanning om in stand te houden
    \item Spanningsval over voorschakelapparaat\begin{itemize}
        \item stroombegrenzer
    \end{itemize}
    \item Spanning niet voldoende om neonlamp te ontsteken
\end{itemize}
\paragraph{Voordelen en problemen:}
\begin{itemize}
    \item Veel beter rendement en veel langere levensduur (10 tot 60.000u)
    \item Flikker (na opstart):\begin{itemize}
        \item Smoorspoel kan oververhitten of starter uitbranden
        \item Oplossingsmogelijkheden:\begin{itemize}
            \item Controleer spanning
            \item Probeer buis die nog werkt:\begin{itemize}
                \item Opgelost? $\rightarrow$ buis vervangen
                \item Zelfde probleem?
            \end{itemize}
            \item Starter (condensator) uitmeten
            \item Als laatste smoorspoel
        \end{itemize}
    \end{itemize}
    \item Buis:\begin{itemize}
        \item zwart 
        \item elektrodes controleren 
        \item Gas uit buis ontsnapt
    \end{itemize}
    \item Energiezuinig 
    \item Niet dimbaar 
    \item Hoge lichtopbrengst
    \item Opvatting: meer energie nodig om te starten dan te branden 
    \item Opvatting: Aanschakelen verslijt de lamp
\end{itemize}
\subsubsection{Spaarlamp:}
\begin{itemize}
    \item Combinatie TL-lamp en gloeilamp\begin{itemize}
        \item CFL
    \end{itemize}
    \item opgevouwen tl-lamp
    \item E-ftting 
    \item stuk? $\rightarrow$ volledig vervangen 
    \item tot 15000u werking 
    \item 4x lichtopbrengst gloeilamp
\end{itemize}
\subsubsection{LED:}
\begin{itemize}
    \item Helder licht, beperkt verbruik, lange levensduur (jaren ipv maanden) 
    \item Beperkte warmteontwikkeling
    \item Schok- en waterbestendig
    \item Fitting
    \item Verschillende golflengtes (IR)
    \item Uf = 1.5V tot 3V
    \item Nominale stroom +-20mA
    \item Datasheet!
\end{itemize}
\begin{itemize}
    \item Dimmen:\begin{itemize}
        \item Regelbare weerstand of spanningsdeler
        \item Spanning verlagen met bv transfo\begin{itemize}
            \item Let op van drempelspanning
            \item Efficiënter maar duurder
        \end{itemize}
        \item PWM
    \end{itemize}
\end{itemize}
\paragraph{Hoe bepalen we belastingskarakteristiek:}\begin{itemize}
    \item I=(Ubron-Uled)/R
    \item eerste punt is bij Uled 0 dus dan is het Ubron/R 
    \item tweede is bij Uled=Ubron dus dan is het 0
    \item Dan vergelijk je die rechte met uw datasheet van uw led.
\end{itemize}
\subsection{Sturing:}
Waarom:\begin{itemize}
    \item Energiebesparing 
    \item Flexibiliteit 
    \item Noodverlichting 
    \item Verlichting 
    \item Comfort en gezondheid
\end{itemize}
\begin{itemize}
    \item Manuele schakelaars
    \item Time control\begin{itemize}
        \item Nog steeds vaak gebruikt in bv shoppingscentra
        \item Eerder gebruikt als wijzigen/beperken van functies
    \end{itemize}
    \item Dimmen
    \item Sensoren als input voor sturing:\begin{itemize}
        \item Aanwezigheidsdetectoren:\begin{itemize}
            \item PIR
            \item Thermisch
            \item Cameras
            \item WiFi 
            \item \dots
        \end{itemize}
    \item Luxmetingen 
    \item RFID 
    \item Geluidssensoren
    \end{itemize}
\end{itemize}
\subsubsection{Sensoren:}\begin{itemize}
    \item Drukknoppen 
    \item Dimmers 
    \item Afstandsbediening 
    \item Timers 
    \item Beweging\begin{itemize}
        \item PIR 
        \item Ultrasoon 
    \end{itemize}
    \item Licht\begin{itemize}
        \item Fotocel
        \item Daglicht
    \end{itemize}
    \item Tijdsgebasseerd:\begin{itemize}
        \item Volgens de dag (weekdag/weekenddag)
        \item volgens de plaats 
        \item Meestal in niet-residentiele toepassingen
    \end{itemize}
    \item Aanwezigheidssensoren:\begin{itemize}
        \item PiR (Passive Infrared):\begin{itemize}
            \item Fotocel werkt via PiR
        \end{itemize}
        \item Ultrasonisch (beter voor langere afstanden)
        \item RFID sensor 
        \item Camera
    \end{itemize}
    \item Bedraad:\begin{itemize}
        \item DALI (Digital Addressable Lightning Interface)
        \item DMX 512
        \item Analoog 1-10V
        \item PWM
        \item RS232
        \item RS485 (Modbus)
        \item KNX
    \end{itemize}
    \item Draadloos:\begin{itemize}
        \item RF
        \item Bluetooth mesh
        \item Zigbee
    \end{itemize}
\end{itemize}
\paragraph{Dali:}\begin{itemize}
    \item Gestandaardiseerd protocol voor verlichtingsbeheer mbv uniek digitaal adres.
    \item Elk licht apart gecontroleerd (of per groep).
    \item Terugkoppeling voor status of alarmen
    \item Integratie in gebouwbeheersystemen
    \item Open protocol dus compatibel met breed scala aan
    verlichtingsproducten.
\end{itemize}
\paragraph{Zigbee:}\begin{itemize}
    \item Draadloos open 2-wegs communicatieprotocol
    \item Mesh netwerkarchitectuur
    \item Standaardisatie::\begin{itemize}
        \item Zo kan bv Philips Hue met Ikea gekoppeld worden
        \item Compatibel met veel devices:\begin{itemize}
            \item Tuya
            \item Sonoff
            \item WeLink
        \end{itemize}
    \end{itemize}
    \item API en MQTT
\end{itemize}
\subsection{Transformatoren:}
\subsubsection{Inleiding:}
\paragraph{Elektromagnetisme:}
Magnetisch veld (magnetische inductie):\begin{itemize}
    \item Een magnetisch veld is een veld dat de ruimte doordringt en dat magnetische kracht kan overbrengen:
    \item Onstaat door:\begin{itemize}
        \item Magneten 
        \item Elektrische stroom 
        \item Aarde
        \end{itemize}
    \item Aangegeven door vector B:\begin{itemize}
        \item Grootte op afstand r met stroom:\begin{itemize}
            \item B=$\mu *\frac{1}{2\pi*r}$
        \end{itemize}
    \end{itemize}
    \item In een geleider met lus:
    \item \item in een spoel:\begin{itemize}
        \item B=$\mu*\frac{I*N}{l}$
    \end{itemize}
    \item Tesla (T) of Weber (Wb)/m$^2$
\end{itemize}
Magnetische veldsterkte (intensiteit) (H):\begin{itemize}
    \item A/m 
    \item Hoe sterk kan je een andere substantie magnetiseren 
    \item H=$\frac{B}{\mu_o}$
\end{itemize}
ELektromagnetische krachtwerking:\begin{itemize}
    \item Stroomvoerende geleider in magnetisch veld:\begin{itemize}
        \item $\hookrightarrow$ Lorentzkracht $\rightarrow$ F=B.I.L
        \item $\hookrightarrow$ Newton
    \end{itemize}
\end{itemize}
Magnetische flux:\begin{itemize}
    \item Hoeveelheid magnetische veldlijnen door bepaald oppervlak 
    \item $\Phi$=B*A*cos($\phi$)
    \item Uitgedrukt in Weber (Wb)
\end{itemize}
Elektromagnetische inductie:\begin{itemize}
    \item Een winding zal een spanning opwekken als de magnetische flux door die winding verandert:\begin{itemize}
        \item $\hookrightarrow$ Opgewekte spanning noemt men inductiespanning 
        \item $\hookrightarrow$ Spoel werkt veranderde stroom (eigen flux) tegen door tegen spanning in te induceren 
    \end{itemize}
    \item De grootte van deze spanning is U=-N*$\frac{\Delta\Phi}{\Delta t}$
\end{itemize}
Magentische permeabiliteit $\mu$\begin{itemize}
    \item Hoe goed geleidt een materiaal de magnetische veldlijnen 
    \item $\mu_0$ magnetische permeabiliteit van vacuum.
    \item $\mu_r$ magnetische permeabiliteit van een materiaal in vergelijking met $\mu_0$ (dimensieloos)\begin{itemize}
        \item $\mu_r$ >1 $\Rightarrow$ ferromagnetische weerstand 
    \end{itemize}
\end{itemize}
Magnetische reluctantie $\mathcal{R}$:\begin{itemize}
    \item Tegenhanger elektrische weerstand 
    \item Hoe hard verzet een materiaal zich tegen magnetisme 
    \item $\mathcal{R}$=$\frac{I}{\mu_0 *\mu_r *A}$
\end{itemize}
\paragraph{Transformator:}
\begin{itemize}
    \item Een transformator, ook wel tranfo of trafo is een statische elektrisch apparaat, bestaande uit magnetische gekoppelde spoelen waartussen de energie-overdracht gebeurt via een magnetische koppeling, zodat er geen galvanische verbinding is tussen beide spanningsniveau's.
    \item Koppeling tussen 2 spanningsniveaus'!\begin{itemize}
        \item Vb. centrale van 20kV opboosten naar 36kV of 380kV
    \end{itemize}
    \item Zowel eenfasig als driefasig 
    \item Energie-overdracht op hoog rendement, werkt enkel op wisselspanning (Niet op DC).Men wil immers minimalisering van spannings- en vermogensverliezen en dit gebeurt bij zeer hoge
    spanningen. Optimaal voor op te transformeren van centrales en neer te transformeren bij woningen
\end{itemize}
\subsubsection{Eenfasige transformatoren:}
\paragraph{Bouw van eenfasige transformator:}
\begin{itemize}
    \item 2 spoelen, gewikkeld rond een kern van materiaal met een goede magnetische geleidbaarheid. 
    \item Het aanbrengen van magnetisch materiaal verbetert in hoge mate de werking van transformatoren.
\end{itemize}
\paragraph{Ideale eenfasige transformatoren:}
Sterk vereenvoudigd model:\begin{itemize}
    \item $\mu$FE is zeer groot:\begin{itemize}
        \item Relatieve permeabiliteit van de kern is zeer groot, dus magnetische flux volledig in het ijzer loopt
        \item Alle windingen van beide spoelen omsluiten de volledige flux, waardoor er geen lekflux is.
    \end{itemize}
    \item Geen ijzerverliezen:\begin{itemize}
        \item Een wisselflux veroorzaakt wervelstroom- en hysterisisverliezen, samen ook wel de ijzerverliezen genoemd:\begin{itemize}
            \item Wervelstromen kunnen beperkt worden door de kern niet uit massief materiaal, maar uit blikpakket samen te stellen (dunne blikplaten die ten opzichte van elkaar geïsoleerd zijn).
            \item Hysteresisverlies wordt veroorzaakt door remanent magnetisme dat in kern achterblijft wanneer deze gemagnetiseerd en gedemagnetiseerd wordt. Best weekijzer of ferriet gebruiken, deze kunnen eenvoudig gemagnetiseerd en gedemagnetiseerd worden.
        \end{itemize}
    \end{itemize}
    \item Geen koperverliezen:\begin{itemize}
        \item De wikkelingen hebben geen ohmse weerstand
        \item R1=R2=0
    \end{itemize}
    \item Geen verlies, er wordt gewerkt met een rendement van 100\%
\end{itemize}
\begin{enumerate}
    \item Koperverliezen 
    \item Lekflux 
    \item Wervelstroom,Hysteresisverlies=ijzerverliezen
\end{enumerate}
Remanent magnetisme:\begin{itemize}
    \item Bij het aanleggen van een uitwendig veld H, ontstaat een inductie B parallel met het veld.
\end{itemize}
\subparagraph{Nullast:}\begin{itemize}%zie pagina 15 van de slides
    \item Op de secundaire spoel zijn geen verbruikers aangesloten, Is = 0 
    \item Aan de eerste spoel wordt een wisselspanning u1 aangelegd. 
    \item De inductiewet* (wet van Faraday-Lenz) geeft de relatie weer tussen de geïnduceerde spanning en de wijziging van het magnetisch veld, daardoor volgt dat er een spanning e1 in de spoel 1 geïnduceerd wordt. Bij wijziging/verandering van de flux, wijzigt de spanning.
\end{itemize}
\paragraph{Technische eenfasige transformator:}
Ommagnetiseren van kern + wervelstroom:\begin{itemize}
    \item warmte-ontwikkeling 
    \item energie uit het net 
\end{itemize}
nullaststroom I$_0$:\begin{itemize}
    \item Magnetisatiestroom Im 
    \item Iw (Iy)=ijzerverliezen
\end{itemize}
\subparagraph{Belasting:}
\begin{itemize}
    \item Sluiten van secundaire winding van de ideale transformator met een impedantie, dan vloeit door deze wikkeling een stroom I2 waarvan de grootte en faseverschuiving ten opzichte van de secundaire spanning U2 bepaald worden door de impedantie.
    \item Primaire stroom bestaat uit 2 componenten (ideale transfo):\begin{itemize}
        \item Magnetiseringsstroom im1, die de nodige ampèrewindingen levert om de flux op te bouwen:\begin{itemize}
            \item Bij vollast is deze zeer slechts een heel klein aandeel van de totale primaire stroom.
            \item Magnetiseringsstroom is 90° naijlend op de aangelegde spanning. Het opgenomen actief vermogen voor een transformator waarin alleen Im vloeit is dus 0 (cos90° = 0) , in fase met de flux
        \end{itemize}
        \item De belastingsafhankelijke stroomcomponente nodig om de ampèrewindingen van de secundaire belastingsstroom mee te compenseren
        \item (3de stroomcomponente bij reële transformator)\begin{itemize}
            \item Ten gevolge van hysteresis en wervelstroomverliezen
        \end{itemize}
    \end{itemize}
\end{itemize}
Vectordiagramma's:% zie pagina 20 slides.
Om de secundaire flux tegen te werken gaat de primaire flux even groot worden om dit tegen te werken zodat we enkel nog onze Im over houden. Als we dit op schema willen tekenen zal onze Ip' recht tegenover onze Is. Dus als we een spoel met een weerstand hebben zal onze Is links van de Ep,Es lijn liggen en onze Ip' zal daar recht tegenover liggen. Ip' zal in dit voorbeeld in kwadrant 1 liggen en Is in kwadrant 3.
\subparagraph{Verliezen transformator:}
\begin{itemize}
    \item Ijzerverliezen (nullastverliezen:)\begin{itemize}
        \item Watt-verliezen in de kern (hysterese en wervelstroom)
        \item Opwarming kern 
        \item Onafhankelijk van belasting 
        \item Verhogen bij verhogen van spanning en frequentie
    \end{itemize}
    \item Koperverliezen (Jouleverliezen)\begin{itemize}
        \item Bij nullast zeer klein, verhogen kwadratisch met de belasting
        \item Ohmse weerstand
        \item $Pcu_totaal = Pcu_p + Pcu_s = Ip^2*Rp + Is^2*Rs$
    \end{itemize}
\end{itemize}
\subsection{Vermogen en rendementen:}
We moeten een in driehoek geschakelde motor aansluiten op een transfo die wij moeten
dimensioneren. Het net levert 400V - 50Hz. De kenplaat gegevens van de motor:
Als we 400V op driehoek aansluiten dan zullen onze wikkelingen doorbranden van de transfo. Als we 230V op ster zetten dan zal de transfo wel werken maar zal je een heel slecht rendement hebben.
\begin{itemize}
    \item We kunnen berekenen hoeveel wikkelingen we moeten hebben voor spanning op secundaire door $\frac{N1}{N2}$=$\frac{U1}{U2}$. als we 230V willen op secundaire en we hebben op primaire 400V dan zullen we $\frac{400}{230}$=1.739. Voor stroom te weten 
\end{itemize}
\paragraph{Serie en parallel schakeling:}
Moeten we niet kennen maar is wel belangrijk.\begin{itemize}
    \item Secundaire wikkelingen in serie schakelen?\begin{itemize}
        \item Spanning in secundaire wordt 2U
    \end{itemize}
    \item Secundaire wikkelingen in parallel schakelen?\begin{itemize}
        \item Stroom in secundaire wordt 2I
    \end{itemize}
\end{itemize}
\paragraph{Belang kortsluitspanning:}
\subparagraph{Oefeningen:}
\begin{itemize}
    \item Oefening 1:\begin{itemize}
        \item Een transformator 3000V/240V heeft primair 1500 wikkelingen. De secundaire levert
        een stroom van 75A. Bereken het aantal secundaire windingen en de primaire stroom.
        \item Gegeven:\begin{itemize}
            \item Wikkelingen primair=1500
            \item Up=3000 
            \item Us=240 
            \item Is=75A
        \end{itemize}
        \item Gevraagd:\begin{itemize}
            \item Wikkelingen secundair 
            \item Ip
        \end{itemize}
        \item Uitwerking:U1/U2 $\rightarrow$ 3000/240=12.5 $\rightarrow$ $\frac{N1}{N2}$=12.5
        \item $\rightarrow$ $\frac{1500}{N2}$=12.5$\rightarrow$N2=1500/12.5=120 wikkelingen.
        \item De secundaire wikkelingen is gelijk aan 120
        \item Ip=$\frac{Is*Vs}{Vp}$ $\rightarrow$Ip=$\frac{240*75}{3000}$=6A
    \end{itemize}
\end{itemize}
\subsection(Motoren en generatoren:)
\subsubsection{Motoren:}
\paragraph{efficiëntieklasses:}
Veel besparing mogelijk, maar ook investeringskost!\begin{itemize}
    \item Geldig voor 50 en 60Hz motoren 
    \item Tot 375 kW! 
\end{itemize}
In europa Ie3 motoren.
\paragraph{Inductiemotoren:}
Verschillende namen:\begin{itemize}
    \item Inductiemotor 
    \item Asynchrone motor 
    \item Draaistroommotor 
    \item \dots 
\end{itemize}
Meest gebruikte elektrische motor:\begin{itemize}
    \item 90 tot 95\% van de industriële motoren zijn inductiemotoren
    \item Eenvoudig qua constructie, vrijwel onderhoudsvrij, robuust en niet duur, lage productiekost, relatief hoge efficiëntie,...
    \item 50\% van de elektriciteit wordt geleverd voor elektromechanische energieconversie
    \item Aankoopprijs van inductiemachines vormt slechts 1,5 tot 4\% van de totale levenskost! De rest van de kosten zijn vooral energiekosten en onderhoudskosten!
\end{itemize}
Nadeel: Toerental van inductiemachine ligt vast. Hiervoor hebben we vermogenelektronische omvormers (drive of frequentieregelaar).\begin{itemize}
    \item Is in staat om frequentie te wijzigen terwijl de spanning mee varieert en het koppel kan behouden blijven 
\end{itemize}
\paragraph{Opbouw:}
\begin{itemize}
    \item Stator:Opgebouwd uit gelamelleerd Silicium-staal\begin{itemize}
        \item Doel: om elektrische weerstand ijzer te verhogen worden lamellen geïsoleerd ten opzichte van elkaar om het vermogenverlies in de kern door wervelstroomverliezen te beperken.
        \item Gleuven voorzien voor driefasen statorwikkeling die onderling 120 graden verschoven zijn.
    \end{itemize}
    \item Rotor: Verschillende uitvoringsvormen:\begin{itemize}
        \item Kooirotor:\begin{itemize}
            \item In de gleuven aan de omtrek van de rotor zijn staven aangebracht. 
            \item Voor- en achterzijde van de rotor zijn staven door kortsluitringen verbonden met elkaar, waardoor de elektrische
            kring gesloten is
        \end{itemize}
        \item Bewikkelde rotor:\begin{itemize}
            \item Bij dit type rotor zijn er in de gleuven van de rotor 3 wikkelingen aangebracht in ster. 
            \item Andere uiteinde wordt dmv sleepringen naar buiten gebracht.
            \item Dmv weerstanden (extra jouleverliezen) worden uiteindes extern kortgesloten zodat kring gesloten is.
        \end{itemize}
    \end{itemize}
\end{itemize}
\paragraph{Werking:}
Draaiveld opwekken:\begin{enumerate}
    \item Aansluiten van driefasenspanning (L1,L2,L3)
 op 3 wikkelingen (ster of driehoek) (U1-U2/V1-V2/W1-W2) in de stator zal er een driefase stroom vloeien in de wikkelingen.
    \item Fasestroom in wikkelingen via U1 naar U2,... laat ons toe om de richting van de veldlijnen te bepalen.
    \item De veldlijnen rond de wikkelingen zorgen voor een magnetisch veld met x poolparen. Dit magnetisch veld draait rond (draaiveld) met een bepaalde rotatiefrequentie
\end{enumerate}
\subparagraph{poolparen:}
2 poolparen=4-polige motor.\begin{itemize}
    \item 
\end{itemize}
Opwekken van rotorspanningen:\begin{itemize}
    \item Wanneer de rotor stilstaat, worden de staven van de kooirotor, of de
    wikkelingen van de bewikkelde rotor gesneden door het draaiveld, opgewekt
    door de stator.
    \item Hierdoor zal een spanning opgewekt worden erotor = Bstator x lrotor x $\Delta$v\begin{itemize}
        \item Afhankelijk van: magnetische flux, de lengte van de geleider/spoel en de relatieve snelheid
    \end{itemize}
\end{itemize}
Opwekken van rotorstromen:\begin{itemize}
    \item Kooirotor -> Rotorstaven dmv kortsluitringen kortgesloten, hierdoor zal door de opgewekte rotorspanningen een rotorstroom vloeien.
    \item Bewikkelde rotor -> Rotorwikkelingen extern kortgesloten.
\end{itemize}
Hoe sneller dat uw motor draait hoe minder koppel dat hij kan geven. In praktijk kan uw $\Delta$v nooit 0 zijn want anders zouden uw rotor en stator reven snel draaien. en dan zou je geen stroom,kracht hebben. Uw rotor zal altijd wat achter draaien (slip).
\begin{itemize}
    \item Ontstaan van slip:\begin{itemize}
        \item In de rotorstaven wordt maar een spanning opgewekt wanneer er een snelheidsverschil is tussen de bewegende rotorstaven en het draaiveld van de stator. 
        \item Rotor mag dus nooit aan zelfde snelheid bewegen als draaiveld stator, anders zouden er geen veldlijnen gesneden worden en zal er geen stroom in de rotor opgewekt worden en ontstaat er geen magnetische pool
        \item Rotorsnelheid altijd trager dan snelheid draaiveld stator! Dit verschil in toerental noemen we slip.\begin{itemize}
            \item Vandaar ook de naam, asynchrone machine!
        \end{itemize}
        \item s=nS-n:\begin{itemize}
            \item s = slip (tr/min)
            \item ns = synchroon toerental (toerental draaiveld) (tr/min)
            \item n = toerental rotor (tr/min)
        \end{itemize}
        \item S = (ns - nr)/ns:\begin{itemize}
            \item uitgedrukt in [\%]
            \item Meestal tussen 4 en 11\%
        \end{itemize}
    \end{itemize}
\end{itemize}
om te weten of toerental kan: 50hz*60seconden=toerental (1 poolpaar)
indien meerdere poolparen moet je dan uw 50*60 delen door uw aantal poolparen.
\paragraph{koppel-toerentalkarakteristiek:}
\begin{itemize}
    \item Wanneer we het koppel in functie van het toerental van de rotor in een grafiek
    uitzetten spreken we over de koppel-toerentalkarakteristiek.
    \item Aflezen hoe groot het koppel is bij een bepaald toerental
\end{itemize}
%tot slide 23 ongeveer te kennen.
\end{document}
