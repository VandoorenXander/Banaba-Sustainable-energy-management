
\documentclass[12pt]{article}

\usepackage{amsmath}

\usepackage{microtype}

\usepackage{graphicx}

\usepackage{hyperref}

\usepackage[utf8]{inputenc}

\usepackage{listings}

\usepackage{matlab-prettifier}
% % voor code syntax highlighting
% \usepackage{minted}

% beter font
\usepackage[T1]{fontenc}
\usepackage{helvet}
\setcounter{secnumdepth}{5}
\setcounter{tocdepth}{5}
\renewcommand{\familydefault}{\sfdefault}
\graphicspath{{imagessoft/}}
\begin{document}
\begin{titlepage}
    \author{Xander Vandooren}
    \title{Business Strategy}
\end{titlepage}
\pagenumbering{gobble}
\maketitle
\newpage
\tableofcontents
\newpage
\pagenumbering{arabic}
\section{Missie en Visie:}
Een organisatie of bedrijf heet een 
textbf{missie} en een \textbf{visie}. De missie en visie zijn ook de leidraad bij alles wat er in het bedrijf gebeurt.
\begin{itemize}
    \item In een missie wordt het \underline{bestaansrecht} van een organisatie beschreven vanuit de \textbf{waarden} en identiteit. De vraag die onder dit verhaal ligt is: waar staan we voor?
    \item Een missie is iets wat door de jaren heen gegroeid is, het is \underline{het fundament} van een organisatie, veelal gekoppeld aan de oprichter van het bedrijf of organisatie. Deze oprichter had een droom en wilde met zijn organisatie iets wenzelijks bereiken in de wereld. Het mooiste is dat op een bepaald moment de missie iets is geworden dat door heel het bedrijf gedragen wordt. Dan is de missie geintegreerd in de handelswijze van de medewerkers.
    \item Een missie is een dieper liggende motivator om je bijdrage als bedrijf of organisatie te leveren. Als de missie helder is, dan kan een bedrijf in een storm terecht komen, maar dan weet iedere medewerker waarom hij deed wat hij deed en die wetenschap helpt
\end{itemize}
\subsection{Missie:}
Een missie is een blijvend m.a.w iets dat \underline{niet elk jaar verandert}.\newline
Hulpvragen om een missie te formuleren:
\begin{itemize}
    \item Waar staan we voor?
    \item Welke normen en waarden hebben we?
    \item Welke overtuigingen motiveren ons?
    \item Wat is ons bestaansrecht?
\end{itemize}
\subsection{visie:}
De visie geeft een antwoord op de vraag \underline{waar gaan we voor}?
\begin{itemize}
    \item In een visiestuk wordt beschreven wat de \underline{toekomstdroom van de organisatie} is en welke bijdrage de organisatie aan de maatschappij wil leveren.
    \item In een visie krijgt de missie van de organisatie vorm en richting.
    \item Een visie wordt vaak geschreven voor een bepaalde periode, wordt geevalueerd en bijgesteld indien nodig.
\end{itemize}
Hulpvragen om een visie te formuleren:\begin{itemize}
    \item Wat willen we bereiken?
    \item Wat zijn onze kernwaarden?
\end{itemize}
\subsection{Missie en visie door elkaar gehaald}
%kijk naar slide
\subsection{waarden}
%kijk naar slide
\subsection{Visie-Missie-Strategie}
Waarden $\Rightarrow$ Missie $\Rightarrow$ Visie $\Rightarrow$ Strategie
\begin{itemize}
    \item Waarden: Waar staan we voor?
    \item Missie: Waarom bestaan we? (primaire opdracht)
    \item Visie: Waar willen we binnen 10 jaar staan? (Ambitieus beeld van de toekomst)
    \item Strategie: Hoe geraken we daar? (Concrete doelstellingen en kritische succesfactoren)
\end{itemize}
\subsection{Formuleren doelstellingen}
\textbf{Wat willen we bereiken?}
Aan de hand van een SWOT-analyse kunnen we de gerichte doelstellingen formuleren.
\begin{itemize}
    \item Waar zijn we goed in? (\textbf{S}trenghts)
    \item Waar zijn we minder goed in? (\textbf{W}eaknesses)
    \item Wat zijn kansen? (\textbf{O}pportunities)
    \item Wat zijn onze bedreigingen? (\textbf{T}hreats)
\end{itemize}
Doelstellingen worden vaak te vaag en vrijblijven geformuleerd als wensen, intenties, of goede voornemens. Een methode om dit effectiever te maken is gebruik maken van SMART-formulering:
\begin{itemize}
    \item Specifiek 
    \item Meetbaar
    \item Acceptabel
    \item Realistisch
    \item Tijdgebonden
\end{itemize}
Een SMART-doelstelling is dus richtinggevend: het geeft aan wat je wilt bereiken en stuurt het gedrag van je medewerkers en van jezelf. Bovendien wordt aangegeven welke resultaten wanneer moeten worden bereikt.
\subsubsection{Specifiek:}
\begin{itemize}
    \item Wat willen we bereiken?
    \item Wie is betrokken?
    \item Waar gaan we het doen?
    \item Wanneer gebeurt het?
    \item Welke delen van de doelstelling zijn essentieel?
    \item Waarom willen we dit doel bereiken?
\end{itemize}
Het moet een \textbf{waarneembare} actie, gedrag of resultaat beschrijven waaraan een getal, bedrag, percentage of een ander \textbf{kwantitief} gegeven verbonden is.
\subsubsection{Meetbaar}
%slides
\subsubsection{Acceptabel}
%slides
\subsubsection{Realistisch}
%slides
\subsubsection{Tijdsgebonden}
\subsection{Kwaliteit:}
kwaliteit = voldoen aan \underline{de verwachtingen} $\Rightarrow$
\begin{itemize}
    \item bedrijf
    \item klant
\end{itemize}
Niet vanbovenaf $\rightarrow$ stappen zelf laten controleren. En hopelijk later het controleren kunnen achterwege laten weten dat het kwaliteitsvol uitgevoerd wordt.
\underline{Wat is kwaliteit?}
De mate waarin we iets goeds doen.
kwaliteit is de vereisten, verwachtingen en eisen van klant en bedrijf.
\subsubsection{Kwaliteitsmanagementsysteem:}
een \underline{doelmatige} (winstgevende,kostenefficiente) manier van werken binnen een organisatie die er toe leidt dat de organisatie goede producten en diensten levert waar de klanten tevreden over zijn. Een systeem kan al of niet gestandaardiseerd zijn (vb. ISO of EFQM).\
\begin{itemize}
    \item[] Kwaliteitsysteem moet een organisatie toegevoegde waarde bieden. Dit zijn praktische systemen waar uw organisatie en uw klanten wat aan hebben.
\end{itemize}
\begin{itemize}
    \item ISO 9000/ ISO 90001:\begin{itemize}
        \item Iso 9000-serie is sinds jaren negentig uitgegroeid tot de bekendste standaard norm op het gebied van kwaliteitszorg. Alle gangbare normen zijn afgeleid van de ISO 9000 serie. ISO 9001 bestaat uit een aantal eisen waaraan een organisatie dient te voldoen. 
        \item In ISO 90001 staat uw klant, en haar behoeften en verwachtingen centraal.
    \end{itemize}
\end{itemize}
Voordeel van kwaliteitssystemen is tweeledig: het geeft \textbf{vertrouwen naar de klant} en het \textbf{verhoogt de efficientie van uw organisatie}.
\subsubsection{PDCA:}
Plan-Do-Check-Act (PDCA) ook wel de Deming cyclus genoemd wordt vaak gebruikt voor de basis voor continue verbetering. \begin{itemize}
    \item Plan: Uw plannings fase is de belangrijkste in de cyclus, voor je iets doet zorg je zeker dat je een goed plan hebt. Er is hier nog niets gedaan/ gerealiseerd maar we zetten op papier wat we gaan doen.
    \item Do: Hier voeren we ons plan uit, ondernemen we bepaalde acties, verwezenlijken we iets.
    \item Check: Zorg zeker dat dit goed is want dit kan u veel tijd en geld kosten indien het niet goed gecontroleerd wordt. We checken hier of het gerealiseerde overeenkomt met het plan. We doen bepaalde metingen en vragen bv. feedback van de klant
    \item Act: ALs er iets niets goed is of kan verbeterd worden kan je het bijsturen.
    Er zijn 3 mogelijke resultaten:\begin{itemize}
        \item Eerste geval P = D , is je product of dienst Ok 
        \item tweede geval P > D voldoet ons product of dienst (nog) niet (volledig) aan het gevraagde. Je zal dus best bijsturen.
        \item Derde geval, P < D , is je product of dienst beter dan het geplande,gevraagde.
    \end{itemize}
\end{itemize}
Aanpassen proces/manier van werken leidt tot bijsturing maar:\begin{itemize}
    \item maak da in toekomst niet zelfde fouten gemaakt worden
    \item Trek dit door naar vergelijkbare processen
    \item neem het op in aangepaste procedures
\end{itemize}
Verankeren in de werking (= borgen)
schematiseren,structureren
\subsubsection{ISO 9001:}
\begin{itemize}
    \item Internationaal erkende norm \textbf{expertise en knowhow rond kwaliteitsmanagement} van laatste decennia in 1 document bundelt.
    \item eist dat \textbf{processen beschrijven, analyseren, meten en continu verbeteren (PDCA)}
    \item opgesteld volgens \textbf{High Level Structure (HLS)} hiermee krijgen tientallen ISO-managementnormen een uniforme basisstructuur en dito vereisten.
\end{itemize}
\subsubsection{High level structure (HLS):}
ISO 9001 omvat 10 hoofdstukken:\begin{enumerate}
    \item Toepassingsgebied (Scope)\begin{itemize}
        \item Zorg dat uw scope correct en duidelijk is.
    \end{itemize}
    \item Normatieve verwijzingen
    \item Definities
    \item Context van de organisatie
    \item Leiderschap
    \item Planning
    \item Ondersteuning
    \item Uitvoering
    \item Evaluatie
    \item Verbetering
\end{enumerate}
\subsubsection{Voordelen van ISO 9001:}
\begin{enumerate}
    \item Hogere omzet
    \item Aantoonbare kwaliteit
    \item Lagere kosten
    \item Grotere flexibiliteit
    \item Nieuwe markten
    \item Meer klantentevredenheid
    \item Beter overzicht
    \item Op maat van elke organisatie
    \item Sterk engagement
\end{enumerate}
\subsubsection{EFQM}
European Foundation for Quality management\begin{itemize}
    \item EFQM-Exellence model is een gestructureerde methode om een inzicht te verkrijgen in het niveau van uitmuntendheid van een organisatie.
    \item Met als doel \textbf{excellente resultaten te realiseren} voor zowel klanten,medewerkers,budgetverschaffers en de samenleving.
\end{itemize}
Het EFQM model kan opgesplits worden in 2 grote kaders \begin{itemize}
    \item Factoren:\begin{itemize}
        \item Leiderschap
        \item Personeel
        \item Strategie
        \item Partnerschappen en middelen
        \item Processen, Producten en diensten
    \end{itemize}
    \item Resultaten:\begin{itemize}
        \item Personele resultaten
        \item Klantresultaten
        \item Maatschappelijke resultaten
        \item Bedrijfsresultaten
    \end{itemize}
\end{itemize}
We hopen de omzet van onze klanten te zien en te vergelijken met onze verwachtingen om te zien of alles goed gaat. Dit alles samen kunnen we key performance indicators (KPI's) noemen.
\begin{itemize}
    \item[] Het EFQM-model heeft een sterk \textbf{strategisch karakter:}\begin{itemize}
    \item[] Structuur van het EFQM-model is gebasseerd op het stellen van 3 vragen:
        \item Waarom $\Rightarrow$ Richting
        \item Hoe $\Rightarrow$ Uitvoering
        \item Wat $\Rightarrow$ Resultaten
    \end{itemize}
\end{itemize}
\paragraph{Radar:}
Radar is de EFQM versie van PDCA (praktisch hetzelfde).
Radar staat voor Results, Apporoaches,Deploy,Assess \& Refine
\subsection{Key Performance Indicators:}
Dit zijn variabelen om de prestaties van een bedrijf, merk of product te analyseren.
\begin{itemize}
    \item[]\textbf{KPI's worden gebruikt om:}\begin{itemize}
        \item Het success van een organisatie in het algemeen objectief te kunnen meten.
        \item Of het succes van een bepaalde actie of campagne te peilen.
        \item Mate van succes en de voortgang van langetermijndoelen inzichtelijk weer te geven.
    \end{itemize}
\end{itemize}
\subsubsection{12 gouden regels voor KPI's:}
\begin{enumerate}
    \item Meet een proces (geen eenmalige dingen)
    \item Weet wat het doel is
    \item Duidelijk voor iedereen (helder)
    \item Meetbaar
    \item Eenvoudig uit te leggen (betrokkenheid)
    \item Toekomstgericht
    \item Afgestemd op andere KPI's
    \item De KPI is bij te sturen
    \item KPI heeft een doel
    \item Evalueer de KPI's ifv de tijd en pas aan 
    \item Beloon medewerkers op basis van het behalen van de KPI's
    \item beperk het aantal KPI's
\end{enumerate}
\paragraph{verschil ISO en EFQM:}
EFQM en ISO zijn grotendeels hetzelfde, het grootste verschil is dat je bij ISO voornamelijk wacht op feedback terwijl je bij EFQM al begint stappen te ondernemen voor je feedback hebt omdat je streeft voor 'perfectie' (Je wil continue resultaten/verbetering).
\section{Bedrijfslexicon:}
\subsection{Agile:}
\begin{itemize}
    \item Agile staat letterlijk voor wendbaar, lenig en flexibel.
    \item Agile organiseren is een manier van denken, werken en organiseren. Het stelt organisaties in staat om \underline{snel en effectief} in te spelenop veranderingen in de buitenwereld.
    \item Daarbij staat \underline{tevredenheid van klanten} voorop, met oog voor het welzijn van de mederwerkers en de gezonde bedrijfsresultaten van de organisatie zelf.
    \item Afkomstig uit software-ICT 
    \item "Sprint organisatie"
\end{itemize}
\begin{itemize}
    \item[]Agile managment:\begin{itemize}
        \item Scrum
        \item Kanban
    \end{itemize}
\end{itemize}
\textbf{Agile:} breekt grote productontwikkelingen op in korte, overzichtelijke periodes (iteraties) van 2 tot maximaal 4 weken. Die iteraties zijn kleine op zichzelf staande projecten die door 'Timeboxing' gestuurd worden. 'Timeboxing' is belangrijk, omdat sprint na sprint wordt getrokken in een vast 'ritme'.
\subsubsection{Waterfall vs Agile:}
\begin{itemize}
    \item \textbf{Waterfall:}\begin{itemize}
        \item Structuur: Lineaire en sequentiele aanpak.Elke fase moet volledig worden afgerond voor je de volgende fase kan beginnen.
        \item Flexibiliteit: Minder flexibel. Veranderingen zijn moeilijk door te voeren zodra een fase is afgerond.
        \item Feedback:Feedback wordt meestal pas aan het einde van het project verzameld.
        \item Documentatie: Uitgebreide documentatie aan het begin van het project. Alle vereisten en specificaties worden vooraf vastgelegd.
        \item Risico's: Hogere risico's omdat problemen vaak pas laat in het project worden ontdekt
    \end{itemize}
    \item \textbf{Agile:}\begin{itemize}
        \item Structuur: Iteratieve en incrementele aanpak. Werk wordt uitgevoerd in korte cycli (sprints), continue evaluatie en aanpassing mogelijk is.
        \item Flexibiliteit: Zeer flexibel, Veranderingen kunnen op elk moment worden doorgevoerd op basis van feedback en nieuwe inzichten.
        \item Feedback: Regelmatige feedback van klanten en stakeholders tijdens elke sprint, wat helpt om problemen vroegtijdig te identificeren en op te lossen.
        \item Documentatie: Minder nadruk op documentatie: Focus ligt meer op werkende software en continue communicatie.
        \item Risico's: Lagere risico's doorcontinue evaluatie en aanpassingen, waardoor problemen vroegtijdig worden aangepakt.
    \end{itemize}
\end{itemize} 
\paragraph{voorbeeld van waterfall:}
 \textbf{DBFM: Design, build,finance,maintain gebouw/huis}:\begin{itemize}
    \item vooraf moet in design alles vast staan en als er aanpassingen gebeuren moeten er veel administratie gebeuren.
    \item eenmaal het gebouwt is moeilijk aanpassingen te doen
    \item wordt zeker 20-30 jaar gefinancieerd.
    \item moet dus ook 20-30 jaar zeker onderhouden worden.
\end{itemize}
\subsection{Lean:}
\begin{itemize}
    \item Lean is een business strategie en vooral een manier van werken waarbij alles en iedereen in de onderneming zich richt op het creeren van \underline{waarde} voor de klant in alle processen.
    \item Hiervoor worden verspillingen geelimineerd. DOor verspillingen te elimineren gaan de operationele kosten omlaag, wat in het algemeen leidt tot een verbetering van het bedrijfsresultaat.
    \item "slanke organisatie".
\end{itemize}
Verspillingen?$\rightarrow$\begin{itemize}
    \item Tijd 
    \item Ruimte\begin{itemize}
        \item opslag 
        \item productieruimte
    \end{itemize}
    \item Personeel 
    \item Materiaal
\end{itemize}
\subsubsection{Activiteiten:}
\begin{itemize}
    \item Waardetoevoegende activiteiten. Die moeten behouden worden.
    \item Niet-waardetoevoegende activiteiten, die echter wel nodig zijn om aan externe en interne eisen te voldoen. Denk hierbij aan wet en regelgeving en intern beleid. Die moeten verminderd worden.
    \item verspillingen. Die moeten geelimineerd worden.
\end{itemize}
\subsubsection{8 standaardvormen van verspilling:}
\begin{itemize}
    \item Correctie: aan een product of dienst moet voor de tweede keer gewerkt worden omdat er de eerste keer iets fout is gegaan.
    \item Overproductie: meer of sneller producten dan de klant vraagt.
    \item Wachten: de medewerker of klant moet het product of de dienst stilleggen totdat de collega of leverancier iets heeft gedaan.
    \item Talent: Inzetten van mensen op werk dat ze niet leuk vinden, hen niet uitdaagt of waarvoor ze de kennis en vaardigheden missen.
    \item Transport: tussen- , eindproducten, medewerkers en soms zelfs klanten worden verplaatst.
    \item Voorraad: Materialen en capaciteit die op dit moment niet gebruikt worden.
    \item Beweging: De medewerker of de klant moet niet-waardetoevoegende handelingen uitvoeren om het product of de dienst te kunnen opleveren of gebruiken
    \item Overbewerking: Meer toevoegen aan het product of de dienst dan de klant nodig heeft.
\end{itemize}
\subsubsection{Six Sigma(6$\sigma$):}
Wat is Six Sigma:\begin{itemize}
    \item Six sigma is een methode van organisatie tools die de kapabiliteit verbeteren van het business process. Deze verbetering in performance en verminderding van process variatie helpt met defect reductie en verbeterd winsten,werknemer moraal en de kwaliteit van producten of services.
\end{itemize}
\paragraph{DMAC:}
DMAIC-cyclus:\begin{itemize}
    \item Define 
    \item Measure 
    \item Analyze 
    \item Improve 
    \item Control
\end{itemize}
%misschien nog toevoegen verschil lean en six sigma
\paragraph{6$\sigma$ != 6S:}
Six sigma is voor gefocused op afwijkingen terwijl 6S gaat over een manier van werken.
\subsubsection{6S}
\begin{enumerate}
    \item Sort:De eerste stap in 6S, probeerd klutter te reduceren.\begin{itemize}
        \item Red-tagging: neem alle dingen/voorwerpen die je niet nodig hebt weg.
        \item Yellow-tagging: voor specifieke dingen/voorwerpen die niet direct nodig zijn maar wel af en toe van belang kunnen zijn.
    \end{itemize}
    \item Straighten: systemen,voorwerpen,werkmateriaal en andere essentiele dingen moeten op een logische manier gesorteerd/geplaatsd worden (ergonomie).
    \item Scrub: Zorgen dat de werkplaats proper blijft, zorg dat uw omgeving nettjes blijft want een vuile omgeving kan effect hebben op uw product en gezondheid van werknemers.
    \item Standardize: Indetificeer de beste practices en creeer consistente procedures voor de eerste 3 stappen.
    \item Sustain: Dit is de moeilijke stap van de 6S methodes omdat het doel is om er voor te zorgen dat de gestandaardiseerde procedures constant toegepast worden tot het second nature wordt voor de werknemers.
    \item Safety: De laatste stap is veiligheid (Safety) is later toegevoegd bij de 5S methologie. Het is belangrijk om gevaren te kunnen identificeren en preventeren dat er gevaarlijke situaties kunnen plaatsvinden. De werk omgeving moet verplichte safety standards hebben.
\end{enumerate}
\subsection{MRP/ERP:}
MRP is voornamelijk gefocused op materialen en ERP is op het geheel gefocused (materiaal en de productie etc.)
\subsubsection{MRP:}
\begin{itemize}
    \item Material requirements planning (MRP) wordt in de logistiek gebruikt voor de planning van de materiaalinkoop (grondstoffen en componenten).
    \item Daarnaast kan hiermee ook het productieproces worden vastgesteld, omdat er bij MRP gewerkt wordt met een van tevoren vastgesteld verkoop- en productieplan van het eindproduct.
\end{itemize}
\paragraph{Bill of materials (BOM)}
\subparagraph{Economic Order Quantity:}
\begin{itemize}
    \item De formule van Camp is een klassieke, wiskundige berekening van de optimale bestelhoeveelheid. De formule wordt ook wel de Economic Order Quantity (EOQ)-formule genoemd.
    \item Hoe vaker een product besteld wordt, des te hoger zullen de bestelkosten zijn. Echter, wie vaker bestelt hoeft minder voorraad aan te houden en heeft lagere voorraadkosten. De formule van Camp  berekent de optimale verhouding tussen de \textbf{bestelkosten} en de \textbf{voorraadkosten}.
\end{itemize}
De formule hiervoor is: $Q=\sqrt{\frac{2DF}{hP}}$\begin{itemize}
    \item Q= optimale bestelheoveelheid (Quantity).
    \item D= Jaarlijkse vraag naar het product (Demand).
    \item F= Bestelkosten per bestelling (Fixed costs).
    \item h= Voorraadkosten als percentage van de prijs (Holding cost rate).
    \item P= Prijs per eenheid van het product (Price).
\end{itemize}
\subsection{ERP:}
Enterprise resource plannning (ERP) is het geintegreerde beheer van belangrijke bedrijfsprocessen, vaak in realtime door software en technologie. ERP wordt gewoonlijk een categorie \underline{bedrijfsbeheersoftware} genoemd - meestal een reeks geintegreerde applicaties - die een organisatie kan gebruiken om gegevens van veel bedrijfsactiviteiten te verzamelen, op te slaan, te beheren en ter interpreteren.
\begin{itemize}
    \item backups
    \item Software upgrade
    \item Geintegreerd
    \item Security
    \item Keuze?
\end{itemize}
\subsection{Supply Chain Management}
Supply chain management (SCM), ook wel integraal ketenbeheer genoemd, is een principe waarbij door middel van het verbeteren van processen en samenwerking met leveranciers en afnemers een betere functionaliteit van het deelnemende bedrijf in de keten ontstaat.
\subsection{Jaarrekening/balans:}
\subsubsection{Balans:}
\begin{itemize}
    \item Actief:\begin{itemize}
        \item het balans toont alle bezittingen (\textbf{'activa'}) van het bedrijf:\begin{itemize}
            \item Hoef heeft dit bedrijf zijn geld besteed en gebruikt?
            \item Welke investeringen heeft het bedrijf gedaan?
            \item Denk bvb aan terreinen,gebouwen,machines,kantoormeubelen,\dots
        \end{itemize}
    \end{itemize}
    \item Passief:\begin{itemize}
        \item De balans vertelt je hoe al die bezittingen op actief werden bekostigd:\begin{itemize}
            \item Vanwaar haalt het bedrijf zijn geld?
            \item Wie heeft geld geinvesteerd in het bedrijf?
        \end{itemize}
    \end{itemize}
\end{itemize}
\subsection{Sales/marketing:}
\subsubsection{Marketing:}
Marketing of vermarkten is alles wat een bedrijf doet om de verkoop van de producten of diensten te bevorderen.
\paragraph{Marketingmix:}
\subparagraph{Philip Kotler: de 5 P's van de marketingmix:}\begin{itemize}
    \item Product 
    \item Prijs
    \item Place
    \item Promotion
    \item People
\end{itemize}
\subsubsection{Verkoop:}
Verschil tussen marketing en verkoop omvat een maatschappij waarin consumenten kunnen kiezen uit allerlei producten en diensten die nauwkeurig op hun wensen en behoeften zijn afgestemd, in plaats van een maatschappij waarin de mensen die keuzes niet of nauwelijks hebben.\newline
Verkopen is 'zien kwijt te raken wat je op de plank hebt liggen', terwijl marketing is:'ervoor zorgen da je de juiste spullen voor je klanten op de plank krijgt'.\begin{itemize}
    \item b-to-b: business to business
    \item b-to-c: business to customer
\end{itemize}
\paragraph{Customer relation management (CRM):}
Dit kan helpen voor te helpen dat uw klant uw "promoter" wordt. een goeie customer survice kan daar ook goed meehelpen. 
\paragraph{Sales pitch:}
Sales pitches zijn presentaties die visueel de voordelen van de producten of diensten van een bedrijf aan een potentiele klant tonen.\newline
Het belangrijkste is da je sales pitch moet \textbf{overtuigen} en \textbf{converteren}.
\begin{itemize}
    \item Het probleem/de kans 
    \item De ideale oplossing 
    \item Het product /concurrentievoordeel
    \item Het bewijs
\end{itemize}
\subsection{Innovatie:}
\begin{itemize}
    \item Kennis-intensief
    \item Cumulatief
    \item Samenwerking 
    \item Integraal 
    \item Ondersteund van de overheid.
\end{itemize}
\subsection{Globalisatie:}
Globalisering is het proces waardoor de wereld steeds meer met elkaar 
verbonden raakt als gevolg van de enorm toegenomen handel en culturele 
uitwisseling.
\begin{itemize}
    \item Globalisatie geresulteerd in:\begin{itemize}
        \item Toegenomen internationale handel
        \item Een bedrijf dat in meer dan 1 land actief is
        \item Grotere afhankelijkheid van de wereldeconomie
        \item Vrijer verkeer van kapitaal, goederen en diensten
        \item Erkenning van wereldwijde bedrijven.
    \end{itemize}
\end{itemize}
Hoewel globalisering waarschijnlijk bijdraagt aan het creeeren van meer welvaart in ontwikkelingslanden, helpt het niet om de kloof tussen de armste landen ter wereld en de rijkste ter wereld te dichten.
\subsection{GDPR:}
GDPR (of ook Algemene Verordening Gegevensbescherming – AVG 
genoemd) gaat over het beheer en de beveiliging van persoonlijke 
gegevens van Europese burgers. Elke organisatie moet u vanaf mei 2018 
kunnen aantonen welke persoonsgegevens deze verzamelt, hoe u deze 
data gebruikt en hoe u ze beveiligt.
\section{Bedrijfsorganisatie:}
\begin{enumerate}
    \item de manier waarop een bedrijf is georganiseerd in hierarchische lagen, functionele eenheden.
    \item Juridische organisatie van bedrijf: juridische vorm
\end{enumerate}
Bedrijfsorganisatie (1):
\begin{itemize}
    \item Wat moet er gebeuren?
    \item Hoe moet het gebeuren?
    \item Wie gaat het doen?
\end{itemize}
\subsection{Juridische ondernemingsvormen}
\begin{itemize}
    \item Profit organisaties:\begin{itemize}
        \item Elke bedrijfsentiteit, waarvan het primaire doel is om winst te genereren uit de reguliere activiteiten, met het oog op het maximaliseren van het vermogen van eigenaren, wordt een winstorganisatie genoemd.
    \end{itemize}
    \item Non-profit organisatie:\begin{itemize}
        \item Een non-profitorganisatie is een rechtspersoon die is georganiseerd en geexploiteerd voor een collectief, openbaar of sociaal voordeel, in tegenstelling tot een entiteit die opereert als een bedrijf dat winst wil genereren voor zijn eigenaren.
    \end{itemize}
\end{itemize}
\subsubsection{Eenmanszaak of vennootschap:}
\subsubsection{Eenmanszaak:}
\begin{itemize}
    \item Voordelen eenmanszaak:\begin{itemize}
        \item Eenvoudige ondernemingsvorm met meer ruimte voor eigen beslissingen 
        \item Vlotte oprichting en stopzetting
        \item Eenvoudige boekhouding
        \item Winst meteen beschikbaar
    \end{itemize}
    \item Nadelen eenmanszaak:\begin{itemize}
        \item Je bent zelf 100\% aansprakelijk
        \item beperkte mogelijkheden om fiscaliteit en sociale bijdragen te optimaliseren
        \item Beperkte mogelijkheden om opvolging en overname te regelen 
        \item De inkomsten van je onderneming zijn persoonlijke inkomsten, die belast worden aan de (hoge) progressieve tarieven  in de personenbelasting.
    \end{itemize}
\end{itemize}
\subsubsection{Cooperatieve vennootschap:}
\begin{itemize}
    \item Voordelen vennootschap:\begin{itemize}
        \item Bij de meeste vennootschapsvormen ben je als aandeelhouder beperkt aansprakelijk
        \item Meer mogelijkheden om fiscaliteit en sociale bijdragen te optimaliseren
        \item Meer mogelijkheden om opvolging en overname te regelen
        \item Je kunt de samenwerking met meerdere partners beter structureren
        \item De inkomsten van je vennootschap worden afzonderlijk belast in de vennootschapsbelasting en je kunt je
        persoonlijke inkomsten uit de vennootschap optimaliseren
    \end{itemize}
    \item Nadelen vennootschap:\begin{itemize}
        \item Complexere ondernemingsvorm met meer formaliteiten en procedures
        \item Oprichting en stopzetting onderworpen aan bepaalde regels
        \item Meestal een dubbele boekhouding
        \item Vennootschappen vereisen een startkapitaal of een toereik
    \end{itemize}
\end{itemize}
\subsection{Besloten vennootschap (BV):}
\begin{itemize}
    \item de bv is de aangewezen vennootschapsvorm voor kleine en (middel)grote ondernemingen.
    \item Je kunt met een bv ongeveer alles wat een NV kan en je privevermogen is afgeschermd.
\end{itemize}
\subsection{Naamloze vennootschap (NV):}
\begin{itemize}
    \item De NV is de aangewezen rechtsvorm voor grote en beursgenoteerde ondernemingen met een vereist startkapitaal van minimaal 61.500 euro.
\end{itemize}
\subsection{Cooperatieve vennootschap (CV):}
\begin{itemize}
    \item Voor ondernemingen waarvan de aandeelhouders samen een \underline{gemeenschappelijk doel} willen bereiken.
    \item De aandeelhouders kunnen burgers, organisaties, freelancers, kmo's of zelfstandigen zijn. De cv wordt ook gebruikt voor bedrijven waarvan de medewerkers mede-eigenaars worden van de vennootschap.
    \item er bestaan van de maatschap ook 2 varianten die wel rechtspersoonlijkheid bezitten: de vof en de CommV.
\end{itemize}
\subsection{Elementen voor praktische bedrijfsorganisatie:}
\underline{SiX key elements:}
\begin{enumerate}
    \item Werkspecialisatie:\begin{itemize}
        \item Differentiatie\begin{itemize}
            \item Voordeel:\begin{itemize}
                \item efficient gebruik werkkracht
                \item routine in werk
                \item automatiseren
            \end{itemize}
            \item nadeel:\begin{itemize}
                \item Coordinatieproblemen
                \item eentonig werk
                \item niet-flexibel personeel
            \end{itemize}
        \end{itemize}
        \item specialisatie:\begin{itemize}
            \item Voordeel:\begin{itemize}
                \item Betere coordinatie 
                \item Minder eentonig 
                \item Snelle communicatie en probleemoplossing
            \end{itemize}
            \item Nadeel:\begin{itemize}
                \item minder efficient (doublures)
                \item minder deskundigheid
                \item minder automatiseren
            \end{itemize}
        \end{itemize}
    \end{itemize}
    \item Indeling in departementen:\begin{itemize}
        \item Hergroeperen om werk \underline{gecordineerd} en \underline{geintegreerd} te laten lopen.
        \item[] \begin{itemize}
            \item[] 1) Functioneel:\begin{itemize}
                \item Schaalvoordelen binnen functionele afdelingen
                \item Grondige ontwikkeling van vaardigheden
                \item Functionele doelen worden bereikt 
                \item Het werkt het best in kleine tot middelgrote organisaties en met slechts 1 of enkele producten
            \end{itemize}
            \item[] 2) Product:\begin{itemize}
                \item Gegroepeerd rond gelijksoortige producten
            \end{itemize}
            \item[] 3) Klant:\begin{itemize}
                \item Gegroepeerd rond klantengroepen
            \end{itemize}
            \item[] 4) Geografisch:\begin{itemize}
                \item Gegroepeerd naar geografische afzetgebieden
            \end{itemize}
            \item[] 5) Proces:\begin{itemize}
                \item Gegropeerd op basis van workflow of klant-flow
            \end{itemize}
        \end{itemize}
    \end{itemize}
    \item Autoriteit en verantwoordelijkheid:\begin{itemize}
        \item \underline{Chain of command:} waar zijn authoriteit en verantwoordelijkheid in de organisatie?
        \item \underline{Underline:} rechten inherent aan positie
        \item \underline{Verantwoordelijkheid:} gedelegeerde authoriteit, verplichting om bepaalde taken uit te voeren
        \item[]\textbf{Lijn authoriteit:}\begin{itemize}
            \item Is een organisatie waarbij boven elke werknemer een manager of meerdere staat en waarin de taken opgedeeld zijn in logisch bij elkaar horende afdelingen. De lijnorganisatie is een oude en veel gebruikte organisatiestructuur. Als je een schema maakt van de organisatie lijkt ide op een piramide: smal van boven en breed uitlopend naar beneden. Bovenaan de piramide staat de directeur, het hoger management of topmanagement genoemd. opdracht directeur gaat via middenmanagement en lager management naar de werknemer toe. iedereen in organisatie heeft maar 1 baas, en het is duidelijk wie leiding geeft aan wie.\begin{itemize}
                \item Voordelen:\begin{itemize}
                    \item Systeem is duidelijk en eenvoudig. Het staat vast wie verantwoordelijk is voor wat.
                    \item Het is duidelijk wie de baas is, bevoegdheden zijn goed afgebakend.
                    \item Taken en verantwoordelijkheden zijn duidelijk bepaald
                    \item Beslissingen nemen gaat snel.
                    \item Problemen worden snel opgelost omdat iedereen weer bij wie hij moet zijn als die zich voordoen.
                \end{itemize}
                \item Nadelen:\begin{itemize}
                    \item Het systeem kan leiden tot bureaucratie, doordat de opdrachten via een lange weg naar de uitvoerders gaan, waarbij veel regels en papierwerk horen.
                    \item De afdelingen staan erg op zichzelf , als de afdelingen goed met elkaar moeten samenwerken is dit systeem minder geschikt.
                    \item Een manager van een bepaalde afdeling wordt geacht zich helemaal daar op te richten, waardoor er weinig ruimte is voor specialisatie.
                    \item Managers krijgen een te zware verantwoordelijkheid met te veel taken.
                    \item Er is weinig flexibliteit, omdat de gezagsverhoudingen strak vastliggen.
                \end{itemize}
            \end{itemize}
        \end{itemize}
        \item[]\textbf{STAF authoriteit:}\begin{itemize}
            \item De lijn-staforganisatie is een vorm van lijnorganisatie, maar naast de directie bevindt zich een \textbf{staf van deskundigen}. De staf bestaat uit mensen die zich gespecialiseerd hebben, bijvoorbeeld een accountant, maar verder buiten de organisatie staan. De staf staat de directie bij, door te adviseren en te ondersteunen. Net als bij de gewone lijnorganisatie heeft de lijn-staforganisatie met afdelingen. Elke leider of manager heeft een grotere afdeling onder zich dan waar hij zelf in zit. De staf geeft dus adviezen, maar kan ook bepaalde werkzaamheden overnemen, zoals de voorbereiding voor bepaalde werkzaamheden, voorlichting geven, controle verrichten en onderzoek en productontwikkeling realiseren.
        \end{itemize}
    \end{itemize}
    \item Controle omvang:\begin{itemize}
        \item Aantal werknemers een manager kan overzien\dots
        \item Omvang van controle groeit.
        \item[] afhankelijk van:\begin{itemize}
            \item Ervaring
            \item Complexiteit
            \item Gelijkaardigheid
            \item Fysieke aanwezigheid
            \item Standardized procedures
            \item \dots
        \end{itemize}
    \end{itemize}
    \item Centralizatie vs decentralizatie:\begin{itemize}
        \item waar worden de beslissingen genomen?
    \end{itemize}
    \item Formalizering:\begin{itemize}
        \item Formele organisatie:\begin{itemize}
            \item structuur
            \item regels
            \item procedures
            \item werkverdeling
            \item \dots
        \end{itemize}
        \item Informele organisatie:\begin{itemize}
            \item Coalitties
            \item Psychologische behoeften
            \item Macht 
            \item Informele leiderschap 
            \item \dots
        \end{itemize}
    \end{itemize}
\end{enumerate}
\subsubsection{Organizatiestructuren:}
\begin{itemize}
    \item Maxtrix structuur:\begin{itemize}
        \item De verschillende werknemers rapporteren aan meerdere personen. Medewerkers die hetzelfde soort werk doen zitten dan samen in één afdeling en rapporteren aan het hoofd van die afdeling, maar ze zijn ook aan verschillende projectteams toegewezen waardoor ze ook aan die projectleiders moeten rapporteren.
    \end{itemize}
    \item Team structuur:\begin{itemize}
        \item Ook wel de platte organisatiestructuur genoemd.
        \item Hierbij gaat het om samenwerkende teams.
        \item De functies op gelijkwaardig niveau worden gesplitst in afzonderlijke deeltaken.
        \item[] Werkt niet overal maar wordt wel vaak gebruikt in nieuwe IT bedrijven.
    \end{itemize}
    \item Vrije structuur:\begin{itemize}
        \item Geen horizontale, verticale noch grenzen gedefineerd
        \item Interne grenzen (Departementen, specialisatie)
        \item Externe grenzen (Klanten, leveranciers, stakeholders)
        \item Virtuele organizatie 
        \item Losse netwerkstructuur
    \end{itemize}
    \item Lerende structuur:\begin{itemize}
        \item Bedrijfscultuur
        \item Organizatie
        \item Informeren en delen van informatie
        \item Leadership
        \item Continue delen en opnemen van kennis
        \item Toepassen van deze kennis in de producten/diensten: oplossen van problemen door kennis
        \item Eerder mind-set dan organisatievorm
        \item[] Informatie!
        \item kennismanagement: open, accuraat en tijdig delen van informatie
    \end{itemize}
    \item  Hybride structuur:\begin{itemize}
        \item De divisie 'klantenservice' van Ford is een voorbeeld van een hybride structuur met functionele en horizontale structuur. Verschillende horizontaal geformeerde groepen, samengesteld uit multidisciplinaire teams, richten zich op kernprocessen. Elke groep heeft een proceseigenaar die ervoor verantwoordelijk is dat de teams aan de algehele doelstellingen voldoet. De divisie 'klantenservice' van Ford past een functionele structuur voor wat betreft financienm strategie en communicatie, en de afdeling personeel. Elk van deze afdelingen levert diesnten voor de hele divisie. 
    \end{itemize}
\end{itemize}
\section{Fundamentals of management:}
\subsection{Roos van Leary}
\subsection{Kernkwadranten van Ofman:}
\begin{itemize}
    \item Het model is erop gebaseerd dat iemands kernkwaliteit automatisch een uitdaging in zich meedraagt, evenals een valkuil en een allergie. (bvb. als je goed en veel kan spreken kan het zijn dat je andere mensen niet echt aan het woord laat. Dit kan tot gevolg hebben dat je niet goed kan luisteren naar anderen.)
\end{itemize}
\subsection{DISC:}
\begin{itemize}
    \item 4 kleuren
    \item Elke kleur staat voor een bepaald gedrag
    \item We hebben van elke kleur iets!
    \item Bepaalde kleuren zijn dominanter dan andere
    \item Gedrag is gebasseerd op gewoonte
    \item We kunnen gedrag ontwikkelen 
    \item Je kan vanuit elke kleur succesvol zijn
    \item Je toont verschillend gedrag in verschillende omgevingen 
    \item Je hebt zelden van elk kleur alles\begin{itemize}
        \item[]Horizontaal verschil:
        \item Indirect:\begin{itemize}
            \item Onderbroken oogcontact
            \item Praat trager en stiller
            \item Wacht om zijn mening te geven
            \item Stopwoorden
            \item Langere pauzes
            \item voorzichtige handdruk
            \item Volg een proces om te beslissen
        \end{itemize}
        \item Direct\begin{itemize}
            \item Direct oogcontact
            \item Praat luider en sneller
            \item Geeft gemakkelijk zijn mening 
            \item weinig stopwoorden 
            \item Houdt niet van stiltes 
            \item Sterke handdruk 
            \item Beslist gemakkelijk
        \end{itemize}
        \item[]Vertikaal verschil:\begin{itemize}
            \item Taak:\begin{itemize}
                \item Gericht op wat mensen doen
                \item Kijken naar feiten en resultaten 
                \item Focus op controle 
                \item Tonen minder emotie 
                \item Vlakkere stem 
                \item Hoekige of geen gebaren
            \end{itemize}
            \item Mens:\begin{itemize}
                \item Gericht op wie mensen zijn
                \item Gericht op voelen 
                \item Relationeel 
                \item Meer emotie
                \item Meer intonatie in de stem
                \item Ronde gebaren
            \end{itemize}
        \end{itemize}
        De 4 kleuren zijn:\begin{itemize}
            \item Compliance:\begin{itemize}
                \item[]\underline{Teamwaarde}:
                    \item Consciëntieus
                    \item Gelijkgestemd
                    \item Perspectief: het anker van de realiteit 
                    \item Grondig in alle activiteiten 
                    \item Definieert situaties 
                    \item Verzamelt, kritiseert en test informatie
            \end{itemize}
            \item Dominance:\begin{itemize}
                \item[]\underline{Teamwaarde}:
                \item Resultaat gerichte organisator 
                \item Hecht veel belang aan tijd en gebruik ervan 
                \item Daagt de status quo uit
                \item Innovatief
            \end{itemize}
            \item Steadiness:\begin{itemize}
                \item[]\underline{Teamwaarde}:
                    \item betrouwbaar
                    \item Je kan er op rekenen
                    \item Loyale teamspeler 
                    \item Volgt autoriteit 
                    \item Goede luisteraar 
                    \item Geduldig en empatisch 
                    \item Goed in het verzoenen van conflicten
                \end{itemize}
            \end{itemize}
            \item Influence:\begin{itemize}
                \item Creatieve probleemoplosser 
                \item Moedigt anderen aan 
                \item Motiveert anderen om zaken te bereiken 
                \item Positieve zin voor humor 
                \item Onderhandelt bij conflicten 
                \item Vredestichter
            \end{itemize}
        \end{itemize}
    \end{itemize}
\section{Werken in team:}
Kennismaken is weten van elk teamlid:\begin{itemize}
    \item de ervaringen
    \item de leerstijl
    \item de teamrol
    \item de communicatiestijl
\end{itemize}
\subsection{Leerstijlen:}
\begin{itemize}
    \item Doener: accomoderen (in voor alles)
    \item Dromer: divergeren (bezinnen)
    \item Denker: assimileren (relaties leggen)
    \item Beslisser: convergeren (toepassen)
\end{itemize}
\subsubsection{Belbin teamrol management:}
Studoe over de effectiviteit van management teams:\begin{itemize}
    \item Bijdrage aan het teamresultaat:\begin{itemize}
        \item Professionele rol: deskundigheid op vakgebied
        \item Organisatorische rol
        \item Persoonlijkheid of teamrol
    \end{itemize}
    \item evenwicht tussen teamrollen 
    \item Natuurlijke teamrol, temperament
\end{itemize}
\subsubsection{Verwachtingen:}
\begin{itemize}
    \item Wat verwacht je van het team?
    \item Wat verwacht het team van jou?
\end{itemize}
Samenwerkings afspraken:\begin{itemize}
    \item Evaluatie
    \item Rapportering 
    \item Respect 
    \item Omgaan met irritaties 
    \item Op tijd komen/stoppen 
    \item Feedback 
    \item Iedereen neemt deel 
    \item Communicatie 
    \item Agenda
    \item \dots
\end{itemize}
CPBW $\rightarrow$ overleggen met een afvaardiging aan het personeel \underline{Paritair} (gelijke afvaardiging) samengesteld.\begin{itemize}
    \item Personeel (vakbonden)
    \item werkgever
\end{itemize}\begin{itemize}
    \item GPP: GLobaal Preventie Plan:\begin{itemize}
        \item Elke 5 jaar 
    \end{itemize}
    \item JAP: Jaar Actie Plan:\begin{itemize}
        \item Elk jaar
    \end{itemize}
\end{itemize}
\subsubsection{Feedback:}
Feedback $\rightarrow$ effectieve feedback:\begin{itemize}
    \item Feed-up (wat je in de toekomst zal doen):\begin{itemize}
        \item Waar ga ik heen?
        \item Wat zijn mijn doelen?
    \end{itemize}
    \item Feedback (wat je nu hebt gedaan):\begin{itemize}
        \item Hoe sta ik ervoor?
        \item Welke voorruitgang is er op weg naar het doel?
    \end{itemize}
    \item Feedforward:\begin{itemize}
        \item Wat is de volgende stap?
        \item Wat kan ik vervolgens het beste doen om vooruitgang te boeken?
    \end{itemize}
\end{itemize}
\paragraph{Effectieve feedback:}\begin{itemize}
    \item Gever en ontvanger moeten elkaar vertrouwen
    \item Beide moeten feedback als belangrijk hulpmiddel zien om de communicatie te verbeteren.
    \item Beide zijn bereid van elkaar te leren 
    \item Kort op de bal 
    \item Sandwichmethode
\end{itemize}
\subsubsection{Review}\begin{itemize}
    \item Ervaringen 
    \item Leerstijl 
    \item Teamrol 
    \item Communicatiestijl 
    \item Verwachtingen 
    \item Samenwerkingsafspraken 
    \item Feedback 
\end{itemize}
\subsection{Overleggen:}
\subsubsection{Meeting:}
Succesvol overleggen:
\begin{itemize}
    \item voorbereiding
    \item uitnodiging
    \item constructief vergaderen 
    \item besluiten 
    \item afspraken vastleggen 
    \item valkuilen
\end{itemize}
\paragraph{Voorbereiding:}
Doelen bepalen:\begin{itemize}
    \item Uitwisselen van informatie 
    \item Meningen vormen of uitwisselen 
    \item Problemen oplossen 
    \item Besluiten nemen
\end{itemize}
Zo specifiek mogelijk!
\begin{itemize}
    \item De voorzitter: inhoudelijk en organisatorische voorbereiding (ken de deelnemers).
    \item Deelnemers: agenda doornemen en relevante stukken doornemen
    \item Secretaris: verslag opmaken, afspreken hoe!
\end{itemize}
\paragraph{uitnodiging:}
\begin{itemize}
    \item Tijd,plaats 
    \item Agenda 
    \item Formele en informele vergaderingen
\end{itemize}
Aanwezigheid van deelnemers , vorm van uitnodiging
\paragraph{Constructief vergaderen:}
\begin{itemize}
    \item Wie heeft de leiding?
    \item Luisteren\begin{itemize}
        \item actief
        \item LSD (Luisteren, Samenvatten, Doorvragen)
        \item niet onderbreken
        \item WIER luistermethode (Waarnemen, Interpreteren, Evalueren, Reageren)\begin{itemize}
            \item Waarnemen:\begin{itemize}
                \item Luisteren gaat niet alleen over horen, maar je gebruikt tijdens het luisteren meer zintuigen. Om beter te kunnen luisteren, probeer je beter waar te nemen.
            \end{itemize}
            \item Interpreteren:\begin{itemize}
                \item De meeste luisterfouten ontstaan op het niveau van interpretatie. Mensen zullen altijd proberen te begrijpen wat ze horen, maar die interpretaties kloppen lang niet altijd. Dus je leert beter luisteren door beter te interpreteren.
            \end{itemize}
            \item Evalueren:\begin{itemize}
                \item Nadat de spreker klaar is met zijn verhaal is het belangrijk om te gaan evalueren.
            \end{itemize}
            \item Reageren:\begin{itemize}
                \item het luisterproces is pas compleet na een reactie. (Ook geen reactie is een reactie!) Hoe effectiever je reageert hoe meer kans op een geslaagde communicatie.
            \end{itemize}
        \end{itemize}
        \item Spreken:\begin{itemize}
            \item woord nemen
            \item Voorzitter
        \end{itemize}
    \end{itemize}
\end{itemize}
\subparagraph{Vragen stellen:}\begin{itemize}
    \item Doen!
    \item open/gesloten vragen:\begin{itemize}
        \item Open vragen: je kan redelijk veel/breed antwoorden op deze vragen
        \item Gesloten vragen: nie zoveel verschillende antwoorden mogelijk bvb ja nee vragen ofzo.
    \end{itemize}
    \item Retorische vragen: De vraag is eigelijk al beantwoord (het antwoord is duidelijk).
    \item Slechte vragen
\end{itemize}
\subparagraph{Non-verbale communicatie:}\begin{itemize}
    \item Gebaren 
    \item Gezichtsexpressie 
    \item Lach 
    \item Lichaamshouding
\end{itemize}
Onbewuste signalen wijzen op emotie.
\paragraph{Besluiten:}
Besluiten nemen:\begin{itemize}
    \item Mening vormen 
    \item Stemmen 
    \item Consensus/compromis\begin{itemize}
        \item Consensus: iedereen is er mee eenms 
        \item Compromis: niet iedereen gaat volledig akkoord dus je moet bijwerkingen maken tot iedereen content is.
    \end{itemize}
\end{itemize}
\paragraph{Afspraken vastleggen:}
\begin{itemize}
    \item Verslaggeving 
    \item Archief 
    \item Online
\end{itemize}
\paragraph{Pitfalls:}
\begin{itemize}
    \item Teamrollen goed vastleggen 
    \item Communicatiestijlen 
    \item Afwijkende doelen en verwachtingen 
    \item Feedback ontaard in verwijten 
    \item Partijdige voorzitter 
    \item Taakgericht handelen 
    \item Goede sfeer... 
    \item Interculturele verschillen
\end{itemize}
\section{Kwaliteit, veiligheid en milieu}
Quality, Health, Safety and environment (QHSE) is gebaseerd op volgende aspecten:\begin{itemize}
    \item Voldoenaan de geldende regelgeving (compliance)
    \item Integratie van:\begin{itemize}
        \item Kwaliteit
        \item Gezondheid 
        \item Veiligheid 
        \item Omgeving
    \end{itemize}
\end{itemize}
Doel: Het bedrijf of de organisatie ondersteunen om een kwaliteits- en veiligheidscultuur uit te werken.\newline
\textbf{Safety:}
Het is belangrijk dat een organisatie niet alleen een soort papieren veiligheidsbeleid heeft, maar dat veiligheid van personeel en betrokken buitenstaanders ook in de praktijk wordt gegarandeerd, dat zij zich bewust zijn van mogelijke gevaren op de werkplek en hoe deze
voorkomen kunnen worden. Papieren veiligheidsbeleid $\rightarrow$ cultuur?\begin{itemize}
    \item Papier:\begin{itemize}
        \item procedures 
        \item wetgeving
    \end{itemize}
    \item Cultuur:\begin{itemize}
        \item Draagvlak $\nearrow$ $\rightarrow$ € (Cao 90$\nearrow$ $\rightarrow$ ongevallen)
        \item Uitvoeren van de afspraken 
        \item meedenken
    \end{itemize}
\end{itemize}
\subsection{De kwaliteitsverantwoordelijke:}
In de bril van de kwaliteitsverantwoordelijke kan het altijd beter. De functie van kwaliteitsverantwoordelijke heeft algemeen coordinerende, leidinggevende, adviserende, toezichthoudende en vertegenwoordigende kanten. Coordineert de uitvoering van werkzaamheden en communicatie binnen de afdeling kwaliteit.\newline
Kwaliteit garanderen door het volgen van procedures en controles op de effectiviteit
ervan.
\subsection{Veiligheid:}
\subsubsection{de preventieadviseur:}
\textbf{Welzijnsbeleid:}Een beleid dat erop gericht is om elk risico dat aanleiding kan
geven tot aantasting van de veiligheid of de gezondheid van het personeel of
derden, te voorkomen, de risico’s vroegtijdig te detecteren, en als het risico zich
toch voordoet, het te beëindigen of de schade te beperken.\newline
\textbf{7 domeinen:}\begin{itemize}
    \item arbeidsveiligheid
    \item bescherming van de gezondheid van de werknemer op het werk.\begin{itemize}
        \item PBM: persoonlijke berschermingsmiddelen:\begin{itemize}
            \item gehoor 
            \item zicht
        \end{itemize}
        \item CBM: Collectieve berschermingsmiddelen:\begin{itemize}
            \item Er voor zorgen dat mensen niet bij het gevaar kunnen geraken.
        \end{itemize}
    \end{itemize}
    \item psychosociale belasting, veroorzaakt door het werk, waaronder geweld, pesterijen en ongewenst seksueel gedrag op het werk.
    \item ergonomie
    \item arbeidshygiëne
    \item verfraaiing van de arbeidsplaatsen
    \item maatregelen op het vlak van leefmilieu, wat
    betreft hun invloed op de vorige punten
\end{itemize}
Risico analyse:\begin{itemize}
    \item Risicio's in kaart brengen $\rightarrow$ + voorkomen dat er schade is.
\end{itemize}
Elke 5 jaar moet je verplicht een enquête laten uitvoeren.\begin{itemize}
    \item Interne dienst: preventie \& bescherming op het werk (eigen aan het bedrijf)\begin{itemize}
        \item Preventie adviseur is hier deel van
    \end{itemize}
    \item Externe dienst:\begin{itemize}
        \item liandis
        \item IDEWE $\rightarrow$ arbeidsarts/geneesheer
    \end{itemize}
\end{itemize}
Samen dekken ze alle domeinen af.\newline
Een preventieadviseur staat de werkgever bij inzake de toepassing van de maatregelen
bedoeld in de welzijnswet. Hij heeft dan ook een \textbf{adviserende functie} ten opzichte van de werkgever en de werknemers. In het organigram van de onderneming is zijn functie bijgevolg een staffunctie.Preventieadviseurs vervullen hun opdracht in \textbf{volledige onafhankelijkheid} ten opzichte van de werkgever en de werknemers. Zij mogen geen nadeel ondervinden van hun activiteiten als preventieadviseur.\newline
Een directeur kan geen preventieadviseur zijn, je moet als preventie adviseur onafhankelijk zijn van het bedrijf. Enkel bij kleine bedrijven kunnen daar uitzonderingen voor gemaakt worden.
\begin{itemize}
    \item De opleiding van de preventie adviseur:\begin{itemize}
        \item Groep A (meer dan 200 werknemers - bouwsector): preventieadviseur: niveau 1
        \item Groep B (50 tot 199 werknemers - bouwsector): preventieadvisuer: niveau 2
        \item groep C en D (1 tot 49 werknemers - bouwsector): basiskennis, vorming niet
        verplicht, maar wel aanbevolen
        \item groep D (minder dan 20 werknemers - bouwsector): de preventieadviseur kan de
        werkgever zijn, als hij over de nodige basiskennis beschikt
    \end{itemize}
\end{itemize}
De preventie adviseur moet jaarlijks een verslag maken over het bedrijf.
\subsubsection{De vertrouwenspersoon:}
De vertrouwenspersonen zijn een laagdrempelig eerste aanspreekpunt voor medewerkers voor
alle vragen rond psychosociale risico’s zoals stress, burn-out, ongewenst gedrag, ...
Elke medewerker met moeilijkheden rond welzijn op het werk kan op een professionele, kwaliteitsvolle manier bij een vertrouwenspersoon terecht. Belangrijk hierbij is dat de vertrouwenspersoon gebonden is door het beroepsgeheim.
\begin{itemize}
    \item Als je zelf de baas bent (zelfstandige) en je hebt geen werknemers buiten uzelf kan je deze regels wat buigen omdat je technisch gezien geen werknemers hebt. In de bouw zie je dit vaak omdat ze zelfstandigen aannemen om te werken in onderaaneming etc.
\end{itemize}
\subsection{Milieu:}
\subsubsection{de milieucoördinator}
Wat doet een milieucoördinator?\begin{itemize}
    \item[] (Een aantal taken van de milieucoördinator worden opgesomd in artikel 4.1.9.1.3 van VLAREM II)
    \item Bijdragen tot de ontwikkeling, de invoering, de toepassing en de evaluatie van milieuvriendelijke productiemethoden en producten.
    \item Waken over de naleving van de milieuwetgeving door regelmatig controles uit te voeren op de werkplaatsen. 
    \item Waken over of instaan voor de uitvoering van de voorgeschreven emissie- en immissiemetingen en de registratie van de resultaten.
    \item Waken over het bijhouden van een afvalstoffenregister en naleven van de meldingsplicht over afvalstoffen.
    \item Intern en extern communiceren in verband met (het beperken van) de gevolgen van de activiteit van een bedrijf voor mens en milieu.
    \item Adviseren in verband met milieugerelateerde investeringen.
    \item Opmaken van een jaarlijks verslag over zijn activiteiten en uitgebrachte adviezen in het voorbije jaar.
    \item[] Je kan op deze website zien voor wat welke regels/wetten gelden:
    \url{https://vlaremwegwijzer.navigator.emis.vito.be}
\end{itemize}
De milieucoördinator moet jaarlijks een verslag maken over het bedrijf.
\paragraph{Wat zijn beste bechikbare technieken (BBT)?}
Beste beschikbare technieken zijn technieken die, in vergelijking met alle gelijkaardige technieken, het best scoren op milieugebied én betaalbaar zijn én technisch uitvoerbaar zijn.\newline
\url{https://omgeving.vlaanderen.be/nl/beste-beschikbare-technieken-bbt}
\subsection{Stappenplan:}
\begin{itemize}
    \item Idee 
    \item Doelstellingen formuleren:\begin{itemize}
        \item Wetgeving aftoetsten (milieu en omgeving)
        \item Bepalen KPI's of ENPI 
        \item Bestaat er al een nieuwe technologie? (BBT)
    \end{itemize}
    \item Subsidie mogelijk? $\rightarrow$ Vlaio 
    \item Opstellen businessplan\begin{itemize}
        \item Wat levert dit op en wat zal dit besparen?
    \end{itemize}
    \item Van idee naar project:\begin{itemize}
        \item 'Intern verkopen project' $\rightarrow$ Wie?
        \item Impact organisatie (manier van werken, ruimte, veiligheid, kwaliteit,\dots)
        \item opstellen tijdslijn met milestones 
        \item Nagaan via risico-analyse hoe het werk het best kan verlopen 
    \end{itemize}
\end{itemize}
\subsubsection{Oefening:}
Agristo:\begin{itemize}
    \item start:\begin{itemize}
        \item Projectteam:\begin{itemize}
            \item Beginsituatie:\begin{itemize}
                \item[]desk:
                \item google maps 
                \item duurzaamheidsrapport 
                \item stadsbladmonitor (financiele kant) 
                \item website 
                \item Omgevingsvergunning?
                \item Gesprek
            \end{itemize}
            \item Scope:(1 locatie/afdeling/land)
            \item Audit (bestaande situatie te beschrijven)
            \item Rapport $\rightarrow$ voorstellen
        \end{itemize}
    \end{itemize}
    \item swot:\begin{itemize}
        \item idee
    \end{itemize}
\end{itemize}
\paragraph{Beginsituatie:}\begin{itemize}
    \item $\hookrightarrow$ Voeding:\begin{itemize}
        \item via PV (zonnepanelen)
        \item Stoom via restwarmte (gedeeltelijk)
    \end{itemize}
    \item $\hookrightarrow$ eigen warmte via WKK:\begin{itemize}
        \item gas verstoken $\rightarrow$ warmte $\rightarrow$ warm water
        \item Elektriciteit
    \end{itemize}
    \item $\hookrightarrow$ afvalwaterreiniging 
    \item $\hookrightarrow$ Transport volledig over de weg
\end{itemize}
Gesprek?\begin{itemize}
    \item Info:\begin{itemize}
        \item Rapporten 
        \item Metingen 
        \item ervaringen
        \item vergunningen
    \end{itemize}
    \item Doel:\begin{itemize}
        \item scope bepalen
    \end{itemize}
\end{itemize}
\paragraph{Audit:}
Stap 1:\begin{itemize}
    \item Technologie? 
    \item Ruimtegebruik?
    \item Gebouwen?
    \item Meters? (\#,type,submeters?,meterplan)
    \item Plannen? (toekomst,)
\end{itemize}
Stap 2: Analysefase:\begin{itemize}
    \item Verbruik (Energie prestatie indicatoren (EnPI) zoals bvb:KPI's)
\end{itemize}
\paragraph{Swot:}
Voor je de Swot doet zorg je dat je een bezoek aan het bedrijf hebt gedaan.
\begin{itemize}
    \item strengths:\begin{itemize}
        \item aandacht duurzaamheid 
        \item Er zijn \# meters 
        \item gebruikt reststroom,reststoom 
        \item PV 
        \item Watervoorziening
    \end{itemize}
    \item weaknesses:\begin{itemize}
        \item locatie tov klanten,grondstoffen
        \item veel transport
        \item uitbreidingsmogelijkheden voor site' moeilijk in toekomst(niet website maar fysieke site)
    \end{itemize}
    \item opportunities:\begin{itemize}
        \item ijsbuffer,diepvries
        \item PV bijplaatsen
        \item biogas
    \end{itemize}
    \item threats:\begin{itemize}
        \item Prijsevolutie: Elektriciteit,gas,water
        \item Politieke keuzes $\rightarrow$?
        \item Veranderende wetgeving
    \end{itemize}
\end{itemize}
\paragraph{confrontatie:}
\begin{itemize}
    \item (S-O) vanuit data opslag,productie optimaliseren met input op de duurzaamheid 
    \item (O-T) Door eigen productie afhankelijkheid aankoop beperken
    \item (T-S) Extra productie site in Nederland,Frankrijk of Engeland samen met bedrijf die dingen ons kan leveren.
\end{itemize}
We gaan ons verder specifieren op de eerste van de drie (met data).
Doelstelling:(SMART)\newline
Concreet doel:\begin{itemize}
    \item $\hookrightarrow$ Volgend jaar willen we 3\% minder energie verbruiken /kg product.
\end{itemize}
Hoe:\begin{itemize}
    \item beslissen op basis van meters
\end{itemize}
Wat nodig:\begin{itemize}
    \item Aanstuurbare processen?
    \item Porgramma $\rightarrow$ € /uren 
    \item 15k € investeringskosten
\end{itemize}
opbrengst:\begin{itemize}
    \item Stel 2\% minder energie:\begin{itemize}
        \item $\hookrightarrow$ ROI=kostprijs/opbrengst=15/5= 3 jaar.
    \end{itemize}
\end{itemize}
\paragraph{Tijdlijn:}
\begin{itemize}
    \item eerst info vragen 
    \item bezoek bedrijf en gesprek met iemad van het bedrijf.
    \item voorstel project 
    \item intern wordt overlegd om te bepalen wat ze doen (kan enkele tijd duren)
    \item indien goedgekeurd $\rightarrow$ implementatie 
    \item operationeel.
\end{itemize}
\section{Examen:}
\begin{itemize}
    \item Linken kunnen leggen tussen sommige aspecten.
    \item opsomming zal niet echt gevraagd worden 
    \item eerder link en verband tussen andere dingen kunnen leggen.
    \item zou ongeveer 2 uur moeten duren examen.
    \item (volgend jaar wss mondeling).
\end{itemize}
\end{document}
