
\documentclass[12pt]{article}

\usepackage{amsmath}

\usepackage{microtype}

\usepackage{graphicx}

\usepackage{hyperref}

\usepackage[utf8]{inputenc}

\usepackage{listings}

\usepackage{matlab-prettifier}
% % voor code syntax highlighting
% \usepackage{minted}

% beter font
\usepackage[T1]{fontenc}
\usepackage{helvet}
\renewcommand{\familydefault}{\sfdefault}
\graphicspath{{imagessoft/}}
\begin{document}
\begin{titlepage}
    \author{Xander Vandooren}
    \title{Industry Road Map 4.0}
\end{titlepage}
\pagenumbering{gobble}
\maketitle
\newpage
\tableofcontents
\newpage
\pagenumbering{arabic}
\section{Node-Red}
Om node-red dashboard op te starten kan je in de terminal de command "node-red" doen.
\begin{itemize}
    \item[]\begin{itemize}\textbf{voordelen:}
    \item rapid application development
    \item powerful and flexibiliteit
    \item flow-based programming $\rightarrow$ Messages representing events between nodes
    \end{itemize}
    \item[]\begin{itemize}\textbf{nadelen:}
        \item complex mutli-function $\rightarrow$ complex to visualize
        \item flow-based programming is cumbersome.
        \end{itemize}
\end{itemize}
\subsubsection{Json Node}
It allows you to convert between a Json-formatted string and a javascript object, making it highly versatile for tasks involving JSON manipulation.
\subsection{environment variables:}
\begin{itemize}
    \item Node variables
    \item Flow variables
    \item Global variables
    \item Environment variables $\Rightarrow$ store confidential configuration information such as secret and API keys, preventing this data from being accidentally exposed in the flow.
\end{itemize}
\paragraph{Global variables:}
Globale variables in Node-red are accessible  to functions,changes,inject
\subsubsection{Environment variables:}
\section{WEB-API}
A web-API (application programming interface) is a set ofrles and protocols that allows different software applications 
\subsection{Endpoints:}
\subsection{Security:}
\section{Sensoren:}
\subsection{Sensoren en actuatoren:}
\begin{itemize}
    \item Sensor:\begin{itemize}
        \item[]Een element dat 1 of meer ingangsgrootheden en de verandering(en) daarvan omzet in bruikbare elektrische,hydraulische,pneumatische, optische en andere signalen voor verdere verwerking.
        \item Passieve sensor: De nodige conversie energie van de sensor is afkomstif van het te meten proces, de grootheid of materiaal. Potentiele veinvloeding van de te meten groothheid.
        \item Actieve sensoren: sturen hun eigen signaal naar een object of een gebied en ontvangen het gereflecteerde signaal. Conversie energie is afkomstig van de voeding van de actieve sensor.
    \end{itemize}
    \item Actuator:\begin{itemize}
        \item Een toestel dat invloed kan uitoefenen op zijn omgeving.
    \end{itemize}
\end{itemize}
\subsubsection{Indeling sensoren:}
\begin{itemize}
    \item De sensor ondergaat de fysische grootheid die in het proces gemeten moet worden.
    \item Rechstreekste omzetting naar elektrische grootheden\begin{itemize}
        \item omzetting naar een weerstandswaarde: PTC,NTC,LDR,\dots
        \item omzetting naar spanning: thermokoppel, zonnecel, accelero
        \item Omzetting naar een verplaatsing\begin{itemize}
            \item Rekstrookje (R wijzigt)
            \item uitzetten vloeistof
            \item membraam
        \end{itemize}
        \item \dots
    \end{itemize}
\end{itemize}
\subsection{Signaalconditionering:}
Fysische grootheid $\Rightarrow$ Sensor $\Rightarrow$ SC: Signaal-conditionering $\Rightarrow$ ADC: analoog Digitaal conversie $\Rightarrow$ computer
\subsubsection{Meetomvormer:}
\begin{itemize}
    \item De gemeten elektrische grootheden of effecten worden in de meetomvormer omgezet en/of verstrekt naar een standaardsignaal, dat dient als ingang voor de regelaar of controller
\end{itemize}
\subsubsection{filtering:}
\begin{itemize}
    \item Sensor signaal kan naast het gewenste signaal ook ongewenste componenten zoals ruis of andere stoorsignalen van velerlei soort bevatten
    \item Niet de groote van het signaal zelf is bepald voor de kwaliteit van het signaal, maar de verhouding van signaal en ruis (S/N).
    \item Bij lage S/N is analoge filtering aangewezen.
    \item Filters zijn gericht op bepaalde signaalcomponenten (bv. high-pass, low-pass, anti-aliasing).
\end{itemize}
\subsubsection{Multiplexing:}
\begin{itemize}
    \item Met multiplexer kunnen de uitgangssignalen
\end{itemize}
\subsubsection{sample \& hold:}
\begin{itemize}
    \item Digitale verwerking van een analoog signaal vereist bemonstering (sampling) van dat signaal: bepaling van de signaalwaarde op discrete tijdstippen.
    \item Op de daartussen liggende momenten ontbreekt informatie over de signaalwaarde.
    \item Dit is minder dramatisch dan het lijkt, mits 'voldoende' monsters (samples) per tijdseenheid worden genomen (Nyquistcriterium).
    \item[] In principe in de bemonstertijd zeer kort (signaalwaarde op discrete tijdstippen). De verwerking van de bemonsterde waarde kan een langere tijd vergen. In zulke gevallen wordt de bemonsterde waarde enige tijd vastgehouden. Vaak zijn beide functies verenigd in 1 schakeling (sample and hold).
\end{itemize}
\subsection{Spanningsignalen:}
\begin{itemize}
    \item 0-10V: Industriele standaard
    \item 0-5V: Hobby en kleine elektronica (5V voeding)
\end{itemize}
\subsubsection{Spanningsval over geleiders:}
Nadelen:
\begin{itemize}
    \item excitatiespanning nodig.
    \item Spanningsval is vergelijkbaar met uitgangssignalen.
\end{itemize}
\underline{Alternatief is stroomlus voor spanning of weerstandsignaal:}
\begin{itemize}
    \item Sinds de jaren 1950 wordt de 4-20mA stroomlus gebruikt om sensor data te transporteren voor monitoring en control.
    \item Lage gevoeligheid voor ruis.
    \item Geschikt voor lange afstanden
    \item Industrieel veel gebruikt.
\end{itemize}
Wel moeilijker te meten dan spanning.
\subsection{Powermeter/Wattmeter:}
\begin{itemize}
    \item Electrodynamic instrument
    \item Fixed /current coils $\Rightarrow$ parallel 
    \item $P=U*I$
    \item De interactie tussen de magnetische velden van de stroom en de spannings spoel zorgt ervoor dat de magnetische spoel movable zal draaien
    \item 
\end{itemize}
\subsubsection{Elektromagnetische Kilowattuurmeter:}
\begin{itemize}
    \item constructie
    \item 
\end{itemize}
\subsection{Vermogenmeter versus energiemeter:}
Energiemeter:
\begin{itemize}
    \item Meet de totale hoeveelheid elektrische energie die wordt verbruikt over een bepaalde periode.
    \item De eenheid van meting is Kilowattuur (kWh).
    \item Wordt vaak gebruikt om het totale verbruik van een huishoden of bedrijf te meten.
\end{itemize}
Wattmeter:
\begin{itemize}
    \item Meet het directe vermogen dat op een bepaald moment wordt verbruikt.
    \item De eenheid van meting is watt (W).
    \item Wordt gebruikt om het momentane verbruik
\end{itemize}
\subsubsection{Watt omzetten naar kWh:}
\begin{itemize}
    \item Vermogen omzetten naar energie $E=Pxt$\begin{itemize}
        \item E: energie in Joules (J)
        \item P: vermogen in power watts (W)
        \item t: tijd in seconden (1 uur= 3600 seconden)
    \end{itemize}
    \item[] %verder aan vullen
\end{itemize}
\paragraph{oefening:}
\begin{itemize}
    \item Maandkost: 1500W $\Rightarrow$ 0.75 kWh per dag. *30*0.35=7.88 euro per maand (we gaan er van uit dat een maand 30 dagen is).
    \item Joules per maand: 0.75kWh*30=22.5kWh per maand $\Rightarrow$ 22500 *3600=81MJ
\end{itemize}
\subsection{Energiemeters voor submetingen:}
\begin{itemize}
    \item Elektriciteitsmeters om een deel van een gebouw of installatie te meten.
    \item Ondersteunen soms protocollen (bv Modbus).
    \item Soms pulstellers.
    \item Dikwijls lokale aflezing mogelijk.
\end{itemize}
\section{Modbus:}
Modbus is een populair protocol, vooral gekend voor zijn robuustheid. Het is gemakkelijk te implementeren en flexibel.
\subsection{Protocol}
Een communicatie protocol is een set van regels die er voor zorgen dat 2 of meer apparaten in een communicatie systeem informatie kunnen uitwisselen.
\begin{itemize}
    \item Syntax: bepaald de structuur of het formaat van de data
    \item Semantics: Specifieerd de betekenis van elke sectie bits.
    \item Synchronization: Regeldde timing van de data uitwisseling.
    \item Error detection and correction: zorgt voor data integriteit tijdens het versturen.
\end{itemize}
Communicatie protocollen zijn belangrijk in verschillende velden: telecommunicatie,computer netwerken en industriele automatisatie.
\subsection{Modbus RTU:}
Modbus RTU (Remote Terminal Unit:)\begin{itemize}
    \item Communicatie protocol
    \item Frame Structure
    \item Master-Slave architecture (meest gebruikten)
    \item Binary representation.
\end{itemize}
\subsection{Modbus network topology:}
\begin{itemize}
    \item 2 wired bus
    \item Terminating resistors
    \item Multi-drop configuratie is meest voorkomende: Waar meerdere slave devices verbonden zijn met 1 master device op een single communicatie lijn.
\end{itemize}
\paragraph{Network topology}
\begin{itemize}
    \item Point to point: directe connectie tussen 2 devices
    \item Multidrop: 1 master device communiceerd met meerdere slave devices op dezelfde bus.
    \item Multipoint: Meerdere devices communiceren met elkaar op hetzelfde netwerk
    \item Half duplex: communicatie kan gebeuren in 2 richtingen maar niet op hetzelfde moment (modbus is half duplex)
    \item Full duplex: Communicatie kan gebeuren in 2 richtingen op hetzelfde moment.
\end{itemize}
\paragraph{Network interfaces:}
De RS485 interface is heel populair om te gebruiken met het modbus protocol. De RS485 interface gebruikt differentiele signalen en twisted kabels, dit laat de RS485 toe om te communiceren over lange afstanden met een hoge bit rate.
\subparagraph{Network interface settings:}
Master and slave(s) moeten dezelfde settings hebben:
\begin{itemize}
    \item Baud rate: snelehied van data transmissie (gemeten in bits)
    \item Data bits: het aantal bits in elke byte of data
    \item Parity: Een methode voor error checking. Deze opties bevatten None,Even,Odd,Mark of Space.
    \item Stop bits: de nummer van bits gebruikt om het einde van de byte te signaleren. meeste values zijn 1 of 2 bits.
    \item Flow control: Managed data flow om buffer overflow te preventeren.
\end{itemize}
\subsection{Modbus communicatie protocol}
Elke modbus message heeft dezelfde structuur. vier basis elementen zijn present in elke message.\begin{itemize}
    \item Device address
    \item Function Code: definieerd message type en actie gevraagd aan de slave.
    \item Data:bevat de informatie die verzonden moet worden tussen devices. grootte varieerd tot 252 bytes afhankelijk van de function code en de hoeveelheid data gerelateerd aan de operatie.
    \item Error check (CRC Cyclic Redundancy Check) field: 2 byte CRC field wordt gebruikt voor error detectie.
\end{itemize}
\section{Netwerken:}
\subsection{IP address:}
Een unieke numerical identifier assigned to every device that connects
\begin{itemize}
    \item identificatie:
\end{itemize}
\section{OPCUA:}
\subsection{Industry 3.0:}
\begin{itemize}
    \item ERP (Enterprise resource planning)
    \item Executing system $\rightarrow$ Raw materials to finished goods
    \item Scada (Supervisory Control and Data Acquisition)
    \item Programmable logic controllers (PLCs) and programmable automation controllers (PACs)
\end{itemize}
\subsubsection{Industry 3.0 VS Industry 4.0:}\begin{itemize}
    \item Interconnectivity
    \item Better collection and management of data 
    \item Data is available for all levels 
    \item Industrial Internet of Things IoT
\end{itemize}
\subsection{Industry 4.0:}
\begin{itemize}
    \item IoT 
    \item Calculations for assets 
    \item Monitor product quality 
    \item Batching 
    \item Big data and analytics 
    \item AI \& Machine learning
    \item Cyber physical systems 
    \item Cloud computing 
    \item Augmented and Virtual reality
\end{itemize}
\subsubsection{Goals and benefits}
\begin{itemize}
    \item Increased Efficiency: Streamlined operations and reduced downtime through real-
    time monitoring and predictive maintenance.
    \item Flexibility and Customization: Ability to quickly adapt to changing market
    demands and produce customized products without significant cost increases.
    \item Enhanced Quality: Improved precision and consistency in manufacturing
    processes, leading to higher-quality products.
    \item Cost Reduction: Lower operational costs through optimized resource utilization
    and reduced waste.
    \item Better Supply Chain Management: Enhanced visibility and coordination across the
    entire supply chain, leading to more reliable and responsive logistics.
    \item Innovation Acceleration: Fosters a culture of continuous improvement and
    technological advancement.
\end{itemize}
\subsubsection{Applications and examples:}
\begin{itemize}
    \item Smart Factories: Facilities where machines and systems communicate and
    cooperate with each other and with humans in real-time.
    \item Predictive Maintenance: Using sensor data and analytics to predict
    equipment failures before they occur, minimizing downtime.
    \item Digital Twins: Creating virtual replicas of physical assets to simulate and
    analyze performance under various conditions.
    \item Automated Warehousing: Employing robotics and AI to manage inventory,
    order fulfillment, and logistics with minimal human intervention
\end{itemize}
\paragraph{Examples:}
\begin{itemize}
    \item Watch versus smart watch\begin{itemize}
        \item Just a simple device that told time.
        \item Now it's a IoT device\begin{itemize}
            \item Tracking fitness
            \item Collecting health oriented data
            \item Messages
            \item \dots
        \end{itemize}
    \end{itemize}
\end{itemize}
\subsubsection{Why we have to deal with industry 4.0?}
\begin{itemize}
    \item Before Industry 4.0:\begin{itemize}
        \item Engineers controlled level 1 till level 3
        \item It controlled from level 3
    \end{itemize}
    \item Industry 4.0:\begin{itemize}
        \item Engineers, data-scientists and IT guys have to work together
        \item The common concepts will grow in the future
        \item IoT security
    \end{itemize}
\end{itemize}
\paragraph{Smartsensor, IIot sensor:}
\begin{itemize}
    \item Inverter, calibration, ADC, microcontroller,
    communication protocols,.. built into the sensor.
    \item Possibly IoT gateway also built in.
\end{itemize}
\subsubsection{PLC (programmable logic controllers):}
\begin{itemize}
    \item Industrial computer 
    \item High reliable 
    \item Ruggedized (Harsh conditions):\begin{itemize}
        \item Such as strong vibrations 
        \item Extreme temperatures 
        \item wet or dusty conditions
    \end{itemize}
    \item Real time system 
    \item Lot of in-and ouputs
    \item Flexible 
    \item Can automate a production line
\end{itemize}
\begin{itemize}
    \item Logic:\begin{itemize}
        \item Replace relays
        \item Ladder diagram
    \end{itemize}
    \item Input/output $\rightarrow$ 24V DC
    \item Extensible with:\begin{itemize}
        \item Extra I/O 
        \item Communication cards
    \end{itemize}
\end{itemize}
\subsubsection{How to connect with Industry 4.0}
\begin{itemize}
    \item OPC-server (Open Platform Communication)
    \item OPC foundation: founded in 1994
    \item Members are global players in the automation industry. (Siemens, Honeywell, Microsoft, Beckhoff,...)
    \item with OPC:\begin{itemize}
        \item Communication protocol 
        \item Access to machines and devices is standardized.
        \item Manufacturer-independent
        \item Data exchange
    \end{itemize}
\end{itemize}
\paragraph{Before and after OPC:}
Voor OPC moest je zoeken voor de juiste drivers voor elke applicatie/device.
\subsubsection{OPC server:}
\begin{itemize}
    \item Basis of OPC communcication
    \item It's a software that implements the OPC standard
    \item Provided by different parties:\begin{itemize}
        \item Manufacture provides OPC server for its system.
        \item Embedded OPC server (PLC) or stand-alone
    \end{itemize}
    \item Independent OPC server:\begin{itemize}
        \item Broader support of communication protocols.
        \item Offer more functions.
        \item Simpler operation.
        \item Certificate management.
        \item Manufactures of independent OPC:\begin{itemize}
            \item Kepware (over 150 drivers)
            \item Matrikon 
            \item Softing 
            \item Installed at PLC's
            \item \dots
        \end{itemize}
        \item Define Multiple Endpoints
        \item Multiple protocols: opc.tcp, http, and https
        \item User Definable Ports
        \item Definable Security \& User Token Policies
        \item Diagnostics
        \item User configurable operating limits
        \item Local Discovery Server Registration
        \item Configurable HTTP Access Rules and SSL/TLS
        Bindings
        \item Certificate Generator or use your self-signed
        certificate or from a Certificate Authority (CA)
    \end{itemize}
\end{itemize}
\subsubsection{OPC client:}
\begin{itemize}
    \item Can connect with OPC server
    \item Read out the data provided by the server
    \item Pushes data to the server
    \item OPC UA discovery
    \item Any client can connect with any server
    \item Examples of clients :\begin{itemize}
        \item UAExpert (Unified Automation)
        \item Visualisations and SCADA systems
        (supervisory control and data acquisition)
        \item manufacturing execution system (MES)
        \item PLC
    \end{itemize}
\end{itemize}
\subsubsection{OPC UA (Unified architecture):}
\begin{itemize}
    \item Platform independent 
    \item Succesor of OPC classic 
    \item Leading communication protocol for industry 4.0 
    \item OPC UA specifications:\begin{itemize}
        \item Data acces:\begin{itemize}
            \item Read/write 
            \item Data point-oriented 
        \end{itemize}
        \item Historical Acces:\begin{itemize}
            \item Query historical values 
            \item OPC server must have an internal data memory
        \end{itemize}
        \item Alarms and conditions:\begin{itemize}
            \item Logic can be implemented by the OPC server
        \end{itemize}
    \end{itemize}
\end{itemize}
\paragraph{Benefits in OPC UA:}
\begin{itemize}
    \item Enhanced Security: Built-in security features make OPC UA suitable for industrial
    environments where data protection is critical.
    \item Interoperability and Vendor Neutrality: OPC UA enables seamless interaction
    between devices from different vendors, improving flexibility in system design and
    reducing vendor lock-in.
    \item Real-Time Data Access: Provides real-time data access for applications that require
    up-to-the-minute information for decision-making.
    \item Contextual Data Modeling: With structured data, OPC UA allows for complex data
    models that provide context, making data easier to interpret and analyze.
    \item Ease of Integration with IoT and Industry 4.0 Solutions: OPC UA is widely adopted as
    the standard communication protocol in IIoT architectures, simplifying integration
    with other Industry 4.0 technologies.
\end{itemize}
\paragraph{Security in OPC UA:}
\begin{itemize}
    \item “firewallfriendly” i.e. it can be controlled and steered
    via standard network techniques.
    \item Several protocols have been made available:\begin{itemize}
        \item Binary protocol
        \item SOAP (Simple Object Access Protocol) with HTTPS
    \end{itemize}
    \item Encryptions are used to secure the data during
    transmission.
    \item Use of certificates to authenticate the clients
\end{itemize}
\paragraph{OPC UA and industry 4.0:}
\begin{itemize}
    \item Leading communication protocol
    \item OPC UA over TSN:\begin{itemize}
        \item Time Sensitive Network
        \item Between PLC's
        \item TCP/IP and web protocols
    \end{itemize}
\end{itemize}
\end{document}
